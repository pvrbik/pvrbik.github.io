 \documentclass[10pt]{report}
\usepackage{rotating,graphics,psfrag,epsfig,amssymb,amsmath, subfigure, amsxtra, amsthm, color}
\oddsidemargin=-0.25in
\topmargin= -0.5in
\textwidth=7.0in
\textheight=9in
\pagestyle{empty}
\sloppy

\theoremstyle{plain}
\newtheorem{theorem}{Theorem}
\newtheorem{corollary}{Corollary}
\newtheorem{lemma}{Lemma}
\newtheorem{proposition}{Proposition}
\newtheorem{conjecture}{Conjecture}

\theoremstyle{definition}
\newtheorem{definition}{Definition}
\newtheorem{example}{Example}
\newtheorem{remark}{Remark}
\newtheorem{question}{Question}
\newtheorem{problem}{Problem}


\newcommand{\imp}{\textbf}
\newcommand{\dom}[1]{\textsf{dom} (#1)}
\newcommand{\cod}[1]{\textsf{cod}(#1)}
\newcommand{\arr}[1]{\mathcal{#1}\mathrm{-arrows}}
\newcommand{\obj}[1]{\mathcal{#1}\mathrm{-objects}}
\newcommand{\state}[1]{\item[#1] \textcolor{white}{.} \\ \textcolor{white}{.} \\ }
\newcommand{\Lstate}[1]{\item[#1] \nl}
\newcommand{\id}[1]{id_{#1}}
\newcommand{\nl}{\textcolor{white}{.}\ }
\newcommand{\func}[3]{#1 : #2 \rightarrow #3}
\newcommand{\CC}{\mathcal{C}}
\newcommand{\DD}{\mathcal{D}}
\newcommand{\point}[1]{\begin{itemize} \item[$\cdot$] #1 \end{itemize}}
\newcommand{\blank}{\textcolor{white}{-}}
\newcommand{\fieldt}{\mathbb{Z}_q[t]}
\newcommand{\fieldtx}{\mathbb{Z}_q[t][x]}
\newcommand{\impl}{\Rightarrow}
\newcommand{\eps}{\varepsilon}
\newcommand{\brac}[1]{\left( #1 \right)}
\newcommand{\Res}[1]{\textrm{Res}_{#1} }

\title{PROJECT}
\author{Paul Vrbik : 301056796 : MATH 819 : Spring 2007}
\date{\today}

\begin{document}

\maketitle

\section*{The modular GCD algorithm and Hensel lifting algorithm}

\subsection*{Question 1}

%chinese remainder
\begin{description}
\state{Chinese Remainder Theorem for $\fieldt$}
For coprime polynomials $[m_1(t), m_2(t), ..., m_N(t)]$ where $M= {\prod m_k}$. The set of congruences
$$ c(t) \equiv c_k(t) \mod m_k(t) $$
has unique solution satisfying $\deg{f}  < \deg{M}$ given by
\begin{equation} \label{crt} c(t) = \sum_{k=1}^N \bigr( c_k(t)\cdot N_k(t)\cdot M_k(t) \bigr) \mod M(t) \end{equation}
where $$ M_k(t) = \frac {M(t)} {m_k(t)} $$and $N_k(t)$ satisfies
$$N_k(t) \cdot M_k(t) + n_k(t) \cdot m_k(t) = 1.$$ That is, $N_k(t)$ is the inverses of $M_k(t) \mod m_k(t)$.

\state{Proof}
We first show that  $c(t)$ satisfies $c(t) \equiv c_k(t) \mod m_k(t)$ for any $k \in [1,N]$. Denote $s_i = c_i(t) \cdot N_i(t) \cdot M_i(t)$ and rewrite the summand in ~\eqref{crt} as
$$ c(t) = \sum_{i=1}^N s_i \mod M. $$

Observing that $M_i(t) \equiv 0 \mod m_i \Rightarrow s_i \equiv 0 \mod m_i$ for $i \neq k$ and that $N_i(t)$ always exists since $\mathbb{Z}_q[t]$ is a Euclidean domain, we simplify

\begin{align*}
c(t) 	{}& \equiv \sum_{i=1}^N s_i \mod M. \\
	{}& \equiv s_k \mod M \\
	{}& \equiv c_i(t) \cdot \left(  \frac {M(t)} {m_i(t)} \right)^{-1} \cdot \left( \frac {M(t)} {m_i(t)} \right) \mod m_i(t) \\
	{}& \equiv c_i(t) \mod m_i(t).
\end{align*}

To show that this solution is unique suppose we have an alternate solution $c^\prime$ such that $c^\prime \not\equiv c \mod M$ satisfying $c^\prime (t) \equiv c_k \mod m_k(t)$ for any $k$.

\begin{align*}
	{}& \impl c(t)-c^\prime(t) \equiv 0 \mod m_k(t) \\
	{}& \impl m_i(t) \mid c(t) - c^\prime (t)
\end{align*}

Since we have unique factorization  and relatively prime $m_i$'s,  their product most also divide $c(t) - c^\prime(t)$ .

\begin{align*}
	{}& \impl M(t) \mid c(t) - c^\prime(t) \\
	{}& \impl c(t) - c^\prime(t) \equiv 0 \mod M\\
	{}& \impl c(t) \equiv c^\prime(t) \mod M
\end{align*}

This contradicts the assumption that $c^\prime \not\equiv c \mod M$ implying the uniqueness of $c(t)$.
$\hfill \Box$

\end{description}
%chinese remainder

\subsubsection*{Implementation}

\begin{verbatim}
CRT:=(As,fs,x,Q)->h_CRT(As,fs,x,Q,mul(fs[i], i=1..nops(fs))):

h_CRT:=proc(As,fs,x,Q,M) local v1,v2,Abar,`&pmod`:
     
     `&pmod`:=(a,b)-> simplify(Rem(a,b,x) mod Q):
     
    if (nops(As)=1) then
        return As[1] &pmod fs[1]:
    end if:
    
    Abar:=CRT(As[1..-2], fs[1..-2],x,Q):
    
    Gcdex(fs[-1],simplify(M/fs[-1]),x,'f1Inv') mod Q:
    v1:=fs[-1]*f1Inv:
    
    Gcdex(simplify(M/fs[-1]),fs[-1],x,'f2Inv') mod Q:
    v2:=M/fs[-1]*f2Inv:
    
    v1:=simplify(v1):
    v2:=simplify(v2):
    
    return (Abar*v1 + As[-1]*v2) &pmod M:

end proc:
\end{verbatim}

\subsubsection*{Testing}

\begin{verbatim}
> restart;
> read "CRT.mpl";
                                       2   2
                               cs := [t , t  + t + 1]

                                  3           3    2
                          ms := [t  + t + 1, t  + t  + 1]

                                       q := 2

> c:=CRT(cs,ms,t,q);   
                                   5    4    3    2
                             c := t  + t  + t  + t  + t

> Rem(c,ms[1],t) mod q;
                                          2
                                         t

> Rem(c,ms[2],t) mod q;
                                      2
                                     t  + t + 1

\end{verbatim}

\clearpage
\subsection*{Question 2}
\subsubsection*{The Modular GCD Algorithm for $\mathbb{Z}_p[t][x]$ }

\begin{verbatim}
fModGcd:=proc(inA,inB,q)
local Phi,a, b, contA, contB, contG, gam, G, M, p, lcA, ap, bp, gp, u, g:
    
    read "procs.mpl": read "CRT.mpl":
    
    `mod`:=mods;
    
    a:=inA:
    b:=inB:
    
    contA:=Content(a,x,'ppa') mod q;
    contB:=Content(b,x,'ppb') mod q;
    contG:=myGcd(contA,contB,q);
    
#   Inexplicably ppa and ppb are assigned 1 I was unable to resolve this bug.
    
    a:=Primpart(a,x) mod q;
    b:=Primpart(b,x) mod q;
    
    gam:=myGcd(lcoeff(a,x), lcoeff(b,x),q);
    G:=0:
    M:=1:
    
    p:=Nextprime(t^2,t) mod q;
    
    lcA:=lcoeff(a,x):
    
    while (true) do
        
        while (Rem(lcA,p,t) mod q = 0) do 
            print("BAD PRIME",p):
            p:=NextMonicPrime(p,t,q):
        end do:
        
        Phi:=x->Rem(x,p,t) mod q:
        
        ap:=Phi(a):
        bp:=Phi(b):
        gp:=fgcd(ap,bp,p,q):
        
        if degree(gp,x)=0 then 
            return contG:
        end if:
        
#       correct gp
        gp:=Phi(gam*gp):
        gp:=Phi(gp,p,q):
        
        if G=0 then
            G:=gp:
            M:=p:
        elif degree(gp,x) > degree(G,x) then
#           do nothing (unlucky prime)
        elif degree(gp,x) < degree(G,x) then
            G:=gp:
            M:=p:
        else
            u:=CRT([G,gp],[M,p],t,q):
            
            if Expand(u-G) mod q=0 then
                g:=Primpart(contG*u,x) mod q:
                
                if Rem(a,g,t) mod q=0 and Rem(b,g,t) mod q =0 then
                    return g;
                end if:
            end if:
            G:=u:
            M:=M*p:
        end if:
        
        p:=NextMonicPrime(p,t,q):
        
    end do:
    
end proc:
\end{verbatim}

\subsection*{\tt{procs.mpl}}
\begin{verbatim}
myGcd:=proc(a,b,p) local ca,cb,cg:
    ca:=Content(a,t,'ppa') mod p:
    cb:=Content(b,t,'ppb') mod p:
    cg:=Gcd(ca,cb) mod p:
    return Expand(cg*(Gcd(ppa,ppb) mod p)) mod p:
end proc:

fgcd:=proc(inF1,inF2,p,q) local g, f1, f2,r:
    f1:=inF1 mod q:
    f2:=inF2 mod q:
    r:=RootOf(p):
    g:=subs(r=t, (myGcd(subs(t=r,f1),subs(t=r,f2),q))):
    return Expand(g) mod q:
end proc:

NextMonicPrime:=proc(p,t,q) local np:
    np:=Nextprime(p,t) mod q:
    if lcoeff(np,t)=1 then 
        return np:
    else
        return Nextprime(t^(degree(np,t)+1),t) mod q:
    end if
end proc:
\end{verbatim}

\clearpage
\subsubsection*{Testing}

\begin{verbatim}
> restart;
> read "fModGcd.mpl";
                                       q := 3

                            3       5    11  3    7      9
                     g := (t  - t) x  - t   x  + t  x + t  + 1

                                         5    6  2
                              abar := t x  - t  x  + 1

                                          4    2    7
                               bbar := t x  + x  + t

                    5    6  2         3       5    11  3    7      9
           a := (t x  - t  x  + 1) ((t  - t) x  - t   x  + t  x + t  + 1)

                     4    2    7     3       5    11  3    7      9
            b := (t x  + x  + t ) ((t  - t) x  - t   x  + t  x + t  + 1)

> fModGcd(a,b,q);
                        5  3      5    11  3    7      9
                       x  t  - t x  - t   x  + t  x + t  + 1

> -myGcd(a,b,q);
                        5  3      5    11  3    7      9
                       x  t  - t x  - t   x  + t  x + t  + 1
\end{verbatim}

\clearpage
\subsection*{Question 3}
\subsubsection*{Hensel Lifting for $\mathbb{Z}_q[t][x]$}


\begin{verbatim}
UHLA:=proc(inA,q,p,inU0,inW0,B,inGam)
local a, u0, w0, Phi, alpha, gam, rof, ss, tt, u, w, err, modulus, 
c, sigmabar, taubar, rr, qq, sigma, tau, t1:
    
    read "common.mpl";
    
    `mod`:=mods:
    a:=inA; u0:=inU0; w0:=inW0;
    
    Phi:=x-> Rem(x,p,t) mod q:
    
    alpha:=lcoeff(a,x):
    
    if nargs<7 then
        gam:=alpha:
    else
        gam:=inGam:
    end if:
    
    a:=gam*a:
    
    u0 := Phi(gam*Monic(u0,p,q),p,q):
    w0 := Phi(gam*Monic(w0,p,q),p,q):
    
    rof:=RootOf(p):
    Gcdex(subs(t=rof,u0), subs(t=rof,w0), x, 'ss', 'tt') mod q:
    ss := subs(rof=t, ss):
    tt := subs(rof=t, tt):
    
    u:=replace_lc(u0,gam):
    
    w:=replace_lc(w0,alpha):
    
    err:=Expand(a-u*w) mod q:
    modulus:=p:
    err:=Quo(err,modulus,x,'rr') mod q;
    
    while (err <> 0 and degree(modulus,t)<B*degree(gam,t)+1) do
        
        sigmabar:=Phi(ss*err);
        taubar:=Phi(tt*err);
        
        rr:=subs(rof=t, Rem( subs(t=rof,sigmabar), subs(t=rof,w0), x, 'qq') mod q);
        qq:=subs(rof=t, qq);
        
        sigma:=rr;
        tau:=Phi(Expand(taubar+qq*u0) mod q);
        
        err:=Expand(err-(u*sigma+tau*w)-sigma*tau*modulus) mod q;
        err:=Quo(err,p,t,'rr') mod q;
        
        u:=Expand(u+tau*modulus) mod q;
        w:=Expand(w+sigma*modulus) mod q;
        
        modulus:=modulus*p;
        
    end do;
    
    if err=0 then
        sigma:=Content(u,x) mod q;
        u:=Quo(u,sigma,x) mod q;
        t1:=Quo(gam,sigma,x) mod q:
        w:=Quo(w,t1,x) mod q;
        return (u,w);
    else
        return "FAIL";
    end if:
    
end proc:
\end{verbatim}

\subsection*{\tt{common.mpl}}
\begin{verbatim}
locPhi := proc(a, p, q)
    Rem(a, p, t) mod q;
end:

Monic := proc(a, p, q)
local lcA, lcAinv, rof, g;
    lcA := lcoeff(a, x);
    g:=Gcdex(lcA, p, t, 'lcAinv') mod q;
    lcAinv := subs(rof=t, lcAinv);
    return collect(locPhi(a*lcAinv,p,q),x);
end:

replace_lc := proc(f,a) local n,lc:
    n:=degree(f,x):
    lc:=lcoeff(f,x):
    return (f-lc*x^n+a*x^n):
end proc:

fgcd:=proc(inF1,inF2,p,q) local g, f1, f2,r:
    f1:=inF1 mod q:
    f2:=inF2 mod q:
    r:=RootOf(p):
    g:=subs(r=t, (myGcd(subs(t=r,f1),subs(t=r,f2),q))):
    return Expand(g) mod q:
end proc:

myGcd:=proc(a,b,p) local ca,cb,cg:
    ca:=Content(a,t,'ppa') mod p:
    cb:=Content(b,t,'ppb') mod p:
    cg:=Gcd(ca,cb) mod p:
    return Expand(cg*(Gcd(ppa,ppb) mod p)) mod p:
end proc:
\end{verbatim}

\clearpage
\subsubsection*{Testing}
\begin{verbatim}
> read "UHLA.mpl";
                                       q := 3

                                       3
                                 p := t  + 2 t + 2

                          Phi2 := x -> Rem(x, p, t) mod q

                            3       5    11  3    7      9
                     u := (t  - t) x  - t   x  + t  x + t  + 1

                                       5    6  2
                               w := t x  - t  x  + 1

                   3       5    11  3    7      9          5    6  2
           a := ((t  - t) x  - t   x  + t  x + t  + 1) (t x  - t  x  + 1)

                     5      3                     3        3         2
              u0 := x  + 2 x  + x + (2 x + 1 + 2 x ) t + (x  + 2 x) t

                                    2     5    2         2  2
                       w0 := 1 + 2 x  + (x  + x ) t + 2 x  t

> fgcd(u0,w0,p,q);
                                         1

> Phi2(a-u0*w0);           
                                         0

> #B=17 since it is the largest degree in t of the coefficients of a;
> (ut,wt):=UHLA(a,q,p,u0,w0,17);
                      3       5    11  3    7      9         5    6  2
          ut, wt := (t  - t) x  - t   x  + t  x + t  + 1, t x  - t  x  + 1

> Expand(a-ut*wt) mod q;
                                         0

\end{verbatim}
\clearpage
\subsection*{Question 4}

To be more efficient we determine an update formula for
$$ \frac {\eps_k} {p^k} $$
where $\eps_k$ is the error on the $k$th iteration.

\begin{align}
\brac{\frac {\eps_{k+1}} {p^{k+1}}} {}& = \frac {\eps_k - (u^{(k)}\cdot w_k + u_k \cdot w^{(k)})\cdot p^k-u_k \cdot w_k \cdot p^{2k}} {p^{k+1}} \\
	{}& = \brac{ \frac {\eps_k} {p^k} } \cdot \frac {1} {p} - \brac{u^{(k)}\cdot w_k + u_k \cdot w^{(k)}} \cdot \frac {1} {p} - \brac{u_k \cdot w_k \cdot p^k} \cdot \frac {1} {p} \\
	{}& = \brac{ \frac {\eps_k} {p^k}  - \brac{u^{(k)}\cdot w_k + u_k \cdot w^{(k)}} - \brac{u_k \cdot w_k \cdot p^k} } \cdot \frac {1} {p}
\end{align} 

Assuming the classical algorithms for polynomial multiplication and division in $\fieldt$, if $\deg_x(a)=n$ and $\deg_t(a)=m$ then calculating the sum in (4) costs
	$$\sum_{k=1}^m O(n^2k) \in O(n^2m^2) $$

and division by $p$ is $O(n^2m)$. This implies the total cost of the error calculation is reduced to $O(n^2m^2)$ as desired. This change is reflected in the implementation given in three.



\clearpage
\subsection*{Question 5}

\begin{description}
\item[MODGcd Algorithm] An \emph{unlucky prime} is one that satisfies (in $\fieldt$)
$$ \gcd \brac{ \phi_p (\bar{a}), \phi_p (\bar{b})} \neq 1 $$
or
$$\Res{x}(\bar{a}, \bar{b}) \equiv 0 \mod p$$
 
For question two where $$\Res{x}(\bar{a}, \bar{ b}) = t^{41}+t^{39} + t^{31} + 2t^{18} + t^5 + t^2$$ these primes are given below :
\end{description}

\point{$p=t$}
\point{$p=2+t+t^{31}+t^{18}+t^2+2t^7+2t^9+t^{10}+2t^{11}+t^{12}+2t^{13}+t^{15}+t^{16}+2t^{17}+t^{20}+t^{21}+t^{22}+t^{23}+2t^{24}+2t^{25}+2t^{27}+t^{29}+t^{30}+2t^{32}+t^{33}+t^{34}$}
\point{$p=t^3+2t^2+2t+2$}
\point{$p=t^2+1$}
\point{$p=t^2+2t+3$}
Are all unlucky since $\Res{x}(\bar{a}, \bar{b})$ is divisible by p.

\hfill
\begin{description}
\item[Hensel Lifting] An \emph{unlucky prime} is one that satisfies (in $\fieldt$)
$$\Res{x}(u,w) \equiv 0 \mod p$$
 
For $u=(t^3-t)x^5-t^{11}x^3+t^7x+t^9+1$ and $w=(tx^5-t^6x^2+1)$ where 

\begin{align*}
	\Res{x}(u,w)= 	{}& t^9+2t^7+t^{11}+t^5+2t^{13}+t^{14}+t^{16}+t^{15}+2t^{62}+t^{52}+t^{53} \\
				{}& +2t^{61}+2t^{59}+t^{55}+2t^{49}+t^{45}+2t^{46}+t^{20}+t^{31}+t^{27}+t^{22}+2t^{21} \\
				{}& +t^{35}+t^{23}+t^{51}+2t^{42}+t^{39}+t^{32}+t^{71}+2t^{34}+t^{43}+2t^{40}+t^{50} 
\end{align*}
these primes are given below:
\end{description}

\point{$p=t$}
\point{$p=t^{14}+t^{13}=2t^{11}+t^{9}+t^6+t^5+2t^4+t^2+t+2$}
\point{$p=t^{11}+t^{10}+2t^9+2t^7+2t^5+t^4+t^2+2t+2$}
\point{$p=t^5+t^4+2$}
\point{$p=t^{31}+t^{30}+t^{29}+2t^{27}+t^{26}+2t^{25}+t^{23}+2t^{21}+2t^{20}+2t^{18}+2t^{16}+t^{15}+2t^{14}+2t^{12}+2t^{10}+t^8+2t^6+2t^4+t^3+t^2+1$}
\point{$p=t^2+2t+2$}
\point{$p=t^3+2t+1$}

Are all unlucky since $\Res{x}(u, w)$ is divisible by p.

\clearpage
For $u=(t^3-t)x^5-t^{11}x^3+t^7x+t^9+1$ and $w=(tx^4-x^2+t^7)$ where 
\begin{align*}
	\Res{x}(u,w)=	{}& 2t^{11}+t^2+2t^{10}+t^4+2t^{33}+2t^{22}+t^{15}+t^{31}+t^{23}+2t^{14} \\
				{}& +2t^{44}+2t^{35}+t^{25}+t^{32}+t^{21}+t^6+t^{36}+2t^{45}+t^{47}+t^5 \\
				{}& +t^{40}+t^{67}+2t^{49}+2t^{29}+t^{27}+2t^{58}+2t^{37}+2t^{28}+2t^{48} \\
				{}& +2t^{39}+2t^{30}+t^{46}+2t^{57}+t^{55}+2t^{53}+t^{42}+t^{19}
\end{align*}
these primes are given below:

\point{$p=t$}
\point{$p=t^4+t^3+t^2+2t+2$}
\point{$p=t^8+2t^5+2t^3+t^2+2$}
\point{$p=t^3+2t+1$}
\point{$p=1+t+2t^3+t^7+2t^{11}+t^2+2t^8+2t^{13}+2t^4+2t^{24}+t^{15}+2t^{31}+2t^{23}+2t^{14}+2t^{13}+2t^{17}+t^{32}+2t^{16}+t^{21}+2t^6+2t^{18}+2t^5+2t^{29}+2t^{27}+2t^{28}$ }
\point{$p=t^{18}+t^{16}+2t^{13}+t^{12}+t^{11}+t^{10}+t^9+t^8+2t^7+t^4+2t^3+t+1$}

Are all unlucky since $\Res{x}(u, w)$ is divisible by p.
\end{document}