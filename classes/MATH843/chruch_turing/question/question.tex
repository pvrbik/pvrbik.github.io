 \documentclass[12pt]{report}
\usepackage{rotating,graphics,psfrag,epsfig,amssymb,amsmath, subfigure}
\oddsidemargin=-0.25in
\topmargin= -0.5in
\textwidth=7.0in
\textheight=9in
\pagestyle{empty}
\sloppy

\begin{document}

\section*{Questions on computability}

\begin{description}
\item[Computation] A computation is an operation that begins with some initial conditions and gives an output which follows from a definite set of rules. The most common example are computations performed by computers, in which the fixed set of rules may be the functions provided by a particular programming language. (We will later define computation as something that can be modeled by a Universal Turing machine.) 
\end{description}

\subsection*{Questions}

\begin{enumerate}

\item Is hardware or software the basis for computation?

\item Suppose you have a computer program that can evaluate a predicate $P$. Can you think of a predicate such that the program will either return \emph{true} or never stop? \emph{false} or never stop? never stop?

\item What does it mean for something to be uncomputable? i.e. does it mean that something just takes to long to compute or is it something deeper then this?

\item What is an algorithm?


\end{enumerate}
 

\end{document}