\documentclass[fleqn]{seminar}
\nonstopmode

% \usepackage{geometry}
% \geometry{letterpaper,landscape}
% \usepackage{pictex}
%  pictex causes me to run out of dimensions (no room for new #3)
\usepackage{pictex}
\usepackage{dcpic}
\usepackage{semhelv}
\usepackage{color}
\usepackage{alltt}
\usepackage{latexsym}
\usepackage{amsfonts}
\usepackage{amsmath}
\usepackage{calc}
\usepackage{graphicx}
\usepackage{pifont}

\definecolor{CodeColor}{named}{BlueViolet}
% \definecolor{SystemColor}{named}{Mulberry}
% \definecolor{SystemColor}{named}{OliveGreen}
\definecolor{SystemColor}{named}{RedOrange}
\definecolor{MathColor}{named}{Magenta}
\definecolor{QuoteColor}{named}{Salmon}
\definecolor{LawColor}{named}{Green}
\definecolor{DefinitionColor}{named}{Red}
\definecolor{AlgorithmColor}{named}{RawSienna}
\definecolor{BackgroundColor}{named}{Yellow}
\definecolor{OtherColor}{named}{PineGreen} % {Maroon}
\definecolor{Green}{named}{OliveGreen}
% \definecolor{BackgroundColor}{cmyk}{0,0,0.6,0}

\newcommand{\semester}{Fall 2006}

\newcommand{\bs}{\ensuremath{\backslash}}
\newcommand{\zero}{\textcolor{MathColor}{\ensuremath{\mathbf{0}}~}}
\newcommand{\one}{\textcolor{MathColor}{\ensuremath{\mathbf{1}}~}}

\newcommand{\law}[1]{\textcolor{black}{(\textcolor{LawColor}{#1})}}

\newcommand{\sublaw}[1]{\textcolor{black}{(\textcolor{black}{#1})}}

\newcommand{\reason}[1]{\hspace{0.15in}
    \meq\hspace{0.075in}\law{#1}}

\newcommand{\subreason}[1]{\hspace{0.15in}
    \meq\hspace{0.075in}\sublaw{#1}}

\newcommand{\definition}[1]{\hspace{0.05in}
    \textcolor{DefinitionColor}{\textit{#1}\hspace{0.05in}}}

\newcommand{\meq}{\textcolor{MathColor}{\ensuremath{\equiv}}}
\newcommand{\bottom}{\textcolor{MathColor}{\ensuremath{\bot}}}
\newcommand{\dom}{\textcolor{MathColor}{\ensuremath{\mathit{dom~}}}}
\newcommand{\cod}{\textcolor{MathColor}{\ensuremath{\mathit{cod~}}}}
\newcommand{\rarrow}{\textcolor{MathColor}{\ensuremath{\rightarrow}}}

\newcommand{\textcd}[2][CodeColor]{\textcolor{#1}{\texttt{#2}}}
\newcommand{\tc}[2][CodeColor]{\textcolor{#1}{\texttt{#2}}}
\newcommand{\tcx}[2][OtherColor]{\textcolor{#1}{\texttt{#2}}}
\newcommand{\ts}[2][SystemColor]{\textcolor{#1}{\texttt{#2}}}
\newcommand{\textcdl}[2][CodeColor]{\textcolor{#1}{\hspace{0.1in}\texttt{#2}}}
\newcommand{\textcdr}[2][CodeColor]{\textcolor{#1}{\texttt{#2}\hspace{0.1in}}}
\newcommand{\textcdlr}[2][CodeColor]{\textcolor{#1}
{\hspace{0.1in}\texttt{#2}\hspace{0.1in}}}

\newcommand{\tmm}[1]{\textcolor{AlgorithmColor}{\ensuremath{#1}}}


\newcommand{\textmath}[1]{\textcolor{MathColor}{\ensuremath{#1}}}
\newcommand{\tm}[1]{\textcolor{MathColor}{\ensuremath{#1}}}
\newcommand{\tmcc}[1]{\textcolor{CodeColor}{\ensuremath{#1}}}
\newcommand{\tmtt}[1]{\textcolor{MathColor}{\texttt{#1}}}
\newcommand{\unmath}[1]{\textcolor{black}{\mathrm{#1}}}
\newcommand{\tmi}[1]{\textcolor{MathColor}{\ensuremath{\mathit{#1}}}}
\newcommand{\tmr}[1]{\textcolor{MathColor}{\ensuremath{\mathrm{#1}}}}
\newcommand{\tmb}[1]{\textcolor{MathColor}{\ensuremath{\mathbf{#1}}}}


\newcommand{\tbr}[1]{\textcolor{red}{\textbf{#1}}}

\newcommand{\bla}[1]{\textcolor{black}{#1}}
\newcommand{\gre}[1]{\textcolor{Green}{#1}}
\newcommand{\red}[1]{\textcolor{red}{#1}}
\newcommand{\blu}[1]{\textcolor{blue}{#1}}
\newtheorem{lemma}{Lemma}
\newtheorem{theorem}{Theorem}
\newtheorem{corollary}{Corollary}
\newenvironment{proof}{\textbf{Proof:}}{\hfill $\Box$}

\slideframe{none}

\centerslidesfalse

\newlength{\tempwidth}

\newenvironment{cdiag}
   {\vspace{0.25in}
    \begin{center}
    \setlength{\tempwidth}{\linewidth}
    \addtolength{\tempwidth}{-2\columnsep}
    \begin{minipage}[t]{\tempwidth}
    \begindc{\commdiag}[10]
   }
   {\enddc
    \end{minipage}
    \end{center}
    \vspace{0.25in}
   }

\newenvironment{cdiag2}
   {\vspace{0.25in}
    % \begin{center}
    \setlength{\tempwidth}{\linewidth/2}
    \begin{minipage}[t]{\tempwidth}
    \begindc{\commdiag}[10]
   }
   {\enddc
    \end{minipage}
    % \end{center}
    \vspace{0.25in}
   }

\newenvironment{udg}
   {\vspace{0.25in}
    \begin{center}
    \setlength{\tempwidth}{\linewidth}
    \addtolength{\tempwidth}{-2\columnsep}
    \begin{minipage}[t]{\tempwidth}
    \begindc{2}[10]
   }
   {\enddc
    \end{minipage}
    \end{center}
    \vspace{0.25in}
   }

\newenvironment{codenott}[1][CodeColor]
   {\color{#1}
    % \begin{center}
    \setlength{\tempwidth}{\linewidth}
    \addtolength{\tempwidth}{-2\columnsep}
    \hspace*{0.25in}
    \begin{minipage}[t]{\tempwidth}
    \begin{tabbing}
    \hspace{0.075in} \= \hspace{0.075in} \= \hspace{0.075in} \= \hspace{0.075in} \= \hspace{0.075in} \= \kill
   }
   {\end{tabbing}
    \end{minipage}
    % \end{center}
    \color{black}
   }

\newenvironment{codenott2}[1][CodeColor]
   {\color{#1}
    % \begin{center}
    \setlength{\tempwidth}{\linewidth/3}
    \hspace*{0.25in}
    \begin{minipage}[b]{\tempwidth}
    \begin{tabbing}
    \hspace{0.075in} \= \hspace{0.075in} \= \hspace{0.075in} \= \hspace{0.075in} \= \hspace{0.075in} \= \kill
   }
   {\end{tabbing}
    \end{minipage}
    % \end{center}
    \color{black}
   }

\newenvironment{laws}
   {\setlength{\tempwidth}{\linewidth}
    \addtolength{\tempwidth}{-2\columnsep}
    \hspace*{0.25in}
    \begin{minipage}[t]{\tempwidth}
    \begin{tabular}{lr}
   }
   {\end{tabular}
    \end{minipage}
   }

\newenvironment{code}[1][CodeColor]
   {\color{#1}
    % \begin{center}
    \setlength{\tempwidth}{\linewidth}
    \addtolength{\tempwidth}{-2\columnsep}
    \hspace*{0.25in}
    \begin{minipage}[t]{\tempwidth}
    \begin{alltt}
    \begin{tabbing}
    \hspace{0.075in} \= \hspace{0.075in} \= \hspace{0.075in} \= \hspace{0.075in} \= \hspace{0.075in} \= \kill
   }
   {\end{tabbing}
    \end{alltt}
    \end{minipage}
    % \end{center}
    \color{black}
   }

\newenvironment{mcode}[1][MathColor]
   {\color{#1}
    % \begin{center}
    \setlength{\tempwidth}{\linewidth}
    \addtolength{\tempwidth}{-2\columnsep}
    \hspace*{0.25in}
    \begin{minipage}[t]{\tempwidth}
    \[\begin{array}{lrll}}
   {\end{array}\]
    \end{minipage}
    % \end{center}
    \color{black}
   }

\newcommand{\mF}{\textcolor{MathColor}{\ensuremath{\tc{fmap}_{F}~}}}
\newcommand{\Id}{\textcolor{MathColor}{\ensuremath{\mathit{Id}}}}
\newcommand{\nt}{\textcolor{MathColor}{\ensuremath{\stackrel{.}{\rightarrow}}}}
\newcommand{\List}{\textcolor{MathColor}{\ensuremath{\tc{[}?\tc{]}}}}
\newcommand{\LL}{\textcolor{MathColor}{\ensuremath{\tc{[[}?\tc{]]}}}}
\newcommand{\LA}{\textcolor{MathColor}{\ensuremath{\tc{[}A\tc{]}}}}
\newcommand{\LB}{\textcolor{MathColor}{\ensuremath{\tc{[}B\tc{]}}}}
\newcommand{\LLA}{\textcolor{MathColor}{\ensuremath{\tc{[[}A\tc{]]}}}}
\newcommand{\LLB}{\textcolor{MathColor}{\ensuremath{\tc{[[}B\tc{]]}}}}
\newcommand{\lam}{\textcolor{CodeColor}{\ensuremath{\lambda}}}
\newcommand{\redu}{\textcolor{MathColor}{\ensuremath{\longrightarrow}}}
\newcommand{\Has}{\textit{Haskell}\,}

\newcommand{\ack}{I would like to thank Tai Meng for many corrections
and suggestions for improvements to these notes.  Of course, the remaining
errors are my fault.}


\newpagestyle{lectures}{\makebox[1in][l]{CMPT 481/731} \hfill Functional Programming
\hfill \makebox[1in][r]{\semester}}{\makebox[2in][l]{The Lambda Calculus} \hfill F. Warren
Burton \hfill \makebox[2in][r]{\thepage}}

% \newpagestyle{lectures}{CMPT 481/731 \hfill Functional Programming
% \hfill \semester}{The Lambda Calculus \hfill
% F. Warren Burton \hfill \thepage}

\pagestyle{lectures}

\begin{document}
\begin{slide}

\begin{center}
\begin{LARGE}
\textbf{The Lambda Calculus}
\end{LARGE}\\[3ex]
\end{center}

The theoretical foundations of funuctional programming include:
\begin{itemize}
\item The lambda calculus,
\item Domain theory,
\item Category theory, and
\item Type theory.
\end{itemize}

We will briefly consider aspects of these theoretical foundations as well as
more practical aspects of functional programming.  Our aim is to study the
theory to the extent that it helps us with the practical aspects of functional programming.

The lambda calculus is a notation and associated Turing complete model of computation.

\newslide

\section*{Syntax}

The abstract syntax of the lambda calculus is:

%\color{MathColor}
%\[\begin{array}{lrll}
\begin{mcode}
<\mathit{exp}> & ::= & <\mathit{var}>\\
&  | & <\mathit{exp}> <\mathit{exp}> &
\unmath{\color{DefinitionColor}{Application}}\\
&  | & \lam ~ <\mathit{var}>\tc{.}<\mathit{exp}> &
\unmath{\color{DefinitionColor}{Abstraction}}\\
<\mathit{var}> & ::= & \unmath{Some~countably~infinite~set}
\end{mcode}
%\end{array}\]
%\color{black}

In the concrete syntax we allow parentheses.  In addition, to reduce
the number of parentheses, we use the following conventions:
\begin{itemize}
\item Application is left associative, and
\item Abstractions extend as far to the right as possible. 
\end{itemize}

\newslide

Sometimes the lambda calculus it defined to include constants as well as variables,
with the addition of the productions:
%\color{MathColor}
%\[\begin{array}{lrll}
\begin{mcode}
<\mathit{exp}> & ::= & <\mathit{const}> \ldots\\
<\mathit{const}> & ::= & \ldots
\end{mcode}
%\end{array}\]
%\color{black}

For example, the constants might include integers and basic integer operations.

We will see, in section 3.7 of \textit{Bird}, that natural numbers can be represented by functions,
so the addition of constants is not necessary.  With just the basic lambda calculus we
can express any Turing computable function from natural numbers to natural numbers.

\newslide

The lambda notation lets us define functions without giving them names.

For example,
\tm{f ~ \mathrm{where} ~ f(x) = g (h (x))}
can be written as
\tc{\lam x.g (h x)}.

We can still name things when we want to.

For example,
\tm{\mathrm{let} ~ x = y ~ \mathrm{in} ~ f(x)}
can be written as
\tc{(\lam x.f x) y}.

Later on, we will informally mix our notation at times.

\newpage

\subsection*{Examples}

\begin{code}
(\lam a.\lam b.a) c\\
(\lam a.\lam b.a) c d\\
(\lam a.\lam b.b) c\\
(\lam a.\lam b.b) c d\\
(\lam a.\lam b.a) b\\
(\lam a.\lam b.a) b d\\
(\lam x.x x)(\lam x.x x)\\
(\lam x.x x x)(\lam x.x x x)\\
\lam h.(\lam x.h (x x))(\lam x.h (x x))\\
(\lam h.(\lam x.h (x x))(\lam x.h (x x))) f
\end{code}

\newslide

\section*{Reduction}

The process of ``evaluating'' a lambda expression is called \definition{reduction}.

In the following, we will use \tm{v}, \tm{w} and \tm{x} to denote arbitrary variables and
\tm{E}, \tm{E'}, \tm{E_{1}}, and \tm{E_{2}} to denote arbitrary expressions.
\newslide

\subsection*{Free Variables}

Informally, a variable is \definition{free} if it is not within the scope of a variable appearing after a
\lam ~ (that is, it is ``undefined'').

Formally, we define \tm{\mathit{FV}(E)}, the set of free variables in a lambda expression \tm{E}, by
%\color{MathColor}
%\[\begin{array}{l}
\begin{mcode}
\mathit{FV}(\tm{v}) = \{\tm{v}\}\\
\mathit{FV}(E_{1}E_{2}) = \mathit{FV}(E_{1}) \cup \mathit{FV}(E_{2})\\
\mathit{FV}(\lam v \tc{.}E) = \mathit{FV}(E) - \{\tm{v}\}
\end{mcode}
%\end{array}\]
%\color{black}

%\color{MathColor}
%\[\begin{array}{l}
%\mathit{FV}(\tc{v}) = \{\tc{v}\}\\
%\mathit{FV}(E_{1}E_{2}) = \mathit{FV}(E_{1}) \cup \mathit{FV}(E_{2})\\
%\mathit{FV}(\lam \tc{v.}E) = \mathit{FV}(E) - \{\tc{v}\}
%\end{array}\]
%\color{black}

\newslide


\subsection*{Substitution}

\begin{tabular}{l|l|l|l}
\tm{E} & \tm{E[E' / \tm{v}]} & Restriction & Comment \\
\hline
\tm{v} & \tm{E'} & & \\
\tm{w} & \tm{w} & \tm{w} \tm{\not=} \tm{v} \\
\tm{E_1 E_2} & (\tm{E_1 [E' / \tm{v}]}) (\tm{E_2 [E' / \tm{v}]}) & \\
\lam \tm{v} \tc{.}\tm{E_{1}} & \lam \tm{v} \tc{.}\tm{E_{1}} & \\
\lam \tm{w} \tc{.}\tm{E_{1}} & \lam \tm{w} \tc{.}(\tm{E_{1} [E' / \tm{v}]}) &
   \tm{w} \tm{\not=} \tm{v} \tm{\wedge} \\
& & (\tm{v} \tm{ \not\in \mathit{FV}(E_1)} \tm{\vee} & No Substitution\\
& &  \tm{w} \tm{ \not\in \mathit{FV}(E')}) & No Name Conflict\\
\lam \tm{w} \tc{.}\tm{E_{1}} & \lam \tm{x} \tc{.}((\tm{E_{1}[\tm{x}/\tm{w}]})
\tm{[E'  / \tm{v}]}) &
   \tm{w} \tm{\not=} \tm{v} \tm{\wedge}\\
& & \tm{v} \tm{ \in \mathit{FV}(E_1)} \tm{\wedge} & Substitution\\
& &  \tm{w} \tm{ \in \mathit{FV}(E')} \tm{\wedge} & Name Conflict\\
& & \tm{x} \tm{ \not\in \mathit{FV}(E_{1}E')} & \tm{x} is new
\end{tabular}

%\begin{tabular}{l|l|l|l}
%\tm{E} & \tm{E[E' / \tc{v}]} & Restriction & Comment \\
%\hline
%\tc{v} & \tm{E'} & & \\
%\tc{w} & \tc{w} & \tc{w} \tm{\not=} \tc{v} \\
%\tm{E_1 E_2} & (\tm{E_1 [E' / \tc{v}]}) (\tm{E_2 [E' / \tc{v}]}) & \\
%\lam \tc{v.}\tm{E_{1}} & \lam \tc{v.}\tm{E_{1}} & \\
%\lam \tc{w.}\tm{E_{1}} & \lam \tc{w.}(\tm{E_{1} [E' / \tc{v}]}) &
%   \tc{w} \tm{\not=} \tc{v} \tm{\wedge} \\
%& & (\tc{v} \tm{ \not\in \mathit{FV}(E_1)} \tm{\vee} & No Substitution\\
%& &  \tc{w} \tm{ \not\in \mathit{FV}(E')}) & No Name Conflict\\
%\lam \tc{w.}\tm{E_{1}} & \lam \tc{x.}(\tm{E_{1}[\tc{x}/\tc{w}]})
%\tm{[E'  / \tc{v}]} &
%   \tc{w} \tm{\not=} \tc{v} \tm{\wedge}\\
%& & \tc{v} \tm{ \in \mathit{FV}(E_1)} \tm{\wedge} & Substitution\\
%& &  \tc{w} \tm{ \in \mathit{FV}(E')} \tm{\wedge} & Name Conflict\\
%& & \tc{x} \tm{ \not\in \mathit{FV}(E_{1}E')} & \tc{x} is new
%\end{tabular}

\newpage

\subsection*{Reduction Rules}

\subsubsection*{$\alpha$-conversion}

\[\lam \tm{v} \tc{.} \tm{E \underset{\alpha}{\longleftrightarrow} 
\lam \tm{x} \tc{.} E[\tm{x}/\tm{v}]\textcolor{black}{\mathrm{, \hspace{0.25in} if }}~\tm{x}\not\in
\mathit{FV}(E)}\]

\subsubsection*{$\beta$-reduction}
\[\tm{(}\lam \tm{v} \tc{.} \tm{E_{1})E_{2} \underset{\beta}{\longrightarrow} E_{1}[E_{2}/\tm{v}]}\]

\subsubsection*{$\eta$-reduction}
\[\lam \tm{v} \tc{.} \tm{(E\tm{v}) \underset{\eta}{\longrightarrow} E
\textcolor{black}{\mathrm{, \hspace{0.25in} if }}~\tm{v}\not\in \mathit{FV}(E)}\]

\newslide

\subsubsection*{Reduction of subexpressions}

\[\tm{\frac{E_{1} \underset{\beta}{\longrightarrow} E_{2}}
{EE_{1} \underset{\beta}{\longrightarrow} EE_{2}}}\]

\[\tm{\frac{E_{1} \underset{\beta}{\longrightarrow} E_{2}}
{E_{1}E \underset{\beta}{\longrightarrow} E_{2}E}}\]

\[\tm{\frac{E_{1} \underset{\beta}{\longrightarrow} E_{2}}
{\tc{\lam \tm{v}.}E_{1} \underset{\beta}{\longrightarrow} {\tc{\lam \tm{v}.}E_{2}}}}\]

Similar rules apply to $\alpha$-conversion and $\eta$-reduction.

\newslide

An expression, \tm{E}, is called a \definition{redex} if it is of the form
\tm{(\lam v\tc{.}E_{1})E_{2}} or \tm{\lam v\tc{.}(E_{3}v)}, with \tm{v \not\in \mathit{FV}(E_{3})}.

Hence, it is possible to apply at least one of the reduction rules to an
expression iff the expression contains a redex.

An expression, \tm{E}, is said to be in \definition{normal form} if it does not contain a redux.

\newpage


We note that if \tm{\lam v.E_{1} \underset{\alpha}{\longleftrightarrow}
\lam x.E_{2}} then, by definition of $\alpha$-conversion,
\tm{x} cannot be free in \tm{E_{1}} and \tm{v} cannot be free in
\tm{E_{2}}.  It follows that
\tm{(E_{1}[x/v])[v/x]} is exactly the same as \tm{E_{1}}and hence
\tm{\lam v.E_{1} \underset{\alpha}{\longleftrightarrow}
\lam x.E_{2}}
iff
\tm{\lam x.E_{2}
\underset{\alpha}{\longleftrightarrow}
\lam v.E_{1}}.
That is, \tm{\underset{\alpha}{\longleftrightarrow}} is symmetric,
so the reflexive, transitive closure of
\tm{\underset{\alpha}{\longleftrightarrow}}, which we will call
\tm{\Leftrightarrow}, is an equivalence relation.

\vspace{2ex}

Clearly, \tm{E_{1} \Leftrightarrow E_{2}} means that the two expressions
are identical except for the renaming of bound variables.

\newslide

We will define
\color{MathColor}
\[\longrightarrow ~ =  ~ \underset{\alpha}{\longleftrightarrow} ~ \cup
 ~ \underset{\beta}{\longrightarrow} ~ \cup ~ \underset{\eta}{\longrightarrow}\]
\color{black}
and let \tm{\overset{*}{\longrightarrow}} be the reflexive, transitive
closure of \tm{\longrightarrow}, and \tm{\overset{*}{\longleftrightarrow}}
be the symmetric, reflexive, transitive closure of \tm{\longrightarrow}.

\vspace{2ex}

The relation \tm{\overset{*}{\longleftrightarrow}} is an equivalence relation,
but with larger equivalence classes than
\tm{\Leftrightarrow}.
That is, whenever \tm{E_{1} \Leftrightarrow E_{2}} then
\tm{E_{1} \overset{*}{\longleftrightarrow} E_{2}}, but not conversely.

\newslide

If \tm{n \geq 0} and \tm{E_{0} \overset{*}{\longrightarrow} E_{n}},
then a sequence of reduction steps
\tm{E_{0} \redu E_{1} \redu E_{2} \redu \cdots \redu E_{n}}
is called a \definition{reduction}.  \tm{E_{n}} may or may not be
in normal form.

% Note 2:  It doesn't particularly matter whether we
% allow $\alpha$-conversions to count as reduction
% steps.  

\vspace{2ex}

In general, an expression may contain a number of redexes, so there may be
a number of different ways to go about reducing an expression.

\vspace{2ex}

If at each step during a reduction the leftmost outermost redex is chosen for reduction,
then the reduction is called a \definition{normal order reduction}.

\vspace{2ex}

If at each step during a reduction the leftmost innermost redex is chosen for reduction,
then the reduction is called a \definition{applicative order reduction}.

\newslide

\begin{theorem}[Church-Rosser Theorem I]
If \tm{E \overset{*}{\longrightarrow} E_{1}} and \tm{E \overset{*}{\longrightarrow} E_{2}}
then there exists an \tm{E'} such that \tm{E_{1} \overset{*}{\longrightarrow} E'} and
\tm{E_{2} \overset{*}{\longrightarrow} E'}.
\end{theorem}

\paragraph{Proof}  This is an Area II course, so you'll just have to take my word for this.

\vspace{2ex}

\begin{corollary}
If \tm{E \overset{*}{\longrightarrow} E_{1}} and \tm{E \overset{*}{\longrightarrow} E_{2}}
and \tm{E_{1}} and \tm{E_{2}} are both in normal form, then \tm{E_{1} \Leftrightarrow E_{2}}.
\end{corollary}

\newslide

\begin{theorem}[Church-Rosser Theorem II]
If \tm{E \overset{*}{\longrightarrow} E_{1}} and \tm{E_{1}} is in normal form, then
there exists a normal order reduction of \tm{E} to \tm{E_{1}},
\end{theorem}

A \definition{lazy functional programming language} uses normal order reduction, but with an
optimization that avoids reevaluation of
subexpressions that are duplicated during the reduction.

Normal order reduction with this optimization is called \definition{lazy evaluation}.

Normal order reduction without this optimization
yields call-by-name parameter passing,
and applicative order reduction is somewhat similar
to call-by-value parameter passing.

\end{slide}

\end{document}
