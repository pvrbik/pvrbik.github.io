\documentclass{seminar}
\nonstopmode

% \usepackage{geometry}
% \geometry{letterpaper,landscape}
% \usepackage{pictex}
%  pictex causes me to run out of dimensions (no room for new #3)
\usepackage{pictex}
\usepackage{dcpic}
\usepackage{semhelv}
\usepackage{color}
\usepackage{alltt}
\usepackage{latexsym}
\usepackage{amsfonts}
\usepackage{amsmath}
\usepackage{calc}
\usepackage{graphicx}
\usepackage{pifont}

\definecolor{CodeColor}{named}{BlueViolet}
% \definecolor{SystemColor}{named}{Mulberry}
% \definecolor{SystemColor}{named}{OliveGreen}
\definecolor{SystemColor}{named}{RedOrange}
\definecolor{MathColor}{named}{Magenta}
\definecolor{QuoteColor}{named}{Salmon}
\definecolor{LawColor}{named}{Green}
\definecolor{DefinitionColor}{named}{Red}
\definecolor{AlgorithmColor}{named}{RawSienna}
\definecolor{BackgroundColor}{named}{Yellow}
\definecolor{OtherColor}{named}{PineGreen} % {Maroon}
\definecolor{Green}{named}{OliveGreen}
% \definecolor{BackgroundColor}{cmyk}{0,0,0.6,0}

\newcommand{\semester}{Fall 2006}

\newcommand{\bs}{\ensuremath{\backslash}}
\newcommand{\zero}{\textcolor{MathColor}{\ensuremath{\mathbf{0}}~}}
\newcommand{\one}{\textcolor{MathColor}{\ensuremath{\mathbf{1}}~}}

\newcommand{\law}[1]{\textcolor{black}{(\textcolor{LawColor}{#1})}}

\newcommand{\sublaw}[1]{\textcolor{black}{(\textcolor{black}{#1})}}

\newcommand{\reason}[1]{\hspace{0.15in}
    \meq\hspace{0.075in}\law{#1}}

\newcommand{\subreason}[1]{\hspace{0.15in}
    \meq\hspace{0.075in}\sublaw{#1}}

\newcommand{\definition}[1]{\hspace{0.05in}
    \textcolor{DefinitionColor}{\textit{#1}\hspace{0.05in}}}

\newcommand{\meq}{\textcolor{MathColor}{\ensuremath{\equiv}}}
\newcommand{\bottom}{\textcolor{MathColor}{\ensuremath{\bot}}}
\newcommand{\dom}{\textcolor{MathColor}{\ensuremath{\mathit{dom~}}}}
\newcommand{\cod}{\textcolor{MathColor}{\ensuremath{\mathit{cod~}}}}
\newcommand{\rarrow}{\textcolor{MathColor}{\ensuremath{\rightarrow}}}

\newcommand{\textcd}[2][CodeColor]{\textcolor{#1}{\texttt{#2}}}
\newcommand{\tc}[2][CodeColor]{\textcolor{#1}{\texttt{#2}}}
\newcommand{\tcx}[2][OtherColor]{\textcolor{#1}{\texttt{#2}}}
\newcommand{\ts}[2][SystemColor]{\textcolor{#1}{\texttt{#2}}}
\newcommand{\textcdl}[2][CodeColor]{\textcolor{#1}{\hspace{0.1in}\texttt{#2}}}
\newcommand{\textcdr}[2][CodeColor]{\textcolor{#1}{\texttt{#2}\hspace{0.1in}}}
\newcommand{\textcdlr}[2][CodeColor]{\textcolor{#1}
{\hspace{0.1in}\texttt{#2}\hspace{0.1in}}}

\newcommand{\tmm}[1]{\textcolor{AlgorithmColor}{\ensuremath{#1}}}


\newcommand{\textmath}[1]{\textcolor{MathColor}{\ensuremath{#1}}}
\newcommand{\tm}[1]{\textcolor{MathColor}{\ensuremath{#1}}}
\newcommand{\tmcc}[1]{\textcolor{CodeColor}{\ensuremath{#1}}}
\newcommand{\tmtt}[1]{\textcolor{MathColor}{\texttt{#1}}}
\newcommand{\unmath}[1]{\textcolor{black}{\mathrm{#1}}}
\newcommand{\tmi}[1]{\textcolor{MathColor}{\ensuremath{\mathit{#1}}}}
\newcommand{\tmr}[1]{\textcolor{MathColor}{\ensuremath{\mathrm{#1}}}}
\newcommand{\tmb}[1]{\textcolor{MathColor}{\ensuremath{\mathbf{#1}}}}


\newcommand{\tbr}[1]{\textcolor{red}{\textbf{#1}}}

\newcommand{\bla}[1]{\textcolor{black}{#1}}
\newcommand{\gre}[1]{\textcolor{Green}{#1}}
\newcommand{\red}[1]{\textcolor{red}{#1}}
\newcommand{\blu}[1]{\textcolor{blue}{#1}}
\newtheorem{lemma}{Lemma}
\newtheorem{theorem}{Theorem}
\newtheorem{corollary}{Corollary}
\newenvironment{proof}{\textbf{Proof:}}{\hfill $\Box$}

\slideframe{none}

\centerslidesfalse

\newlength{\tempwidth}

\newenvironment{cdiag}
   {\vspace{0.25in}
    \begin{center}
    \setlength{\tempwidth}{\linewidth}
    \addtolength{\tempwidth}{-2\columnsep}
    \begin{minipage}[t]{\tempwidth}
    \begindc{\commdiag}[10]
   }
   {\enddc
    \end{minipage}
    \end{center}
    \vspace{0.25in}
   }

\newenvironment{cdiag2}
   {\vspace{0.25in}
    % \begin{center}
    \setlength{\tempwidth}{\linewidth/2}
    \begin{minipage}[t]{\tempwidth}
    \begindc{\commdiag}[10]
   }
   {\enddc
    \end{minipage}
    % \end{center}
    \vspace{0.25in}
   }

\newenvironment{udg}
   {\vspace{0.25in}
    \begin{center}
    \setlength{\tempwidth}{\linewidth}
    \addtolength{\tempwidth}{-2\columnsep}
    \begin{minipage}[t]{\tempwidth}
    \begindc{2}[10]
   }
   {\enddc
    \end{minipage}
    \end{center}
    \vspace{0.25in}
   }

\newenvironment{codenott}[1][CodeColor]
   {\color{#1}
    % \begin{center}
    \setlength{\tempwidth}{\linewidth}
    \addtolength{\tempwidth}{-2\columnsep}
    \hspace*{0.25in}
    \begin{minipage}[t]{\tempwidth}
    \begin{tabbing}
    \hspace{0.075in} \= \hspace{0.075in} \= \hspace{0.075in} \= \hspace{0.075in} \= \hspace{0.075in} \= \kill
   }
   {\end{tabbing}
    \end{minipage}
    % \end{center}
    \color{black}
   }

\newenvironment{codenott2}[1][CodeColor]
   {\color{#1}
    % \begin{center}
    \setlength{\tempwidth}{\linewidth/3}
    \hspace*{0.25in}
    \begin{minipage}[b]{\tempwidth}
    \begin{tabbing}
    \hspace{0.075in} \= \hspace{0.075in} \= \hspace{0.075in} \= \hspace{0.075in} \= \hspace{0.075in} \= \kill
   }
   {\end{tabbing}
    \end{minipage}
    % \end{center}
    \color{black}
   }

\newenvironment{laws}
   {\setlength{\tempwidth}{\linewidth}
    \addtolength{\tempwidth}{-2\columnsep}
    \hspace*{0.25in}
    \begin{minipage}[t]{\tempwidth}
    \begin{tabular}{lr}
   }
   {\end{tabular}
    \end{minipage}
   }

\newenvironment{code}[1][CodeColor]
   {\color{#1}
    % \begin{center}
    \setlength{\tempwidth}{\linewidth}
    \addtolength{\tempwidth}{-2\columnsep}
    \hspace*{0.25in}
    \begin{minipage}[t]{\tempwidth}
    \begin{alltt}
    \begin{tabbing}
    \hspace{0.075in} \= \hspace{0.075in} \= \hspace{0.075in} \= \hspace{0.075in} \= \hspace{0.075in} \= \kill
   }
   {\end{tabbing}
    \end{alltt}
    \end{minipage}
    % \end{center}
    \color{black}
   }

\newenvironment{mcode}[1][MathColor]
   {\color{#1}
    % \begin{center}
    \setlength{\tempwidth}{\linewidth}
    \addtolength{\tempwidth}{-2\columnsep}
    \hspace*{0.25in}
    \begin{minipage}[t]{\tempwidth}
    \[\begin{array}{lrll}}
   {\end{array}\]
    \end{minipage}
    % \end{center}
    \color{black}
   }

\newcommand{\mF}{\textcolor{MathColor}{\ensuremath{\tc{fmap}_{F}~}}}
\newcommand{\Id}{\textcolor{MathColor}{\ensuremath{\mathit{Id}}}}
\newcommand{\nt}{\textcolor{MathColor}{\ensuremath{\stackrel{.}{\rightarrow}}}}
\newcommand{\List}{\textcolor{MathColor}{\ensuremath{\tc{[}?\tc{]}}}}
\newcommand{\LL}{\textcolor{MathColor}{\ensuremath{\tc{[[}?\tc{]]}}}}
\newcommand{\LA}{\textcolor{MathColor}{\ensuremath{\tc{[}A\tc{]}}}}
\newcommand{\LB}{\textcolor{MathColor}{\ensuremath{\tc{[}B\tc{]}}}}
\newcommand{\LLA}{\textcolor{MathColor}{\ensuremath{\tc{[[}A\tc{]]}}}}
\newcommand{\LLB}{\textcolor{MathColor}{\ensuremath{\tc{[[}B\tc{]]}}}}
\newcommand{\lam}{\textcolor{CodeColor}{\ensuremath{\lambda}}}
\newcommand{\redu}{\textcolor{MathColor}{\ensuremath{\longrightarrow}}}
\newcommand{\Has}{\textit{Haskell}\,}

\newcommand{\ack}{I would like to thank Tai Meng for many corrections
and suggestions for improvements to these notes.  Of course, the remaining
errors are my fault.}


\newcommand{\NAT}{\textcolor{MathColor}{\ensuremath{\mathbb{N}}}}
\newcommand{\suc}{\tmi{succ}}
\newcommand{\Ha}{\tmi{Hask}}
\newcommand{\vs}{\vspace{0.3in}}
\newcommand{\CC}{\tm{\cal C}}
\newcommand{\CCC}{\tm{{\cal C}\,}}
\newcommand{\COP}{\tm{\cal C^\mathrm{op}}}
\newcommand{\DD}{\tm{\cal D}}
\newcommand{\DDD}{\tm{{\cal D}\,}}

\newcommand{\pb}{\textcolor{MathColor}{\ensuremath{+\!\!_{_{_{\bot}}}}}}

\newpagestyle{lectures}{\makebox[1in][l]{CMPT 481/731} \hfill Functional Programming
\hfill \makebox[1in][r]{\semester}}{\makebox[2in][l]{Category Theory} \hfill F. Warren
Burton \hfill \makebox[2in][r]{\thepage}}

% \newpagestyle{lectures}{CMPT 481/731 \hfill Functional Programming
% \hfill \semester}{Category Theory \hfill
% F. Warren Burton \hfill \thepage}

\pagestyle{lectures}

\begin{document}
\begin{slide}

\begin{center}
\begin{LARGE}
\textbf{Category Theory}
\end{LARGE}\\[3ex]
\end{center}

Category theory is a generalization of set theory.

By having only a limited
number of carefully chosen
requirements in the definition of a category, we are able to
produce a general algebraic structure with a large number of useful
properties, yet permit the theory to apply to many different specific
mathematical
systems.

Using concepts from category theory in the design of a programming
language leads to some powerful and general programming constructs.
While it is
possible to use the resulting constructs without
knowing anything about category theory,
understanding the underlying theory helps.

\newslide

\section*{Definition of a Category}

A \definition{category} \tm{\cal{C}} is:
\begin{enumerate}
\item a collection of \tm{objects};
\item a collection of \tm{arrows};
\item operations assigning to each arrow \tm{f}
\begin{enumerate}
\item an object \tm{\dom f} called the \definition{domain} of \tm{f},
and
\item an object \tm{\cod f} called the \definition{codomain} of
\tm{f}
\end{enumerate}
often expressed by \tm{f: \dom f \rightarrow \cod f};
\item an associative composition operator \tm{\circ} assigning to
each pair
of arrows, \tm{f} and \tm{g}, such that \tm{\dom f = \cod g}, a
\definition{composite} arrow \tm{f \circ g: \dom g \rightarrow \cod
f}; and
\item for each object \tm{A}, an identity arrow, \tm{id_A: A
\rightarrow A}
satisfying the law that for any arrow \tm{f:A \rightarrow B},
\begin{codenott}
\tm{id_B \circ f = f = f \circ id_A}.
\end{codenott}
\end{enumerate}

\newslide

In diagrams, we will usually express \tm{A = \dom f} and \tm{B =
\cod f}
(that
\linebreak
is \tm{f:A\rightarrow B}) by

\begin{cdiag}
\obj(1,1)[a]{\textmath{A}}
\obj(11,1)[b]{\textmath{B}}
\mor{a}{b}{\textmath{f}}
\end{cdiag}

The associative requirement for the composition operators means
that
when \tm{f \circ g} and \tm{g \circ h} are both defined, then
\tm{f \circ (g \circ h) = (f \circ g) \circ h}.

This allow us to think of arrows defined by paths throught diagrams.

\begin{cdiag}
\obj(1,1)[A]{\textmath{\dom h}}
\obj(1,0)[X]{\textmath{=A}}
\obj(11,1)[B]{\textmath{\cod h = \dom g}}
\obj(11,0)[Y]{\textmath{=B}}
\obj(21,1)[C]{\textmath{\cod g = \dom f}}
\obj(21,0)[C]{\textmath{=C}}
\obj(31,1)[D]{\textmath{\cod f}}
\obj(31,0)[D]{\textmath{=D}}
\mor(2,2)(30,2){\textmath{f \circ g \circ h}}
\mor(2,0)(9,0){\tm{h}}
\mor(13,0)(19,0){\tm{g}}
\mor(23,0)(30,0){\tm{f}}
\end{cdiag}

\newslide

I will sometimes write \tm{(;)} for \tc{flip}\tm{(\circ)}, since

\begin{cdiag}
\obj(1,1)[X]{\textmath{A}}
\obj(11,0)[Y]{\textmath{B}}
\obj(21,0)[C]{\textmath{C}}
\obj(31,1)[D]{\textmath{D}}
\mor(2,2)(30,2){\textmath{h; g; f}}
\mor(2,0)(10,0){\tm{h}}
\mor(12,0)(20,0){\tm{g}}
\mor(22,0)(30,0){\tm{f}}
\end{cdiag}

is less confusing than

\begin{cdiag}
\obj(1,1)[X]{\textmath{A}}
\obj(11,0)[Y]{\textmath{B}}
\obj(21,0)[C]{\textmath{C}}
\obj(31,1)[D]{\textmath{D}}
\mor(2,2)(30,2){\textmath{f \circ g \circ h}}
\mor(2,0)(10,0){\tm{h}}
\mor(12,0)(20,0){\tm{g}}
\mor(22,0)(30,0){\tm{f}}
% \mor(2,0)(9,0){\tm{h}}
% \mor(13,0)(19,0){\tm{g}}
% \mor(23,0)(30,0){\tm{f}}
\end{cdiag}


\newslide

\subsection*{Examples}

The category that we will use the most has all possible \Has
types
(ground types, not polymorphic types) as objects and all possible
\Has
(non-polymorphic) functions as arrows.

(A polymorphic datatype corresponds to an infinite collection of
non-polymorphic
datatypes, and similarly, a polymorphic function corresponds to the
infinite
collection of the possible non-polymorphic instances of the functions.)

We will call this category \definition{Hask}.  Let us complete the
obvious details of the definition of \Ha\, and confirm that \Ha\,
satisfies the definition of a category.

\newslide

\paragraph{Set (\definition{Set}):}  Category theory grew out
of a generalization of
set theory.  Sets and functions between them may be viewed as a
category.
Objects are sets and arrows are total functions.  Usually subsets
are regarded
as distinct sets.  For example, the set of integers is not related
to the
set of reals.  Each function is associated with a particular domain
and
codomain.

\vs

\paragraph{Discrete Categories:} A discrete category is one in which
the only
arrows are the identity arrows.

A single set can be viewed as a discrete category, where each element
of the
set is an object, although this is not usually particularly useful.

The category, \one, is the one object discrete category, which is
sometimes
useful.  (Note:  Later we will use \tm{1} to denote a terminal (or
final)
object in a category. )

\newslide

\paragraph{Power Sets:}  A power set \tm{{\cal P}(S)}
of a set \tm{S} is the set of all subsets of the set \tm{S}.  For
any set, we have a category where the objects of the category are
the
elements of the power set.  There is at most one arrow between any
two
objects.  There is an arrow from \tm{A} to \tm{B} iff \tm{A \subseteq
B}.

\paragraph{Any Partial Order:}  The power
set of a
set is a partial order.  In a similar manner, any partial order can be
view as
a category.

\paragraph{All Partial Orders(\definition{Poset}):}
There is also a category where the objects
are partial orders and the arrows are monotonic functions between
these partial orders.

\newslide

\paragraph{Predicate Calculus:}  We can view the predicate
calculus as a category, where predicates
are objects and there is an arrow from \tm{A} to \tm{B} if and only
if
\tm{A} implies \tm{B}.  Note that \tm{A} implies itself, so we have
identity
arrows, and logical implication is transitive, so composition is
well
defined.  There is never more than one arrow between any two
predicates.

\newslide

\paragraph{Any Group:}  Any group can be viewed as a category with
a single object.  The arrows correspond to the elements of the
group.
The arrow composition operator corresponds to the binary group
operation.

Certain other algebraic structures can be viewed as categories in
a
similar manner.

\paragraph{All Groups (\definition{Grp}):}  There is also
a category where each group
is an object and arrows are homomorphisms between groups.

\newslide

A homomorphism is a structure preserving mapping from one set to
another.
For most algebraic structures, you can build categories where
the arrows are homomorphisms.

For example, a homomorphism \tm{h} from the group of \tmi{integers}
under \tmi{addition}, with identity element \tm{0}, to the
\tmi{positive~rationals} under \tmi{multiplication}, with identity
element \tm{1}, must satisfy the following equations.

\begin{codenott}
\tm{h(0) = 1}\\
\tm{h(a + b) = h(a) * h(b)}
\end{codenott}

The exponential function (with any base) satisfies this condition,
as does the constant function \tm{h(x) = 1}.

\newslide

Graphs and graph homomorphism form another category of this nature.

\vs

Functors (see the following page) are homomorphisms between categories.

\newslide

\section*{Functors}

A \definition{functor} \tm{F} from category \tm{\cal C} to \tm{\cal
D}
is a pair of functions,
\begin{enumerate}
\item \tm{F_0: {\cal C}}-\tm{\unmath{objects} \rightarrow {\cal
D}}-\tm{\unmath{objects,}}
\item \tm{F_1: {\cal C}}-\tm{\unmath{arrows} \rightarrow {\cal
D}}-\tm{\unmath{arrows,}}
\end{enumerate}
such that
\begin{enumerate}
\item If \tm{f} is an arrow in \tm{\cal C} and \tm{f: A \rightarrow
B}
then \tm{F_1(f): F_0(A) \rightarrow F_0(B)},
\item \tm{F_1(id_A) = id_{F_0(A)}}, and
\item \tm{F_1(f \circ g) = F_1(f) \circ F_1(g)} whenever \tm{f \circ
g}
is defined.
\end{enumerate}

\newslide

We write \tm{F: {\cal C} \rightarrow {\cal D}} to express that
\tm{F} is a functor from \tm{\cal C} to \tm{\cal D}.

\vs

Usually we will write \tm{F} for \tm{F} itself,
and also for both \tm{F_{0}} and \tm{F_{1}}.
\newslide

Functors, as defined above, are sometimes called \definition{covariant
functors}.

\vs

Once in a while, we may want a functor that reverses the direction
of the arrows.  That is, we would like to change\\
\begin{codenott}[black]
If \tm{f} is an arrow in \tm{\cal C} and \tm{f: A \rightarrow B}
then \tm{F_1(f): F_0(A) \rightarrow F_0(B)},
\end{codenott}\\
to\\
\begin{codenott}[black]
If \tm{f} is an arrow in \tm{\cal C} and \tm{f: A \rightarrow B}
then \tm{F_1(f): F_0(B) \rightarrow F_0(A)},
\end{codenott}

Of course, this also means that\\
\begin{codenott}[black]
\tm{F_1(f \circ g) = F_1(f) \circ F_1(g)} whenever \tm{f \circ g}
is defined
\end{codenott}\\
must be changed to\\
\begin{codenott}[black]
\tm{F_1(f \circ g) = F_1(g) \circ F_1(f)} whenever \tm{f \circ g}
is defined.
\end{codenott}

\vs

Functors that reverse the direction of the arrows are called
\definition{contravariant functors}.

\newslide

\subsection*{Examples}

\paragraph{\Ha:}  In \Ha, objects are \Has types, so the object
part of a functor, \tm{F_{0}}, is a function from types to types.

Therefore, a type constructor is suitable for use as the object
function
of a functor.  For example

\begin{codenott}[MathColor]
$F_{0}\,A = [A]$
\end{codenott}

is suitable.  In this case

\begin{codenott}[MathColor]
$F_{1}$ = map
\end{codenott}

completes the definition of a functor.

\newslide

We can think of this functor as lifting \Ha\, up to a category of
list types,
which is a subcategory of \Ha.

\begin{cdiag}
\obj(1,1)[A]{\textcd{a}}
\obj(10,1)[B]{\textcd{b}}
\obj(1,6)[FA]{\textcd{[a]}}
\obj(10,6)[FB]{\textcd{[b]}}
\mor{A}{B}{\textcd{f}}
\mor{FA}{FB}{\textcd{map f}}
\end{cdiag}

\newslide

This idea of lifting a type with a type constructor is common in
\Has.

Unfortunately, with this sort of functor in \Has,
we can't use the same name for both the function on objects and the
function on arrows, as is usually done in category theory.

Instead we have a functor class

\begin{code}
class Functor f where\\
\> fmap :: (a -> b) -> f a -> f b
\end{code}

This allows us to associate an \tc{fmap} function with a type
constructor
and use \tc{fmap} as a generic name for the function on arrows for
a
type constructor viewed as the object function part of a functor.

\vs

Notice that while we have previously had classes for types (e.g.
\tc{Eq},
\tc{Show}), \tc{Functor} is a \definition{constructor class} whose
members
are type constructors rather than types.

\newslide

The \textit{Haskell Standard Prelude} declares the list type to be a functor
by

\begin{code}
instance Functor [] where\\
\> fmap = map
\end{code}

so that \tc{fmap} can be used instead of map in contexts where is
it
clear that we are working with lists.

\newslide

Two other functor instances are defined in the \textit{Haskell Standard
Prelude}.

\begin{code}
data Maybe a =\\
\> Nothing | Just a\\
\> \> deriving (Eq, Ord, Read, Show)\\
\\
instance Functor Maybe where\\
\> fmap f Nothing = Nothing\\
\> fmap f (Just x) = Just (f x)\\
\\
instance Functor IO where\\
\> fmap f x = x >>= (return . f)
\end{code}

We won't worry about the \tc{IO} type until later in the course.

\newslide

\paragraph{\textcolor{black}{Quiz:}}

\color{QuoteColor}
\begin{enumerate}
\item Find a functor \tm{G: \Ha \rightarrow \one}.
\item Prove that \tc{Maybe} together with its \tc{fmap} instance
obeys
the functor laws.
\item Show that for any category there exists an identity functor
for that category.
\item Show that the composition of two compatible functors is a
functor.
\end{enumerate}
\color{black}

\newslide

\subsection*{The Category of ({\small small}) Categories:}

A \definition{small category} is a category where the collection
of objects
is a set and the collection of arrows is a set.

\vs

\paragraph{The category of all small categories (\definition{Cat}):}
The category \tmi{Cat}\, has all small categories as objects and
all functors
between these as arrows.

\vs

Of course, \tmi{Cat}\, is not a small category and therefore not
an object
within itself, thereby keeping Bertrand Russell happy.

\newslide

\section*{Duality}

Given any category \CC, it is possible to produce another category,
\COP, that is identical to \CC \, except that all arrows are reversed.

A quick inspection of the definition of a category should be
sufficient
to see that this is so.

A category \CC \, might seem
more ``reasonable'' or ``natural'' then \COP.  For example
consider \tmi{Set}.  \tmi{Set^\mathrm{op}} is as valid
as \tmi{Set}\, as far as the mathematics of category theory go,
even though
the arrows don't have a particularly nice interpretation.

We say that \COP \, is the \definition{dual} or \definition{opposite}
of \CC.

\newslide

The \definition{Duality Principle} say that any statement that is
true of
an arbitrary category \CC \, is also true of \COP.

\vs

Similarly, concepts and definitions in category theory have duals.

\vs

As we will see below, an \textit{initial object} is the dual of a
\textit{final object}.  We will also see later that \textit{sums}
and
\textit{products} of objects are dual concepts.


\newslide

\section*{Initial and Final Objects}

An \definition{initial object} in a category \CC \, is an object
\tm{0} such that for any object \tm{A} in \CC \, there is exactly
one arrow in \CC \, from \tm{0} to \tm{A}.

\vs

Since \tm{0} is an object in \CC, the only arrow from \tm{0} to
itself
is the identity arrow.

\vs

Similarly, a \definition{final object} (or \definition{terminal
object})
in a category \CC \, is an object
\tm{1} such that for any object \tm{A} in \CC \, there is exactly
one arrow in \CC \, from \tm{A} to \tm{1}.

\newslide

In any category \CC \, we say that two objects, \tm{A} and \tm{B},
are
isomorphic iff there
exist arrows

\begin{codenott}
\tm{f: A \rightarrow B}\\
\tm{g: B \rightarrow A}
\end{codenott}

such that

\begin{codenott}
\tm{g \circ f = id_{A}}\\
\tm{f \circ g = id_{B}}
\end{codenott}

\vs

In the case of \tmi{Set}, this is just the ordinary definition of
isomorphism.

\vs

A category may have no initial objects, exactly one initial object,
or many
initial objects.  Any two initial objects in a category are isomorphic.
The same applies to final objects.

\newslide

\paragraph{\textcolor{black}{Quiz:}}

\color{QuoteColor}
\begin{enumerate}
\item Find a category with no initial objects.
\item Find a category with exactly one initial object.
\item Find a category with more than one initial object.
\item Prove that any two initial objects are isomorphic.
\item Find a category with no final objects.
\item Find a category with exactly one final object.
\item Find a category with more than one final object.
\item Prove that any two final objects are isomorphic.
\end{enumerate}
\color{black}


\newslide

\subsection*{Examples}

In the predicate calculus, we have exactly on initial object,
\tmi{False},
and one final object, \tmi{True}.

Similarly, with the power set \tm{{\cal P}(S)}, the initial object
is
\tm{\emptyset} and the final object is \tm{S}.

In general, a partial order may or may not have an initial object.
A initial
object, if it exists, is unique.  Similarly for final objects.

\newslide

\subsubsection*{Initial and Final Objects in \tmi{Set}}

The initial and final objects of \tmi{Set}\, are a bit more
interesting.
\tmi{Set}\, has a single initial object but an infinite number of
final objects.

\vs

There is exactly one initial object, \tm{\emptyset}.  For any set
\tm{S},
there is exactly one function, \tm{f: \emptyset \rightarrow S}.
This
function is the empty function.  That is, if we view a function as
a set of
ordered pairs, \tm{\{(x,y)|x \in \dom f, y = f(x)\}}, then \tm{f}
is the
empty set of ordered pairs.

\newslide

A set \tm{S} is a final object in \tmi{Set}\, if and only if the
set contains exactly
one element.  If \tm{S=\{s\}}, then the only
function from any other set is a constant function, \tm{f(x)=s}.

\vs

Note that there exists exactly one arrow from the initial object
to any
final object.

\newslide

We also find in \tmi{Set}\, that the arrows going in the
``wrong direction'', to the initial
object and from the final objects, are of interest.

\vs

In all of \tmi{Set}\, there is only one arrow to the initial object,
and that
is the identity arrow for \tm{\emptyset}.


\vs

On the other hand, there are lots of arrows from the final objects.
We
will let \tm{1} denote any arbitrary final object (singleton set),
and \tm{S} be
any set.  There is exactly one arrow from \tm{1} to \tm{S} for
each element of
\tm{S}.  There arrows from a final object are constants functions.

That is, since there is exactly one object in \tm{1}, each function
to
\tm{S} takes the only element in \tm{1} to a single element of
\tm{S}.

\newslide

Since the arrows from \tm{1} to \tm{S} in \tmi{Set}\, correspond
to the
elements of \tm{S}, category theory can deal with elements of a set
without
any need for a mechanism outside category theory.

Furthermore, this generalizes the concept of
\textit{element} to categories other
than \tmi{Set}.

\vs

Notice that while initial objects and final objects are dual concepts,
they are not dual to each other within the specific category
\tmi{Set}.

\newslide

%\subsubsection*{Natural Numbers as an Initial Object}

%Let's now consider a definition of \textit{Natural Numbers},
%\NAT, based on Peano's axioms. 

%\begin{enumerate}
%\item \tm{0 \in \NAT}
%\item \tm{m \in \NAT \Rightarrow \tmi{succ}(m) \in \NAT}.
%\end{enumerate}

%\newslide

%Oops, we forgot to exclude negative numbers!

%\begin{enumerate}
%\item \tm{0 \in \NAT}
%\item \tm{m \in \NAT \Rightarrow \suc(m) \in \NAT}.
%\item \tm{m \in \NAT \Rightarrow \suc(m) \not= 0}
%\end{enumerate}

%\newslide

%We also need one more axiom.

%\begin{enumerate}
%\item \tm{0 \in \NAT}
%\item \tm{m \in \NAT \Rightarrow \suc(m) \in \NAT}.
%\item \tm{m \in \NAT \Rightarrow \suc(m) \not= 0}
%\item \tm{\suc(m) = \suc(n) \Rightarrow m = n}
%\end{enumerate}

%\newslide

%Oh my gosh!  I just noticed that the nonnegative real numbers, the rational excluding
%negative integers,
%and lots of other sets, all satisfy these axioms.  I forgot the induction
%axiom!

%\begin{enumerate}
%\item \tm{0 \in \NAT}
%\item \tm{m \in \NAT \Rightarrow \suc(m) \in \NAT}.
%\item \tm{m \in \NAT \Rightarrow \suc(m) \not= 0}
%\item \tm{\suc(m) = \suc(n) \Rightarrow m = n}
%\item For any predicate \tm{P} in the first order predicate calculus,
%\tm{(P(0) \wedge (P(m) \Rightarrow P(\suc(m))) \Rightarrow
%(\forall n \in \NAT.P(n))}
%\end{enumerate}

%The induction axiom insures that \NAT \, is the smallest set that satisfies
%the above axioms.

%\newslide

%While you were looking at the previous page, I remembered that there are
%some completeness issues with the axioms for number theory.  The axioms
%on the previous page are satisfied by an infinite number of sets, most of
%which are a bit weird.

%We have to allow \tm{P} to be any predicate.

%\begin{enumerate}
%\item \tm{0 \in \NAT}
%\item \tm{m \in \NAT \Rightarrow \suc(m) \in \NAT}.
%\item \tm{m \in \NAT \Rightarrow \suc(m) \not= 0}
%\item \tm{\suc(m) = \suc(n) \Rightarrow m = n}
%\item For any predicate \tm{P},
%\tm{(P(0) \wedge (P(m) \Rightarrow P(\suc(m))) \Rightarrow
%(\forall n \in \NAT.P(n)))}
%\end{enumerate}

%\newslide

%Since we have no way of expressing more than a countable subset of the
%predicates on \NAT, perhaps it is better to restate the final axiom in terms
%of sets.

%\begin{enumerate}
%\item \tm{0 \in \NAT}
%\item \tm{m \in \NAT \Rightarrow \suc(m) \in \NAT}.
%\item \tm{m \in \NAT \Rightarrow \suc(m) \not= 0}
%\item \tm{\suc(m) = \suc(n) \Rightarrow m = n}
%\item \tm{(X \subseteq \NAT \wedge 0 \in X
%\wedge (m \in X \Rightarrow \suc(m) \in X))
%\Rightarrow X = \NAT}
%\end{enumerate}

%\newslide

%Are you really sure we finally have a correct definition
%for the natural numbers?

%(Personally, I am not even sure I finally managed to get my parentheses
%balanced correctly.)

%\newslide

%Let's try another approach.

%Consider a category, \tmi{US} (for unary system), where

%\begin{description}
%\item[An Object] is any triple \tm{(S,a,f)} such that
%\begin{itemize}
%\item \tm{S} is a set,
%\item \tm{a \in s}, and
%\item \tm{f: S \rightarrow S} is a function.
%\end{itemize}
%\item[An Arrow] is any homomorphism between objects. 
%\end{description}

%This category has initial objects, which are all isomorphic to
%\tm{(\NAT, 0, \suc)}.

%\newslide

%To see that \tm{(\NAT, 0, \suc)} is an initial object, it is sufficient
%to show that for any other object, \tm{(S,a,f)}, in \tmi{US}
%there is a unique homomorphism from \tm{(\NAT, 0, \suc)} to
%\tm{(S,a,f)}.

%Clearly, \tm{h} defined by

%\begin{codenott}
%\tm{h(0) = a}\\
%\tm{\forall n \in \NAT.h(\suc(n)) = f(h(n))}
%\end{codenott}

%is the unique homomorphism.

%\newslide

\section*{Diagrams, Cones, Limits and their Duals}

A \definition{diagram} in a category \CCC is a collection of objects
in \CCC together with some
(or all or none) of the arrows between those objects.

We say a diagram \definition{commutes} if whenever there are two
(or more) distinct
paths through the diagram from some object \tm{A} to some (possible
other)
object \tm{B}, the composition of the arrows along one path is equal
to
the composition of the arrows along the other path(s).

\newpage

A \definition{cone} for a diagram \tm{D} in a category \CCC is a
\CC-\tm{\unmath{object}}, \tm{A}, together with an arrow
\tm{f_{i}: A \rightarrow B_{i}} for each
object \tm{B_{i}} in \tm{D}, such that for any arrow
\tm{g: B_{i} \rightarrow B_{j}} in \tm{D}, the following holds.\\

\begin{codenott}
\tm{g \circ f_{i} = f_{j}}
\end{codenott}

\vs

We will use the notation \tm{\{f_{i}: A \rightarrow B_{i}\}} for
the cone
described above, and call the cone a
\definition{$D$-cone} to indicate that it is a cone for the diagram
\tm{D}.

\newpage

We can express the equation \tm{g \circ f_{i} = f_{j}} from
the previous page by saying that
\begin{cdiag}
\obj(1,1)[B1]{\tm{B_{i}}}
\obj(6,6)[A]{\tm{A}}
\obj(11,1)[B2]{\tm{B_{j}}}
\mor{A}{B1}{\tm{f_{i}}}[-1,0]
\mor{A}{B2}{\tm{f_{j}}}
\mor{B1}{B2}{\tm{g}}
\end{cdiag}
commutes.

The above commutative diagram is a different diagram than \tm{D}
but
is useful for describing \tm{D}.

\newpage

A \definition{limit} for a diagram \tm{D} (when one exists) is a
\tm{D}-cone
\tm{\{f_{i}: A \rightarrow B_{i}\}} with the
property that for any \tm{D}-cone
\tm{\{f^{\prime}_{i}: A^{\prime} \rightarrow B_{i}\}} there is
exactly
one arrow \tm{f: A^{\prime} \rightarrow A} such
\begin{cdiag}
\obj(1,6)[C1]{\tm{A^{\prime}}}
\obj(6,1)[B]{\tm{B_{i}}}
\obj(11,6)[C2]{\tm{A}}
\mor{C1}{B}{\tm{f^{\prime}_{i}}} [-1,0]
\mor{C2}{B}{\tm{f_{i}}}
\mor{C1}{C2}{\tm{f}}
\end{cdiag}
commutes for every object \tm{B_{i}} in \tm{D}.

\vs

That is, a limit is a final object in a category where the objects are the cones over a given diagram.

\newpage

As a trivial example, consider a limit of an empty diagram, \tm{D},
in a category \CC.

Since the empty diagram has no objects or arrows, a \tm{D}-cone is
just an object, and every object of \CCC is a \tm{D}-cone.

It follows that a limit of an empty diagram is a final object,
since every other \tm{D}-cone has a unique arrow to this object.

\newslide

\subsection*{Diagrams, Cones, Limits}

A \definition{cocone} is the dual of a cone.  Hence:
\begin{quote}
A \definition{cocone} for a diagram \tm{D} in a category \CCC is a
\CC-\tm{\unmath{object}}, \tm{A}, together with an arrow
\tm{f_{i}: B_{i} \rightarrow A} for each
object \tm{B_{i}} in \tm{D}, such that for any arrow
\tm{g: B_{i} \rightarrow B_{j}} in \tm{D}, the following holds.
\begin{codenott}
\tm{f_{i} = f_{j} \circ q}
\end{codenott}
\end{quote}
A \definition{colimit} is the dual of a limit.

\paragraph{\textcolor{black}{Quiz:}}

\color{QuoteColor}
\begin{enumerate}
\item What is a colimit of an empty diagram, and why.
% An initial object.
\item Describe a limit of a diagram consisting of a single object.
% Any isomorphic object and arrow forming half of the isomorphism.
\end{enumerate}
\color{black}

\newslide

\section*{Sums and Products}

We have previously talked about sums (disjoint unions) and products
(cartesian cross products) and their relationship to \Has algebraic
datatypes.  We will now consider these concepts in terms of category
theory.

There are three advantages to the categorical view.  First, it
removes the
issue of representation.  For example, a product can be represented
in any
of a number of ways.  Our categorical definition of category will
not
specify a representation, but will instead yield all possible
(isomorphic)
products.  Second, the concepts of sum and product will extend to
categories
where object need not be sets or even similar to sets.  Third, we
sill see
that product and sum are dual concepts.

A product will be an initial object of a suitable category and a
sum will be
a final object of a similar category.

\newpage

In order to avoid a specific representation for a product, we will
actually consider a product of \tm{A} and \tm{B} to be an object
\tm{C} together with two arrows \tm{\pi_{1}: C \rightarrow A} and
\tm{\pi_{2}: C \rightarrow B}.  In the case of \tmi{Set}, these arrows
will be the projection functions (e.g. \tc{fst} and \tc{snd} is
\textit{Haskell}).
\begin{cdiag}
\obj(1,1)[A]{\tm{A}}
\obj(6,1)[C]{\tm{C}}
\obj(11,1)[B]{\tm{B}}
\mor{C}{A}{\tm{\pi_{1}}}[-1,0]
\mor{C}{B}{\tm{\pi_{2}}}
\end{cdiag}

Similarly, a sum of \tm{A}
and \tm{B} will be an object \tm{C} together with a pair of arrows,
which will be injection functions in the case of \tmi{Set}.
\begin{cdiag}
\obj(1,1)[A]{\tm{A}}
\obj(6,1)[C]{\tm{C}}
\obj(11,1)[B]{\tm{B}}
\mor{A}{C}{\tm{i_{1}}}
\mor{B}{C}{\tm{i_{2}}}[-1,0]
\end{cdiag}

\newpage

A \definition{product} of two objects, \tm{A} and \tm{B} is a limit
of the diagram \tm{D} consisting of the two objects \tm{A} and \tm{B}
and no arrows.

A \definition{coproduct} (or \definition{sum})
of two objects, \tm{A} and \tm{B} is a colimit
of the diagram \tm{D} consisting of the two objects \tm{A} and \tm{B}
and no arrows.

A category may or may not have products and coproducts.

\newslide

Let us express the definition of a produce in another way.

A product of objects \tm{A} and \tm{B} is an object \tm{A \times B}
together with two arrows \tm{\pi_{1}:  A \times B \rightarrow A} and
\tm{\pi_{2}:  A \times B \rightarrow B},

\begin{cdiag}
\obj(1,1)[A]{\textmath{A}}
\obj(11,1)[X]{\textmath{A \times B}}
\obj(21,1)[B]{\textmath{B}}
\mor{X}{A}{\textmath{\pi_1}}[-1,0]
\mor{X}{B}{\textmath{\pi_2}}
\end{cdiag}

such that for any object \tm{C} with arrows \tm{a: C \rightarrow A} and
\tm{b: C \rightarrow B} 
(that is, any cone for the diagram consisting of \tm{A} and \tm{B})
there exists a unique arrow, which we will call
\tm{<a,b>}, such that the diagram on the following page commutes.

\newslide

\begin{cdiag}
\obj(11,11)[C]{\tm{C}}
\obj(1,1)[A]{\textmath{A}}
\obj(11,1)[X]{\textmath{A \times B}}
\obj(21,1)[B]{\textmath{B}}
\mor{X}{A}{\textmath{\pi_1}}[-1,0]
\mor{X}{B}{\textmath{\pi_2}}
\mor{C}{A}{\textmath{a}}[-1,0]
\mor{C}{B}{\textmath{b}}
\mor{C}{X}{\textmath{<a,b>}}[-1,1]
\mor{C}{X}{\textmath{!}}[1,1]
\end{cdiag}

The dotted arrow asserts that the arrow must exist and the \; \tm{!} \;
asserts that the arrow is unique.

\newpage

The diagram on the previous
page states that there must exist a unique arrow, \tm{<a,b>},
such that
\begin{mcode}
\tm{a = \pi_{1} \circ <a,b>}\\
\tm{b = \pi_{2} \circ <a,b>}
\end{mcode}

\newslide

Quiz:  \textcolor{QuoteColor}{In the category \tmi{Set},
what happens if either or both sets
in a product are empty?  What happens if either or both are singleton
sets?}

Quiz:  \textcolor{QuoteColor}{Prove that if a pair of object
in any category has two (or more)
distinct products then the products are isomorphic.}

\newslide

In \tmi{Set}, by considering the case where \tm{C = \tm{1}}
(a singleton set), we can see that a ``value'' in \tm{A \times B}
must be able to store any value from \tm{A} and any value from \tm{B},
since the arrows \tm{a} and \tm{b} can ``select'' any pair of values
in \tm{A} and \tm{B} respectively.

\begin{cdiag}
\obj(11,11)[C]{\tm{1}}
\obj(1,1)[A]{\textmath{A}}
\obj(11,1)[X]{\textmath{A \times B}}
\obj(21,1)[B]{\textmath{B}}
\mor{X}{A}{\textmath{\pi_1}}[-1,0]
\mor{X}{B}{\textmath{\pi_2}}
\mor{C}{A}{\textmath{a}}[-1,0]
\mor{C}{B}{\textmath{b}}
\mor{C}{X}{\textmath{<a,b>}}[-1,1]
\mor{C}{X}{\textmath{!}}[1,1]
\end{cdiag}

\newpage

On the other hand, if \tm{A \times B} could store any additional
information, then the arrow \tm{<a,b>} would not be unique.

\begin{cdiag}
\obj(11,11)[C]{\tm{1}}
\obj(1,1)[A]{\textmath{A}}
\obj(11,1)[X]{\textmath{A \times B}}
\obj(21,1)[B]{\textmath{B}}
\mor{X}{A}{\textmath{\pi_1}}[-1,0]
\mor{X}{B}{\textmath{\pi_2}}
\mor{C}{A}{\textmath{a}}[-1,0]
\mor{C}{B}{\textmath{b}}
\mor{C}{X}{\textmath{<a,b>}}[-1,1]
\mor{C}{X}{\textmath{!}}[1,1]
\end{cdiag}

Furthermore, while a product must contain exactly the right amount
of information, there is no restriction on the way this information
is ``represented''.  Any product is as good as any other, and
they are all isomorphic to each other.

\newpage

A sum (or coproduct) of objects \tm{A} and \tm{B} is an object \tm{A + B}
together with two arrows \tm{i_{1}:  A \rightarrow A + B} and
\tm{i_{2}: B \rightarrow A + B}
such that for any object \tm{C} with arrows \tm{a: A \rightarrow C} and
\tm{b: B \rightarrow C} 
(that is, any cocone for the diagram consisting of \tm{A} and \tm{B})
there exists a unique arrow, which we will call
\tm{[a,b]} such that the following diagram commutes.

\begin{cdiag}
\obj(11,11)[C]{\textmath{C}}
\obj(1,1)[A]{\textmath{A}}
\obj(11,1)[X]{\textmath{A \textcolor{OtherColor}{+} B}}
\obj(21,1)[B]{\textmath{B}}
\mor{A}{X}{\textmath{\textcolor{OtherColor}{i_1}}}
\mor{B}{X}{\textmath{\textcolor{OtherColor}{i_2}}}[-1,0]
\mor{A}{C}{\textmath{a}}
\mor{B}{C}{\textmath{b}}[-1,0]
\mor{X}{C}{\textmath{!}}[1,1]
\mor{X}{C}{\textmath{\textcolor{OtherColor}{[a,b]}}}[-1,1]
\end{cdiag}

\newslide

Notice that the sum figure is the same as the product figure except
for the name changes, shown in \textcolor{OtherColor}{PineGreen},
and the direction of the arrows, all of which are reversed.

This again illustrates the duality of sums and products.

\newslide

To get a better feel for sums,
let's consider some special cases of sums (disjoint unions) in \tmi{Set}.

Since a sum of objects \tm{A} and \tm{B} is an object \tm{A + B}
together with two arrows \tm{i_{1}:  A \rightarrow A + B} and
\tm{i_{2}: B \rightarrow A + B}, we will consider the sum figure
assuming that the bottom part of the figure,

\begin{cdiag}
\obj(1,1)[A]{\textmath{A}}
\obj(11,1)[X]{\textmath{A + B}}
\obj(21,1)[B]{\textmath{B}}
\mor{A}{X}{\textmath{i_1}}
\mor{B}{X}{\textmath{i_2}}[-1,0]
\end{cdiag}

is fixed, but that
consider various possible cases for \tm{C} and the arrows to \tm{C}.

\newslide

\begin{cdiag}
\obj(11,11)[C]{\textmath{C}}
\obj(1,1)[A]{\textmath{A}}
\obj(11,1)[X]{\textmath{A + B}}
\obj(21,1)[B]{\textmath{B}}
\mor{A}{X}{\textmath{i_1}}[-1,0]
\mor{B}{X}{\textmath{i_2}}
\mor{A}{C}{\textmath{a}}[-1,0]
\mor{B}{C}{\textmath{b}}
\mor{X}{C}{\textmath{!}}[1,1]
\mor{X}{C}{\textmath{[a,b]}}[-1,1]
\end{cdiag}

\begin{enumerate}
\item Suppose \tm{C=A} and \tm{a=id_A}, with \tm{b} being arbitrary.
(The other way around is similar.)
\item Suppose \tm{A} and \tm{B} are nonempty and \tm{C=\{1,2\}},
\tm{a=\tc{const}~1} and \tm{b=\tc{const}~2}.
\end{enumerate}

\newslide

\begin{enumerate}
\item Suppose \tm{C=A} and \tm{a=id_A}, with \tm{b} being arbitrary.
(The other way around is similar.)

\paragraph{Observation:}  For any \tm{a \in A}, we can insert the value
into \tm{A+B} and get it back out again.  Similarly for any \tm{b \in B}.
Hence, at the very least, \tm{A+B} can store either an element of \tm{A}
or an element of \tm{B}, as we would expect from any sort of union.

\item Suppose \tm{A} and \tm{B} are nonempty and \tm{C=\{1,2\}},
\tm{a=\tc{const}~1} and \tm{b=\tc{const}~2}.

\paragraph{Observation:}
This shows that we can always determine which insertion function
produced the value in \tm{A+B}.  Notice that this works even in
the case where \tm{A=B}.
\end{enumerate}

\newslide

\begin{cdiag}
\obj(11,11)[C]{\textmath{C}}
\obj(1,1)[A]{\textmath{A}}
\obj(11,1)[X]{\textmath{A {+} B}}
\obj(21,1)[B]{\textmath{B}}
\mor{A}{X}{\textmath{{i_1}}}[-1,0]
\mor{B}{X}{\textmath{{i_2}}}
\mor{A}{C}{\textmath{a}}[-1,0]
\mor{B}{C}{\textmath{b}}
\mor{X}{C}{\textmath{!}}[1,1]
\mor{X}{C}{\textmath{{[a,b]}}}[-1,1]
\end{cdiag}

\begin{enumerate}
\item Suppose that \tm{C=A+B} and \tm{a=i_1} and \tm{b=i_2}.
Clearly \tm{[a,b]=id_{A+B}} will work.  Since \tm{[a,b]} must
be unique, nothing else can work.
\item Consider the case where \tm{A=\emptyset} and/or \tm{B=\emptyset}.
\end{enumerate}

\newslide

\begin{enumerate}
\item Suppose that \tm{C=A+B} and \tm{a=i_1} and \tm{b=i_1}.
Clearly \tm{[a,b]=id_{A+B}} will work.  Since \tm{[a,b]} must
be unique, nothing else can work.

\paragraph{Observation:}
\tm{A+B} can't contain any more than just the necessary information,
since there is no extra information that can be changed in the arrow
from \tm{A+B} back to itself.  If \tm{A+B} contained some extra junk,
then we could get additional arrows from \tm{A+B} back to itself
that would allow the diagram to commute.

\item Consider the case where \tm{A=\emptyset} and/or \tm{B=\emptyset}.

\paragraph{Observation:}  Let us do a general category theoretic proof
here, so we can apply the result more generally.
\end{enumerate}

\newslide

\begin{cdiag}
\obj(11,11)[C]{\textmath{C}}
\obj(1,1)[A]{\tm{0}}
\obj(11,1)[X]{\textmath{A {+} B}}
\obj(21,1)[B]{\textmath{B}}
\mor{A}{X}{\textmath{i_1}}[-1,0]
\mor{B}{X}{\textmath{i_2}}
\mor{A}{C}{\textmath{a}}[-1,0]
\mor{B}{C}{\textmath{b}}
\mor{X}{C}{\textmath{!}}[1,1]
\mor{X}{C}{\textmath{{[a,b]}}}[-1,1]
\end{cdiag}


By definition, there is exactly one arrow from \tm{0} to any object.
Therefore the left triangle must commute.  That is, \tm{i_1;[a,b] = a}
because \tm{i_1;[a,b]} and \tm{a} are both the (same) unique arrow
from \tm{0} to \tm{C}.

Therefore, a possible sum is \tm{A+B = B} so \tm{[a,b] = b} for any
\tm{C} and \tm{b}.
Any other sum must be isomorphic.  (This is the dual of a property
for products that you
proved in a quiz.)

\newslide

\begin{cdiag}
\obj(11,11)[C]{\textmath{C}}
\obj(1,1)[A]{\tm{0}}
\obj(11,1)[X]{\textmath{A {+} B}}
\obj(21,1)[B]{\textmath{B}}
\mor{A}{X}{\textmath{{!}}}[-1,0]
\mor{B}{X}{\textmath{{i_2}}}
\mor{A}{C}{\textmath{!}}[-1,0]
\mor{B}{C}{\textmath{b}}
\mor{X}{C}{\textmath{!}}[1,1]
\mor{X}{C}{\textmath{{[a,b]}}}[-1,1]
\end{cdiag}

It follows that \tm{\tm{0} + B \cong B} for any \tm{B} in any category
with sums.  Similarly, \tm{A + \tm{0} \cong A} and \tm{\tm{0} + \tm{0} \cong
\tm{0}}.

\newslide

Since \tm{\tm{0} + B \cong B}, by duality (reversing the arrows to produce the
opposite category) \tm{\tm{1} \times B \cong B} in any category.  This proves:

In \tmi{Set} \, this gives us the obvious results.

In the predicate calulus, this says \tm{true \wedge B \meq B} and
\tm{false \vee B \meq B}.

In powersets, where \tm{S} is the universal set, we have
\tm{S \cap B = B} and \tm{\emptyset \cup B = B}.

With partial orders, where everything is defined,
\tm{\top \sqcap B = B}.  (That is the minimum of the maximal element and
\tm{B} is \tm{B}.  For
example \tm{\infty \min B = B} with integers --- or reals.)
Similarly \tm{\bot \sqcup B = B}.

Lots of similar results in other categories also follow.

\newpage

Remember that \tm{\tm{0} \times B \cong \tm{0}} in the category of sets.

Similar results apply in lots of other categories, for example
\tm{false \wedge B = false}, \tm{\emptyset \cap B = \emptyset}, etc.

The duals of these last properties, \tm{true \vee B = true} and
\tm{S \cup B = S} also hold.

However, in all of these cases, we have categories
where there is a unique initial object (or unique final object in the dual
cases), and
no other object has an arrow to the unique initial object (or in the dual
cases, from the unique final object).

The dual of our first result above (in the
category of sets) is \tm{\tm{1} + B \cong \tm{1}}, which
is not true in the category of sets.  In this category there are
multiple final objects, and every object except the initial object
has arrows from every final object.

\newslide

\section*{Polynomial Functors, etc.}
 
Let \CCC and \DDD be categories and let \tm{A} be an object of \DD\@.
A \definition{constant functor} \tm{\tmb{K}_{A}} is a functor
\tm{\tmb{K}_{A}: \CCC \rightarrow \DD} that maps every object of \CCC to the
object \tm{A} and every arrow of \CCC to \tm{id_{A}}.

\vs

We may write \tmb{1} for \tm{\tmb{K}_{A}} in the special case where \tm{A}
is a final object and we really don't care which final object \tm{A} is.

\vs

You should confirm that a constant functor is indeed a functor.

\newslide

An \definition{endofunctor} is a functor from some category \CCC
to the same category, \CC.

\vs

The \definition{identity functor}, \tmb{Id}, is an endfunctor 
that maps every object to itself and every arrow to itself.  The
identity functor for a
category is just the identity arrow for that category viewed as an
object in the category
\tmi{Cat}.

\newslide

Assume that all binary products exist in a category \CC.  That is,
for any objects \tm{A} and \tm{B}, \tm{A \times B} exists.

\vs

Let \tm{F} and \tm{G} be any pair of endofunctors defined on \CC.
We define the endofunctor \tm{(F \times G)} by

\begin{codenott}
\tm{(F \times G) \, A = F \, A \times G \, A}\\
\tm{(F \times G) \, h = \, <F \, h \circ \pi_{1}, G \, h \circ \pi_{2}>}
\end{codenott}

\vs

Notice that if \tm{h: A \rightarrow B} then
\tm{(F \times G) \, h : (F A \times G A) \rightarrow (F B \times G B)}.

\newslide

The definition of the product of functors
generalizes in a straightforward manner to products of any finite degree,
but we will restrict our attention to binary products in order to simplify the presentation.

Similarly, we will consider only binary coproducts (sums) of functors.

\newslide

Assume that all binary coproducts (sums) exist in a category \CC.  That is,
for any objects \tm{A} and \tm{B}, \tm{A + B} exists.

\vs

Let \tm{F} and \tm{G} be any pair of endofunctors defined on \CC.
We define the endofunctor \tm{(F + G)} by

\begin{codenott}
\tm{(F + G) \, A = F \, A + G \, A}\\
\tm{(F + G) \, h = \, [i_{1} \circ F \, h, i_{2} \circ G \, h]}
\end{codenott}

\vs

Notice that if \tm{h: A \rightarrow B} then
\tm{(F + G) \, h : (F A + G A) \rightarrow (F B + G B)}.

\newslide

Let's see how these ideas carries over to \Ha \, and \textit{Haskell}.

Recall that the objects of \Ha \, are pointed cpos and each
object correspond to
a particular ground type (i.e. a type with no
type variables) in \textit{Haskell}.  The
arrows in \Ha \, are continuous functions corresponding to
\textit{Haskell}
functions that are not polymorphic or overloaded.

Let's ignore the \bottom ~ element that is included in every object
in \Ha, and further ignore the fact that every \Ha \, object has at least one
value other than \bottom, so \Ha \, has no initial or final objects.

\newslide

To save room, we will use the symbol $\bowtie$ to mean ``corresponds to''.

With \Ha \, objects and \textit{Haskell} types,\\
\begin{codenott}
\tm{<f,g>} ~ \= 
\bla{$\bowtie$} ~ \= \tc{either f g} ~
\= \bla{(See page 42 of Bird.)}\kill
\tm{A \times B} \> \bla{$\bowtie$} \> \tc{(A, B)}\\
\tm{A + B} \> \bla{$\bowtie$} \> \tc{Either A B}\\
\end{codenott}

With \Ha \, arrows and \textit{Haskell} functions,
\begin{codenott}
\tm{(F \times G) A} ~ \= 
\bla{$\bowtie$} ~ \= \tc{either f g} ~
\= \bla{(See page 42 of Bird.)}\kill
\tm{<f,g>} \>
\bla{$\bowtie$} \> \tc{pair(f,g)} \> \bla{(See page 42 of Bird.)}\\
\tm{[f,g]} \> \bla{$\bowtie$} \> \tc{either f g} \>
\bla{which is the same as}\\
\> \>\tc{case(f,g)} \> \bla{(See page 46 of Bird.)}
\end{codenott}

\newslide

With \Ha \, functor operators and \textit{Haskell} functions,
the arrow part of the functor gives\\
\begin{codenott}
\tm{(F \times G) A} ~ \= 
\bla{$\bowtie$} ~ \= \tc{cross(fmap\tm{^{F}},fmap\tm{^{G}})} ~
\= \bla{(See page 42 of Bird.)}\kill
\tm{(F \times G)} \> \bla{$\bowtie$} \>
\tc{cross(fmap\tm{^{F}},fmap\tm{^{G}})} \> \bla{(See page 42 of Bird.)}\\
\tm{(F + G)} \> \bla{$\bowtie$} \>
\tc{plus(fmap\tm{^{F}},fmap\tm{^{G}})} \> \bla{(See page 46 of Bird.)}
\end{codenott}\\
while for the object part we have\\
\begin{codenott}
\tm{(F \times G) A} ~ \= 
\bla{$\bowtie$} ~ \= \tc{either f g} ~
\= \bla{(See page 42 of Bird.)}\kill
\tm{(F \times G) A} \> \bla{$\bowtie$} \>
\tc{(F A, G A)}\\
\tm{(F + G) A} \> \bla{$\bowtie$} \>
\tc{Either (F A) (G A)}
\end{codenott}

Recall Bird's comment, in \P 2 of page 47, that 
``\textcolor{QuoteColor}{The algebraic properties of} \tc{case}
\textcolor{QuoteColor}{and} \tc{plus}
\textcolor{QuoteColor}{are dual to those of} \tc{pair}
\textcolor{QuoteColor}{and} \tc{cross}\textcolor{QuoteColor}{.}''

\newslide

Let's consider a full \textit{Haskell} datatype declaration.
\begin{code}
data Perhaps a = Nope | Only a | Both Int a
\end{code}

We have\\
\begin{codenott}
\tc{Perhaps} $\bowtie$
\tm{(\tmb{1} + \Id +(\tmb{K}_{\tc{Int}} \times \Id))_{\bot}}
\end{codenott}\\
and could make \tc{Perhaps} an instance of \tc{Functor} by
\begin{code}
instance Functor Perhaps where\\
   fmap f Nope = Nope\\
   fmap f (Only a) = Only (f a)\\
   fmap f (Both n a) = Both n (f a)
\end{code}

\newslide

Notice:
\begin{enumerate}
\item We used ``\tc{|}'' directly rather than \tc{Either}, in part because we
have a sum of degree three instead of degree two, and in part because
\tc{Either}
is just a way to encapsulate ``\tc{|}'' and supply data constructors (tags)
\tc{Left} and \tc{Right}.
\item In \textit{Haskell}, \tc{Nope}, \tc{Only} and \tc{Both} are names of
constructors (instead of \tm{i_1} and \tm{i_1}, or \tc{Left} and \tc{Right})
as well as tags used in pattern matching.  These names have no role in the
structure of the functor.
\item We have added the \bottom, whose existence we now acknowledge.  We will
add a \bottom \, to the functor corresponding to the right side of each
\textit{Haskell} \tc{data} statement. 
\end{enumerate}

\newslide

A \definition{polynomial functor} is any functor formed using constant functors,
\Id \, and \tm{+} and \tm{\times}.

\vs

In \Has we need to be able to lift functors, using \bottom.  This is done
in the obvious way and we will skip the details.

\vs

\Has also allows function types to be used in datatype declarations.  While
category theory supports this, we will skip the details, since lifted
polynomial functors are sufficient for illustrating the ideas that we will
be exploring.

\newslide

\section*{F-Algebras}

Let \CC \, be a category and \tm{F: {\CC} \rightarrow {\CC}} \, be
an endofunctor.

An \definition{F-algebra} is a pair \tm{(A,a)} where \tm{A} is a
\CC-\tm{\unmath{object}} and \tm{a: F \, A \rightarrow A} is a
\CC-\tm{\unmath{arrow}}.  The object \tm{A} is called the
\definition{carrier} of the F-algebra.

\newslide

For example, let \tmi{Natural} be the set of natural numbers in the
category
\tmi{Set}.  We will now define an F-algebra on the carrier \tmi{Natural}
with two ``operators''.  The first operator is really a constant,
which can
be viewed as an operator that takes zero arguments.  This operator,
which we
will call \tmi{zero}, ``returns'' the natural number \tm{0}.  The
other
operator is a binary operator, \tmi{plus}, that returns the sum of
its two
arguments.  This gives us a group with an identity element and a
binary
operation.

This example could easily be extended to include more constants and
operators, such as the constant \tmi{one} and the binary operator
\tmi{times}, as well up operators of different arity, such as a
unary
operator, \tmi{negate}, and operators of arity three or more.

\newslide

Let \tm{F: F \, \tmi{Set} \rightarrow \tmi{Set}} be the functor
defined by \tm{F_{0} \, S = 1 + S \times S} and
\tm{F_{1} \, f = [\tmi{id}_{1},<f,f>]}.

We now define the arrow \tm{a: F \, A \rightarrow A} as follows.

\begin{codenott}
\tm{a (i_{1} (x)) = 0}\\
\tm{a (i_{2} (a,b)) = a + b}
\end{codenott}

\newslide

Let us revisit this example in \Has.

We will need to assume that we have a \textit{Haskell} type
\tcx{Natural}
that contains the natural numbers and nothing else.  This is not
actually
possible in \textit{Haskell}, because every type contains a $\bottom$
value,
but we can pretend.  Similarly, for a later example, we will assume
that
\textit{Haskell} contains a type \tcx{PosReal}, which includes all
positive
real numbers (so it has an uncountable number of values), does not
contain
$\bot$.  We will distinguish that fake types having no $\bottom$
element by
using a \tcx{special color (PineGreen)}.

\newslide

We start by defining the functor (type constructor) \tcx{F}.

\begin{code}
datatype \tcx{F} a = Ident \tcx{|} Op a a\\
\\
instance Functor \tcx{F} where\\
   fmap f Ident = Ident\\
   fmap f (Op x y) = Op (f x) (f y)
\end{code}

As the color suggests, we assume that no $\bottom$ element is added
by the
type constructor \tcx{F}\@.  We also create labels, \tc{Zero} and
\tc{Add}
for the two cases in the domain sum.

\newslide

We now define

\begin{code}
a :: \tcx{F Natural} -> \tcx{Natural}\\
a Ident = 0\\
a (Op x y) = x + y
\end{code}

Notice how a single function, \tc{a}, defined both (or in general
case, all) operations at once.

\newslide

Using the same functor, \tcx{F}, we also can define an F-algebra
over the
positive real numbers with a constant \tmi{one} and an operation
\tmi{times}.

\begin{code}
a :: \tcx{F PosReal} -> \tcx{PosReal}\\
a Ident = 1\\
a (Op x y) = x * y
\end{code}

\newslide

\begin{cdiag}
\obj(1,1)[B]{\tcx{F \, Real}}
\obj(1,11)[A]{\tcx{F \, Natural}}
\obj(21,1)[D]{\tcx{Real}}
\obj(21,11)[C]{\tcx{Natural}}
\mor{A}{B}{\tc{fmap h}}
\mor{C}{D}{\tc{h}}
\mor{A}{C}{\tc{a}}
\mor{B}{D}{\tc{b}}
\end{cdiag}


%\section*{F-Algebras}

%Let \CC \, be a category and \tm{F: {\CC} \rightarrow {\CC}} \, be
%an endofunctor.

%An \definition{F-algebra} is a pair \tm{(A,a)} where \tm{A} is a
%\CC-\tm{\unmath{object}} and \tm{a: F \, A \rightarrow A} is a
%\CC-\tm{\unmath{arrow}}.  The object \tm{A} is called the
%\definition{carrier} of the F-algebra.

%\newslide

%For example, let \tmi{Natural} be the set of natural numbers in the
%category
%\tmi{Set}.  We will now define an F-algebra on the carrier \tmi{Natural}
%with two ``operators''.  The first operator is really a constant,
%which can
%be viewed as an operator that takes zero arguments.  This operator,
%which we
%will call \tmi{zero}, ``returns'' the natural number \tm{0}.  The
%other
%operator is a binary operator, \tmi{plus}, that returns the sum of
%its two
%arguments.  This gives us a group with an identity element and a
%binary
%operation.

%This example could easily be extended to include more constants and
%operators, such as the constant \tmi{one} and the binary operator
%\tmi{times}, as well up operators of different arity, such as a
%unary
%operator, \tmi{negate}, and operators of arity three or more.

%\newslide

%Let \tm{F: F \, \tmi{Set} \rightarrow \tmi{Set}} be the functor
%defined by \tm{F_{0} \, S = 1 + S \times S} and
%\tm{F_{1} \, f = [\tmi{id}_{1},<f,f>]}.

%We now define the arrow \tm{a: F \, A \rightarrow A} as follows.

%\begin{codenott}
%\tm{a (i_{1} (x)) = 0}\\
%\tm{a (i_{2} (a,b)) = a + b}
%\end{codenott}

%\newslide

%Let us revisit this example in \Has.

%We will need to assume that we have a \textit{Haskell} type
%\tcx{Natural}
%that contains the natural numbers and nothing else.  This is not
%actually
%possible in \textit{Haskell}, because every type contains a $\bottom$
%value,
%but we can pretend.  Similarly, for a later example, we will assume
%that
%\textit{Haskell} contains a type \tcx{PosReal}, which includes all
%positive
%real numbers (so it has an uncountable number of values), does not
%contain
%$\bot$.  We will distinguish that fake types having no $\bottom$
%element by
%using a \tcx{special color (PineGreen)}.

%\newslide

%We start by defining the functor (type constructor) \tcx{F}.

%\begin{code}
%datatype \tcx{F} a = Ident \tcx{|} Op a a\\
%\\
%instance Functor \tcx{F} where\\
%   fmap f Ident = Ident\\
%   fmap f (Op x y) = Op (f x) (f y)
%\end{code}

%As the color suggests, we assume that no $\bottom$ element is added
%by the
%type constructor \tcx{F}\@.  We also create labels, \tc{Zero} and
%\tc{Add}
%for the two cases in the domain sum.

%\newslide

%We now define

%\begin{code}
%a :: \tcx{F Natural} -> \tcx{Natural}\\
%a Ident = 0\\
%a (Op x y) = x + y
%\end{code}

%Notice how a single function, \tc{a}, defined both (or in general
%case, all) operations at once.

%\newslide

%Using the same functor, \tcx{F}, we also can define an F-algebra
%over the
%positive real numbers with a constant \tmi{one} and an operation
%\tmi{times}.

%\begin{code}
%a :: \tcx{F PosReal} -> \tcx{PosReal}\\
%a Ident = 1\\
%a (Op x y) = x * y
%\end{code}

%\newslide

%\begin{cdiag}
%\obj(1,1)[B]{\tcx{F \, Real}}
%\obj(1,11)[A]{\tcx{F \, Natural}}
%\obj(21,1)[D]{\tcx{Real}}
%\obj(21,11)[C]{\tcx{Natural}}
%\mor{A}{B}{\tc{fmap h}}
%\mor{C}{D}{\tc{h}}
%\mor{A}{C}{\tc{a}}
%\mor{B}{D}{\tc{b}}
%\end{cdiag}


%The signature of an algebra is basically a collection of type
%definitions
%for the operations defined in the algebra.  To keep things simple,
%we will assume that we have only one basic type, so the type of
%each operation is determined by its arity.

%A group, for example has two operators, one, which we will call
%\tmi{Times}, of arity two and one of
%arity zero (i.e. a constnat), which we will call \tmi{Unit}.

%We can express this idea with a \textit{Haskell} datatype declaration.
%\begin{code}
%datatype Group a = Times a a | Unit
%\end{code}

%A group must also satisfy certain equations (laws), but we will
%ignore this
%for the moment.  We will also ignore the fact that algebraic
%datatypes in
%\textit{Haskell} are lifted (contain a \tm{\bot} element that we
%do not
%really want at this point).

%\newslide

%A specific group is determined by the set of elements in the group
%together
%with the specification of the two operators.  The two operators
%can
%be determined by a function from \tc{Group} \tm{A} to \tm{A}.

%For example, (ignoring \tm{\bot}) \tc{addition}, defined by

%\begin{code}
%addition :: Group Integer -> Integer\\
%addition (Times m n) = m + n\\
%addition (Unit)      = 0
%\end{code}

%defines the group consisting of the set of \tmi{Integers} under
%\tmi{addition}.

%(The group laws, which we are ignoring, would determine that
%\tc{Unit} must
%be \tm{0}.)

%Notice how the single \tc{addition} function defines both \tmi{Times}
%and
%\tmi{Unit}, and in general could define any number of operators
%for an
%algebra with more operators.

%\newslide

%Another group is defined by \tc{bool}.

%\begin{code}
%bool :: Group Bool -> Bool\\
%bool (Times a b) = a == b\\
%bool (Unit)      = True
%\end{code}

%Let \tc{h :: Integer -> Bool} be a function that we will view as
%being
%from the group of \tmi{Integers} (under \tmi{addition}) to the
%group of
%\tmi{Bools} as defined above.

%We say that \tc{h} is a \definition{homomorphism} if it maintains
%the structure
%of the algebra.  That is

%\begin{codenott}
%\tc{h 0 }\tm{\equiv}\tc{ True}\\
%\tc{h (m + n) }\tm{\equiv}\tc{ h m == h n}
%\end{codenott}

%\newslide

%There are exactly two functions with the type specified for \tc{h}
%that
%are homomorphisms, and an infinite number that are not.

%\vs

%How can we use the methods of category theory to
%specify in a general way
%that a function from one algebra to another algebra with
%the same signature as the first is a homomorphism?

%\newslide

%First, we should view the type constructor, \tc{Graph} in this
%case, as
%a functor.  This means we need a definition for \tc{fmap}, which
%for
%\tc{graph} must be defined as follows.

%\begin{code}
%instance Functor Group where\\
%\> \> fmap h (Times a b) = Times (h a) (h b)\\
%\> \> fmap h  Unit       = Unit
%\end{code}

%Now our requirements for \tc{h} to be a homomorphism can be expressed
%by the following two equations
%
%\begin{codenott}
%\tc{bool (fmap h (Times m n)) }\tm{\equiv}\\
%\> \> \> \tc{ h (addition (Times m n))}\\
%\tc{bool (fmap h Unit) }\tm{\equiv}\tc{ h (addition Unit)}
%\end{codenott}

%or the single equation

%\begin{codenott}
%\tc{bool.fmap h  }\tm{\equiv}\tc{ h.addition}
%\end{codenott}

%\newslide

%We note that by using abstract data types, which can be implemented
%in Haskell by using modules and controlling exports, we can implement
%groups in such a way that \bottom, or elements containing \bottom,
%can't be produced.

%In general, let \tm{G} be any group, where the ele

%\begin{codenott}
%\tc{bool.fmap h  }\tm{\equiv}\tc{ h.addition}
%\end{codenott}

%from the previous page says, exactly, that the following diagram
%commutes.

%\begin{cdiag}
%\obj(1,1)[B]{\tc{Group Bool}}
%\obj(1,11)[A]{\tc{Group Integer}}
%\obj(21,1)[D]{\tc{Bool}}
%\obj(21,11)[C]{\tc{Integer}}
%\mor{A}{B}{\tc{fmap h}}
%\mor{C}{D}{\tc{h}}
%\mor{A}{C}{\tc{addition}}
%\mor{B}{D}{\tc{bool}}
%\end{cdiag}

%%Let $\Omega =\{f_{1}, f_{2}, \ldots f_{n}\}

%%An algebra, $A$, is a set, $|A|$, called the carrier of the set
%together
%%with a collection of
%%Quoting from http://mathworld.wolfram.com/UniversalAlgebra.html:
%%\begin{quote}
%%\color{QuoteColor}
%%A universal algebra is a pair $\Bbb{A}=(A,(f_{i}^{A})_{i \in I})$,
%where $A$
%%and $I$ are sets and for each ,   is an operation on A. The algebra
%is finitary if each of its operations is finitary.

%%A set of function symbols (or operations) of degree  is called a
%signature (or type). Let  be a signature. An algebra A is defined
%by a domain S (which is called its carrier or universe) and a mapping
%that relates a function  to each n-place function symbol from .

%%Let  and  be two algebras over the same signature , and their
%carriers are A and B, respectively. A mapping  is called a homomorphism
%from  to  if for every  and all  ,

%%\color{black}
%%\end{quote}

%\newslide

%We note in passing that the previous several pages hold if we allow
%\bottom amd
%values containing \bottom in types \tc{Integer} and \tc{Group
%Integer}, but that we no longer have a group if these values are
%allowed.

%If we are interested in the group of integers under addition, then
%the above applies, but we should note that by using datatypes
%\tc{Integer} and
%\tc{Group Integer} we can represent values that are not in the
%sets that we really want.  We can think of these extra values (e.g.
%\tc{Times 5 \bottom})
%as ill formed.

%In a similar manner, if we are interested in addition mod 5, we
%might represent
%values using \tc{Int}s with the understanding that only the values
%\tc{0},
%\tc{1}, \tc{2}, \tc{3} and \tc{4} are valid.

\newslide

%\subsection*{Catamorphisms}

%{\tiny Add some details and examples here.}

%\newslide

%\section*{Natural Transformations}

%{\tiny Add some details and examples here.}

%\newslide

%\section*{Monads}

%{\tiny Add some details and examples here.}

%\newslide

%{\tiny Move and fix, or discard, the rest of this document.}

%\newslide

\subsubsection*{The \textit{Natural Numbers} as an Initial Object}

Consider the \textit{Haskell} datatype declaration:

\begin{code}
data Nat = Zero | Succ Nat
\end{code}

This suggests that \tc{Nat} corresponds to a domain \tmi{Nat}\,
that
satisfies the following isomorphism.

\begin{codenott}[MathColor]
$\tmi{Nat} \cong (\one + \tmi{Nat})_{\bot}$
\end{codenott}

If we define a type constructor, \tc{NatF}, which we will view as
a functor,
by

\begin{code}
data NatF nat = Zero | Succ nat
\end{code}

then we will take \tc{Nat} to be isomorphic to a type \tc{Nat2}
such that

\begin{code}
Nat2 \tm{\cong} NatF Nat2
\end{code}

\newslide

\begin{code}
newtype Mu f = In f (Mu f))
\end{code}

\newslide

\subsubsection*{Initial and Final Objects in \tmi{Grp}}

The category \tmi{Grp}\, has a final object, a single element group,
but no
initial object.

\newslide



\end{slide}

\end{document}

