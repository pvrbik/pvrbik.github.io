% !TEX root = OurExam.tex
\clearpage
\section*{\S 2 Short Answers}
Each solution in this section is worth \emph{4 points} for a total of \emph{20 points}.
\vspace{-0.5cm}
\subsection*{Question 3}
Write a java class method \texttt{int MinElement (int[] xs)} that returns the \emph{least} (or \emph{smallest}) element of the array \texttt{xs}. For instance, if we had the array \texttt{A = \{-3,4,10,-32,0\}} then \texttt{MinElement(A)} would return \texttt{-32}. \\

\AnsBox{9cm}
\vspace{-0.5cm}
\subsection*{Question 4}
Write a java expression that is equivalent to the algebraic expression:
$$ \dfrac { \dfrac {c-(a+b)} {d} \times e } {f}.$$
You may assume that the initialization 
\begin{verbatim}
    double a,b,c,d,e,f;
\end{verbatim}
has been made. \\

\AnsBox{2cm}
\clearpage

\subsubsection*{Question 5}
Write a \texttt{for} loop that, given \texttt{String s}, prints the string backwards. You may assume that the string is already assigned to the variable \texttt{s}. 

For example, given \texttt{String s = "abcde";} the \texttt{for} loop would print \texttt{edcba}.\\

\AnsBox{7cm}

\subsection*{Question 6}
Write a java class method \texttt{void AnnoyUser()} that repeatedly prompts the user for input until the string ``\texttt{banana}'' is entered. (Hint: use \texttt{SimpleInput.getString} and a \texttt{while} loop).\\

\AnsBox{9.2cm}


\subsection*{Question 7}
There are four syntax errors in the following code segment:
\begin{verbatim}
/00/  class Foo {
/01/      public void FooBar ( ) {
/02/           int xs = {1,2,3,4};
/03/
/04/           for ( i=0; i <= 4; i+=2 ) {
/05/               System.out.println( "The value of i is", i );
/06/           }
/07/
/08/          return  xs[0];
/09/       }
/10/  }
\end{verbatim}
For each error, indicate the line where the error is and correct the error (for instance you could write something like: ``Line 15 should be \texttt{double x = 2.5}'' ). \\

\AnsBox{12cm}