\documentclass[10pt]{article}

\usepackage{rotating,graphics,psfrag,epsfig,amssymb,amsmath, subfigure, amsxtra, amsthm, color}
\oddsidemargin=-0.25in
\topmargin= -0.5in
\textwidth=7.0in
\textheight=9in
\pagestyle{plain}

%\sloppy

\theoremstyle{plain}
\newtheorem{theorem}{Theorem}
\newtheorem{corollary}{Corollary}
\newtheorem{lemma}{Lemma}
\newtheorem{proposition}{Proposition}
\newtheorem{conjecture}{Conjecture}

\theoremstyle{definition}
\newtheorem{definition}{Definition}
\newtheorem{example}{Example}
\newtheorem{remark}{Remark}
\newtheorem{question}{Question}
\newtheorem{problem}{Problem}


\renewcommand{\implies}{\Rightarrow}
\newcommand{\imp}{\textbf}
\newcommand{\dom}[1]{\textsf{dom} (#1)}
\newcommand{\cod}[1]{\textsf{cod}(#1)}
\newcommand{\arr}[1]{\mathcal{#1}\mathrm{-arrows}}
\newcommand{\obj}[1]{\mathcal{#1}\mathrm{-objects}}
\newcommand{\state}[1]{\item[#1] \textcolor{white}{.} \\ \textcolor{white}{.} \\ }
\newcommand{\Lstate}[1]{\item[#1] \nl}
\newcommand{\id}[1]{id_{#1}}
\newcommand{\nl}{\textcolor{white}{.}\ }
\newcommand{\func}[3]{#1 : #2 \rightarrow #3}
\newcommand{\CC}{\mathbb{C}}
\newcommand{\DD}{\mathcal{D}}
\newcommand{\point}[1]{\begin{itemize} \item[$\cdot$] #1 \end{itemize}}
\newcommand{\blank}{\textcolor{white}{-}}
\newcommand{\fieldt}{\mathbb{Z}_q[t]}
\newcommand{\fieldtx}{\mathbb{Z}_q[t][x]}
\newcommand{\impl}{\Rightarrow}
\newcommand{\eps}{\varepsilon}
\newcommand{\brac}[1]{\left( #1 \right)}
\newcommand{\Res}[1]{\textrm{Res}_{#1} }

\newcommand{\x}{\boldsymbol{x}}
\renewcommand{\H}{\boldsymbol{H}}
\newcommand{\zero}{\boldsymbol{0}}
\newcommand{\z}{\boldsymbol{z}}
\newcommand{\F}{\boldsymbol{F}}
\newcommand{\p}{\boldsymbol{p}}
\newcommand{\q}{\boldsymbol{q}}
\newcommand{\alp}{{\boldsymbol{\alpha}}}
\newcommand{\bet}{{\boldsymbol{\beta}}}

\newcommand{\M}{\mathbf{M}}
\newcommand{\A}{\mathbf{A}}
\newcommand{\B}{\mathbf{B}}
\newcommand{\C}{\mathbf{C}}
\newcommand{\X}{\mathbf{X}}
\newcommand{\0}{\mathbf{0}}
\newcommand{\II}{\mathbf{I}}
\newcommand{\ZZ}{\mathbb{Z}}
\newcommand{\QQ}{\mathbb{Q}}

\newcommand{\eval}[2]{ \left.#1\right |_{#2}}
\newcommand{\xinit}{x_{\text{init}}}
\newcommand{\idnty}{\text{Id}}

\newcommand{\pdiff}[2]{\frac{\partial #1}{\partial #2}}

\newcommand{\FF}{\mathbf{F}}
\newcommand{\JJ}{\mathbf{J}}
\newcommand{\XX}{\mathbf{X}}

\newcommand{\Xinit}{\XX_{\text{init}}}
\newcommand{\Ginit}{G_{\text{init}}}

\renewcommand{\mod}{\text{ mod }}
\newcommand{\Mod}{\hspace{2mm}\mod}

\usepackage{algorithm}
\usepackage{algorithmic}
\usepackage{subfigure}
\usepackage{tikz}
\usepackage{multicol}

\usepackage{listings,balance}
\lstset{language=Java}

\setlength{\parindent}{0in} 

\begin{document}

%%%TITLE PAGE%%%
{\large
\begin{center}
\textsc{CS1026b Extra Exam Material}
\end{center}
}

\begin{multicols}{2}   
\section*{Strings}
To compare two strings: \texttt{s1} and \texttt{s2} do
$$\texttt{ s1.equals(s2) }$$

To get a length of a string do
$$\texttt{s1.length()}$$

To access a character (single letter) of string \texttt{s1} do
$$\texttt{s1[i]}$$
where $0 \leq \texttt{i}< \texttt{st.length()}$.

\section*{Worlds}
\texttt{World()}\\
Constructor that takes no arguments.\\

\texttt{World(int w, int h)}\\
Constructor that takes a width and height for the world.

\section*{Turtles}
\subsubsection*{Partial Constructor Summary}

\texttt{Turtle(int x, int y, World world)}\\
Constructor that puts the turtle at position (x,y) in its world, facing �up� (toward the top of the world)\\

\texttt{Turtle(World world)}\\
Constructor that puts the turtle at the center of its world, facing �up� (toward the top of the world)

\subsubsection*{Partial Method Summary}

\texttt{void backward(int pixels)}\\
Method to go backward a given number of pixels\\

\texttt{void forward(int pixels)}\\
Method to move the turtle forward the given number of pixels\\

\texttt{int getXPos()}\\
Method to get the current x position\\

\texttt{int getYPost}\\
Method to get the current y position\\

\texttt{void hide()}\\
Stop showing the turtle; does not affect the pen status\\

\texttt{void moveTo(int x, int y)}\\
Method to move to turtle to the given x and y location\\

\texttt{void penDown()}\\
Method to set the pen down\\

\texttt{void penUp()}\\
Method to lift the pen up\\

\texttt{void show()}\\
Make the turtle visible; does not affect the pen status\\

\texttt{void turn(int degrees)}\\
Method to turn the turtle the passed degrees use negative to turn left and pos to turn right\\

\texttt{void turnLeft()}\\
Method to turn left 90 degrees\\

\texttt{void turnRight()}\\
Method to turn right 90 degrees\\

\texttt{void turnToFace(int x, int y)}\\
Method to turn towards the given x and y\\

\texttt{void turnToFace(SimpleTurtle turtle)}\\
Method this turtle object to face the parameter turtle

\section*{Pictures}
\subsubsection*{Partial Constructor Summary}
\texttt{Picture()}\\
Constructor that takes no arguments\\

\texttt{Picture(int width, int height)}\\
Constructor that takes the width and height\\ 

\texttt{Picture(java.lang.String fileName)}\\
Constructor that takes a file name and creates the picture

\subsubsection*{Partial Method Summary}
\texttt{int getHeight()}\\
Method to get the height of the picture in pixels\\

\texttt{Pixel getPixel(int x, int y)}\\
Method to get a pixel object for the given x and y location\\

\texttt{Pixel[] getPixels()}\\
Method to get a one-dimensional array of Pixels for this simple picture\\

\texttt{int getWidth()}\\
Method to get the width of the picture in pixels\\

\texttt{void repaint()}\\
Method to force the picture to redraw itself.\\

\texttt{void show()}\\
Method to show the picture in a picture frame


\section*{Pixel}
\texttt{java.awt.Color getColor()}\\
Method to get a color object that represents the color at this pixel\\

\texttt{int getRed()}\\
\texttt{int getBlue()}\\
\texttt{int getGreen()}\\
Method to get the amount of red, blue, or green (respectively) at this pixel. The value will be between 0 (indicating no red/blue/green) and 255 (indicating maximum amount of red/blue/green).\\

\texttt{int getX()}\\
Method to get the x location of this pixel.\\

\texttt{int getY()}\\
Method to get the y location of this pixel.\\

\texttt{void setColor(java.awt.Color newColor)}\\
Method to set the pixel color to the passed in color objec\\

\texttt{void setRed(int value)}\\
\texttt{void setBlue(int value)}\\
\texttt{void setGreen(int value)}\\
Methods to set the red, blue, or green (respectively) to a new value in [0, 255].



\section*{SimpleInput}
\texttt{String getString}\\
Prompts the user to input a string.\\

\texttt{double getNumber}\\
Prompts the user to input a number.

\balance
\end{multicols}


\section*{The Student Class}
\begin{lstlisting}
public class Student {
    //Attributes
    private String name;
    private int studentID;
    
    //Constructors
    public Student(String theName, int theID) {
        this.name = theName;
        this.studentID = theID;
    }
    
    //Methods
    //Returns the student ID of a student
    public int getID() {
        return this.studentID;
    }
    
    //Tests if two Student objects are the same
    public boolean equals(Student otherStudent) {
        if (otherStudent == null) {
            return false;
        } else if (studentID == otherStudent.getID() ) {
            //student ID's are the same so:
            return true;
        } else {
            //student ID's are different so:
            return false;
        }
    }
}
\end{lstlisting}

\end{document}
