\section*{Polynomial Functors, etc.}
 
Let \CCC and \DDD be categories and let \tm{A} be an object of \DD\@.
A \definition{constant functor} \tm{\tmb{K}_{A}} is a functor
\tm{\tmb{K}_{A}: \CCC \rightarrow \DD} that maps every object of \CCC to the
object \tm{A} and every arrow of \CCC to \tm{id_{A}}.

\vs

We may write \tmb{1} for \tm{\tmb{K}_{A}} in the special case where \tm{A}
is a final object and we really don't care which final object \tm{A} is.

\vs

You should confirm that a constant functor is indeed a functor.

\newslide

An \definition{endofunctor} is a functor from some category \CCC
to the same category, \CC.

\vs

The \definition{identity functor}, \tmb{Id}, is an endfunctor 
that maps every object to itself and every arrow to itself.  The
identity functor for a
category is just the identity arrow for that category viewed as an
object in the category
\tmi{Cat}.

\newslide

Assume that all binary products exist in a category \CC.  That is,
for any objects \tm{A} and \tm{B}, \tm{A \times B} exists.

\vs

Let \tm{F} and \tm{G} be any pair of endofunctors defined on \CC.
We define the endofunctor \tm{(F \times G)} by

\begin{codenott}
\tm{(F \times G) \, A = F \, A \times G \, A}\\
\tm{(F \times G) \, h = \, <F \, h \circ \pi_{1}, G \, h \circ \pi_{2}>}
\end{codenott}

\vs

Notice that if \tm{h: A \rightarrow B} then
\tm{(F \times G) \, h : (F A \times G A) \rightarrow (F B \times G B)}.

\newslide

The definition of the product of functors
generalizes in a straightforward manner to products of any finite degree,
but we will restrict our attention to binary products in order to simplify the presentation.

Similarly, we will consider only binary coproducts (sums) of functors.

\newslide

Assume that all binary coproducts (sums) exist in a category \CC.  That is,
for any objects \tm{A} and \tm{B}, \tm{A + B} exists.

\vs

Let \tm{F} and \tm{G} be any pair of endofunctors defined on \CC.
We define the endofunctor \tm{(F + G)} by

\begin{codenott}
\tm{(F + G) \, A = F \, A + G \, A}\\
\tm{(F + G) \, h = \, [i_{1} \circ F \, h, i_{2} \circ G \, h]}
\end{codenott}

\vs

Notice that if \tm{h: A \rightarrow B} then
\tm{(F + G) \, h : (F A + G A) \rightarrow (F B + G B)}.

\newslide

Let's see how these ideas carries over to \Ha \, and \textit{Haskell}.

Recall that the objects of \Ha \, are pointed cpos and each
object correspond to
a particular ground type (i.e. a type with no
type variables) in \textit{Haskell}.  The
arrows in \Ha \, are continuous functions corresponding to
\textit{Haskell}
functions that are not polymorphic or overloaded.

Let's ignore the \bottom ~ element that is included in every object
in \Ha, and further ignore the fact that every \Ha \, object has at least one
value other than \bottom, so \Ha \, has no initial or final objects.

\newslide

To save room, we will use the symbol $\bowtie$ to mean ``corresponds to''.

With \Ha \, objects and \textit{Haskell} types,\\
\begin{codenott}
\tm{<f,g>} ~ \= 
\bla{$\bowtie$} ~ \= \tc{either f g} ~
\= \bla{(See page 42 of Bird.)}\kill
\tm{A \times B} \> \bla{$\bowtie$} \> \tc{(A, B)}\\
\tm{A + B} \> \bla{$\bowtie$} \> \tc{Either A B}\\
\end{codenott}

With \Ha \, arrows and \textit{Haskell} functions,
\begin{codenott}
\tm{(F \times G) A} ~ \= 
\bla{$\bowtie$} ~ \= \tc{either f g} ~
\= \bla{(See page 42 of Bird.)}\kill
\tm{<f,g>} \>
\bla{$\bowtie$} \> \tc{pair(f,g)} \> \bla{(See page 42 of Bird.)}\\
\tm{[f,g]} \> \bla{$\bowtie$} \> \tc{either f g} \>
\bla{which is the same as}\\
\> \>\tc{case(f,g)} \> \bla{(See page 46 of Bird.)}
\end{codenott}

\newslide

With \Ha \, functor operators and \textit{Haskell} functions,
the arrow part of the functor gives\\
\begin{codenott}
\tm{(F \times G) A} ~ \= 
\bla{$\bowtie$} ~ \= \tc{cross(fmap\tm{^{F}},fmap\tm{^{G}})} ~
\= \bla{(See page 42 of Bird.)}\kill
\tm{(F \times G)} \> \bla{$\bowtie$} \>
\tc{cross(fmap\tm{^{F}},fmap\tm{^{G}})} \> \bla{(See page 42 of Bird.)}\\
\tm{(F + G)} \> \bla{$\bowtie$} \>
\tc{plus(fmap\tm{^{F}},fmap\tm{^{G}})} \> \bla{(See page 46 of Bird.)}
\end{codenott}\\
while for the object part we have\\
\begin{codenott}
\tm{(F \times G) A} ~ \= 
\bla{$\bowtie$} ~ \= \tc{either f g} ~
\= \bla{(See page 42 of Bird.)}\kill
\tm{(F \times G) A} \> \bla{$\bowtie$} \>
\tc{(F A, G A)}\\
\tm{(F + G) A} \> \bla{$\bowtie$} \>
\tc{Either (F A) (G A)}
\end{codenott}

Recall Bird's comment, in \P 2 of page 47, that 
``\textcolor{QuoteColor}{The algebraic properties of} \tc{case}
\textcolor{QuoteColor}{and} \tc{plus}
\textcolor{QuoteColor}{are dual to those of} \tc{pair}
\textcolor{QuoteColor}{and} \tc{cross}\textcolor{QuoteColor}{.}''

\newslide

Let's consider a full \textit{Haskell} datatype declaration.
\begin{code}
data Perhaps a = Nope | Only a | Both Int a
\end{code}

We have\\
\begin{codenott}
\tc{Perhaps} $\bowtie$
\tm{(\tmb{1} + \Id +(\tmb{K}_{\tc{Int}} \times \Id))_{\bot}}
\end{codenott}\\
and could make \tc{Perhaps} an instance of \tc{Functor} by
\begin{code}
instance Functor Perhaps where\\
   fmap f Nope = Nope\\
   fmap f (Only a) = Only (f a)\\
   fmap f (Both n a) = Both n (f a)
\end{code}

\newslide

Notice:
\begin{enumerate}
\item We used ``\tc{|}'' directly rather than \tc{Either}, in part because we
have a sum of degree three instead of degree two, and in part because
\tc{Either}
is just a way to encapsulate ``\tc{|}'' and supply data constructors (tags)
\tc{Left} and \tc{Right}.
\item In \textit{Haskell}, \tc{Nope}, \tc{Only} and \tc{Both} are names of
constructors (instead of \tm{i_1} and \tm{i_1}, or \tc{Left} and \tc{Right})
as well as tags used in pattern matching.  These names have no role in the
structure of the functor.
\item We have added the \bottom, whose existence we now acknowledge.  We will
add a \bottom \, to the functor corresponding to the right side of each
\textit{Haskell} \tc{data} statement. 
\end{enumerate}

\newslide

A \definition{polynomial functor} is any functor formed using constant functors,
\Id \, and \tm{+} and \tm{\times}.

\vs

In \Has we need to be able to lift functors, using \bottom.  This is done
in the obvious way and we will skip the details.

\vs

\Has also allows function types to be used in datatype declarations.  While
category theory supports this, we will skip the details, since lifted
polynomial functors are sufficient for illustrating the ideas that we will
be exploring.