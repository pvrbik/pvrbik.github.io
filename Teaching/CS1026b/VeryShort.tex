% !TEX root = OurExam.tex
\section*{\S 1 Very Short Answers}

\emph{The Java code in the following questions is valid and will provide output (without error) if entered into 
the DrJava interactions pane.} 

\subsection*{Question 1}
Each answer in this Question is worth \emph{1 point} for a total of \emph{10 points}.

\renewcommand{\labelenumi}{\alph{enumi}.}
\begin{enumerate}
\item After the following code segment is executed
\begin{verbatim}
int x = 10 + 2*5 % 10 
\end{verbatim}
the value of x is: \Ans

\item The number of ``*''s printed by
\begin{verbatim}
for (int star = 9; star < 0; star++) {
    System.out.println("*");
}
\end{verbatim}
is: \Ans

\item Suppose we have defined
\begin{verbatim}
int a = 12, b =0;
boolean t = true;
\end{verbatim}
In order to make the expression \texttt{ (t ? !(a > b)) \&\& t } evaluate to \texttt{true}, we should replace \texttt{?} with:

\Ans

\item Suppose we have defined
\begin{verbatim}
int m=18, n=4;
\end{verbatim}
then the value of $\texttt{m / n + (m \% n)}$ is:\Ans

\clearpage
\item After the following code segment is executed
\begin{verbatim}
int x = 3;
x = x * x;
x = x + x;
\end{verbatim}
the value of \texttt{x} is: \Ans

\item Suppose we have defined
\begin{verbatim}
String s1 = new String("Finals");
String s2 = new String("Finals");
\end{verbatim}
then the value of the expression \texttt{(s1==s2)} is: \Ans

\item The \emph{last} value of the array \texttt{int[] xs = \{1,2,3,4\}} is at position: \Ans

\item After the following code segment is exectued
\begin{verbatim}
int x;
for (x = 1; x < 5; x = x + x);
\end{verbatim}
the value of \texttt{x} is: \Ans

\item After the following code segment is execute
\begin{verbatim}
int num = 15;
if (num >= 20)
    num = num - 20;
if (num >= 10)
    num = num -10;
if (num >= 5)
    num = num - 5;
\end{verbatim}
the value of \texttt{num} is: \Ans

\item After the following code segment is executed
\begin{verbatim}
System.out.println("3"+"7");
\end{verbatim}
what is printed? \Ans
\end{enumerate}


\subsection*{Question 2}
Each answer in this Question is worth \emph{2 points} for a total of \emph{10 points}.
\begin{enumerate}
\item For $32$ to be printed by the following code segment
\begin{verbatim}
int total = x;
for (int i = 0; i <= 10 ; i = i + 2)
{
  total += i;
}
System.out.print(total);
\end{verbatim}
the value of \texttt{x} should be: \Ans

\item The shape drawn by the following \texttt{Turtle} class method
\begin{verbatim}
void SecretShape () {
    for (int i =0; i < 13; i++) {
        this.forward(30);
        this.turnLeft();
    }
}
\end{verbatim}
is a: \Ans

\item After the following code segment is run
\begin{verbatim}
int z = 0;
double x = 123./321, y = 0.38317757009345793;
if ( x <= y && x > y ) {
    z = 1;
}
\end{verbatim}
the value of \texttt{z} is: \Ans

\clearpage
\item After the following code segment is run
\begin{verbatim}
int sum = 0; 
for (int i = 0; i < 10; i ++) {
    sum += i ; 
    i ++;
}
\end{verbatim}
the value of \texttt{sum} is: \Ans

\item After the following code segment is run
\begin{verbatim}
int x = 2*2*2*3*3*3*7;
while ( (x % 2 == 0) || (x % 3 == 0 ) ) {
    if (x % 2 == 0) {
        x = x/2;
    } else {
        x = x/3;
    }
}
\end{verbatim}
the value of \texttt{x} is: \Ans

\end{enumerate}
\clearpage