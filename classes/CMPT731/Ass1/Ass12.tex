
\documentclass[11pt]{article}
\usepackage{geometry} 
\geometry{letterpaper} 
\usepackage[parfill]{parskip}    % Activate to begin paragraphs with an empty line
\usepackage{graphicx}
\usepackage{amssymb}
\usepackage{epstopdf}
\DeclareGraphicsRule{.tif}{png}{.png}{`convert #1 `dirname #1`/`basename #1 .tif`.png}

\title{Brief Article}
\author{Richard Koch}

\begin{document}
\maketitle

\section{Introduction}
TeXShop is a front end for \TeX. You can use it to edit your source, typeset, and preview
the resulting pdf file.

However, TeXShop by itself cannot typeset. You need to install TeXLive/teTeX, a vast
collection of command line programs, fonts, style files, and so forth. TeXShop's job is to hide the complexity of
this \TeX\ distribution. To obtain TeXLive/teTeX, go to 
\begin{verbatim}
          http:www.uoregon.edu/~koch/texshop
\end{verbatim}
download ii2, and follow the directions under the tab
``Installing teTeX/TeXLive.''

You can use this example to experiment with TexShop
before you install TeXLive/teTeX. Since the typeset pdf file is included,
all of the TexShop windows appear. But you cannot change the source and
typeset again until you have installed TeXLive/teTeX. 

\section{Style Changes}
{\em Here is a sentence in italics.} 
{\bf Here is a sentence in bold face.} 

\section{Lists}
You need to understand two things:
\begin{itemize}
\item this
\item and this
\end{itemize}  

\section{Mathematics}
\TeX\ understands mathematics. It can typeset
Greek letters: $\alpha, \beta, \gamma.$ It can typeset formulas: $\sqrt{x^2 + 1}$ and 
$\sqrt{ {{x + 1} \over {x + 2}} }.$ It can display formulas:
$$\int_0^\infty e^{- x^2} \ dx = { \sqrt{\pi} \over 2}$$
It can display matrices.
$$A = \left( \begin{array}{ccc} 
	1  &  \sqrt{2} & 3 \\
	\sqrt{x + 1} & {{x + 1} \over {x - 1}} & \sin x \\
	0 & 1 & 5
\end{array} \right)$$

\section{Graphics}

TeX understands graphic commands.

\begin{figure}[htbp]
   \centering
   \includegraphics[width=2in]{saddle} 
   \caption{A Saddle}
\end{figure}

\section{License:}
TeXShop is provided under the GNU General Public License. This means that you are
free to use, copy, and modify the program. If you give your modifications to others,
you must also provide the modified source code under the same GPL.

The source code is available on my web site.



 \end{document}

