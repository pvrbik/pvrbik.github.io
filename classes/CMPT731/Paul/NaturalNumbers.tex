\documentclass{seminar}
\nonstopmode

% \usepackage{geometry}
% \geometry{letterpaper,landscape}
% \usepackage{pictex}
%  pictex causes me to run out of dimensions (no room for new #3)
\usepackage{pictex}
\usepackage{dcpic}
\usepackage{semhelv}
\usepackage{color}
\usepackage{alltt}
\usepackage{latexsym}
\usepackage{amsfonts}
\usepackage{amsmath}
\usepackage{calc}
\usepackage{graphicx}
\usepackage{pifont}

\definecolor{CodeColor}{named}{BlueViolet}
% \definecolor{SystemColor}{named}{Mulberry}
% \definecolor{SystemColor}{named}{OliveGreen}
\definecolor{SystemColor}{named}{RedOrange}
\definecolor{MathColor}{named}{Magenta}
\definecolor{QuoteColor}{named}{Salmon}
\definecolor{LawColor}{named}{Green}
\definecolor{DefinitionColor}{named}{Red}
\definecolor{AlgorithmColor}{named}{RawSienna}
\definecolor{BackgroundColor}{named}{Yellow}
\definecolor{OtherColor}{named}{PineGreen} % {Maroon}
\definecolor{Green}{named}{OliveGreen}
% \definecolor{BackgroundColor}{cmyk}{0,0,0.6,0}

\newcommand{\semester}{Fall 2006}

\newcommand{\bs}{\ensuremath{\backslash}}
\newcommand{\zero}{\textcolor{MathColor}{\ensuremath{\mathbf{0}}~}}
\newcommand{\one}{\textcolor{MathColor}{\ensuremath{\mathbf{1}}~}}

\newcommand{\law}[1]{\textcolor{black}{(\textcolor{LawColor}{#1})}}

\newcommand{\sublaw}[1]{\textcolor{black}{(\textcolor{black}{#1})}}

\newcommand{\reason}[1]{\hspace{0.15in}
    \meq\hspace{0.075in}\law{#1}}

\newcommand{\subreason}[1]{\hspace{0.15in}
    \meq\hspace{0.075in}\sublaw{#1}}

\newcommand{\definition}[1]{\hspace{0.05in}
    \textcolor{DefinitionColor}{\textit{#1}\hspace{0.05in}}}

\newcommand{\meq}{\textcolor{MathColor}{\ensuremath{\equiv}}}
\newcommand{\bottom}{\textcolor{MathColor}{\ensuremath{\bot}}}
\newcommand{\dom}{\textcolor{MathColor}{\ensuremath{\mathit{dom~}}}}
\newcommand{\cod}{\textcolor{MathColor}{\ensuremath{\mathit{cod~}}}}
\newcommand{\rarrow}{\textcolor{MathColor}{\ensuremath{\rightarrow}}}

\newcommand{\textcd}[2][CodeColor]{\textcolor{#1}{\texttt{#2}}}
\newcommand{\tc}[2][CodeColor]{\textcolor{#1}{\texttt{#2}}}
\newcommand{\tcx}[2][OtherColor]{\textcolor{#1}{\texttt{#2}}}
\newcommand{\ts}[2][SystemColor]{\textcolor{#1}{\texttt{#2}}}
\newcommand{\textcdl}[2][CodeColor]{\textcolor{#1}{\hspace{0.1in}\texttt{#2}}}
\newcommand{\textcdr}[2][CodeColor]{\textcolor{#1}{\texttt{#2}\hspace{0.1in}}}
\newcommand{\textcdlr}[2][CodeColor]{\textcolor{#1}
{\hspace{0.1in}\texttt{#2}\hspace{0.1in}}}

\newcommand{\tmm}[1]{\textcolor{AlgorithmColor}{\ensuremath{#1}}}


\newcommand{\textmath}[1]{\textcolor{MathColor}{\ensuremath{#1}}}
\newcommand{\tm}[1]{\textcolor{MathColor}{\ensuremath{#1}}}
\newcommand{\tmcc}[1]{\textcolor{CodeColor}{\ensuremath{#1}}}
\newcommand{\tmtt}[1]{\textcolor{MathColor}{\texttt{#1}}}
\newcommand{\unmath}[1]{\textcolor{black}{\mathrm{#1}}}
\newcommand{\tmi}[1]{\textcolor{MathColor}{\ensuremath{\mathit{#1}}}}
\newcommand{\tmr}[1]{\textcolor{MathColor}{\ensuremath{\mathrm{#1}}}}
\newcommand{\tmb}[1]{\textcolor{MathColor}{\ensuremath{\mathbf{#1}}}}


\newcommand{\tbr}[1]{\textcolor{red}{\textbf{#1}}}

\newcommand{\bla}[1]{\textcolor{black}{#1}}
\newcommand{\gre}[1]{\textcolor{Green}{#1}}
\newcommand{\red}[1]{\textcolor{red}{#1}}
\newcommand{\blu}[1]{\textcolor{blue}{#1}}
\newtheorem{lemma}{Lemma}
\newtheorem{theorem}{Theorem}
\newtheorem{corollary}{Corollary}
\newenvironment{proof}{\textbf{Proof:}}{\hfill $\Box$}

\slideframe{none}

\centerslidesfalse

\newlength{\tempwidth}

\newenvironment{cdiag}
   {\vspace{0.25in}
    \begin{center}
    \setlength{\tempwidth}{\linewidth}
    \addtolength{\tempwidth}{-2\columnsep}
    \begin{minipage}[t]{\tempwidth}
    \begindc{\commdiag}[10]
   }
   {\enddc
    \end{minipage}
    \end{center}
    \vspace{0.25in}
   }

\newenvironment{cdiag2}
   {\vspace{0.25in}
    % \begin{center}
    \setlength{\tempwidth}{\linewidth/2}
    \begin{minipage}[t]{\tempwidth}
    \begindc{\commdiag}[10]
   }
   {\enddc
    \end{minipage}
    % \end{center}
    \vspace{0.25in}
   }

\newenvironment{udg}
   {\vspace{0.25in}
    \begin{center}
    \setlength{\tempwidth}{\linewidth}
    \addtolength{\tempwidth}{-2\columnsep}
    \begin{minipage}[t]{\tempwidth}
    \begindc{2}[10]
   }
   {\enddc
    \end{minipage}
    \end{center}
    \vspace{0.25in}
   }

\newenvironment{codenott}[1][CodeColor]
   {\color{#1}
    % \begin{center}
    \setlength{\tempwidth}{\linewidth}
    \addtolength{\tempwidth}{-2\columnsep}
    \hspace*{0.25in}
    \begin{minipage}[t]{\tempwidth}
    \begin{tabbing}
    \hspace{0.075in} \= \hspace{0.075in} \= \hspace{0.075in} \= \hspace{0.075in} \= \hspace{0.075in} \= \kill
   }
   {\end{tabbing}
    \end{minipage}
    % \end{center}
    \color{black}
   }

\newenvironment{codenott2}[1][CodeColor]
   {\color{#1}
    % \begin{center}
    \setlength{\tempwidth}{\linewidth/3}
    \hspace*{0.25in}
    \begin{minipage}[b]{\tempwidth}
    \begin{tabbing}
    \hspace{0.075in} \= \hspace{0.075in} \= \hspace{0.075in} \= \hspace{0.075in} \= \hspace{0.075in} \= \kill
   }
   {\end{tabbing}
    \end{minipage}
    % \end{center}
    \color{black}
   }

\newenvironment{laws}
   {\setlength{\tempwidth}{\linewidth}
    \addtolength{\tempwidth}{-2\columnsep}
    \hspace*{0.25in}
    \begin{minipage}[t]{\tempwidth}
    \begin{tabular}{lr}
   }
   {\end{tabular}
    \end{minipage}
   }

\newenvironment{code}[1][CodeColor]
   {\color{#1}
    % \begin{center}
    \setlength{\tempwidth}{\linewidth}
    \addtolength{\tempwidth}{-2\columnsep}
    \hspace*{0.25in}
    \begin{minipage}[t]{\tempwidth}
    \begin{alltt}
    \begin{tabbing}
    \hspace{0.075in} \= \hspace{0.075in} \= \hspace{0.075in} \= \hspace{0.075in} \= \hspace{0.075in} \= \kill
   }
   {\end{tabbing}
    \end{alltt}
    \end{minipage}
    % \end{center}
    \color{black}
   }

\newenvironment{mcode}[1][MathColor]
   {\color{#1}
    % \begin{center}
    \setlength{\tempwidth}{\linewidth}
    \addtolength{\tempwidth}{-2\columnsep}
    \hspace*{0.25in}
    \begin{minipage}[t]{\tempwidth}
    \[\begin{array}{lrll}}
   {\end{array}\]
    \end{minipage}
    % \end{center}
    \color{black}
   }

\newcommand{\mF}{\textcolor{MathColor}{\ensuremath{\tc{fmap}_{F}~}}}
\newcommand{\Id}{\textcolor{MathColor}{\ensuremath{\mathit{Id}}}}
\newcommand{\nt}{\textcolor{MathColor}{\ensuremath{\stackrel{.}{\rightarrow}}}}
\newcommand{\List}{\textcolor{MathColor}{\ensuremath{\tc{[}?\tc{]}}}}
\newcommand{\LL}{\textcolor{MathColor}{\ensuremath{\tc{[[}?\tc{]]}}}}
\newcommand{\LA}{\textcolor{MathColor}{\ensuremath{\tc{[}A\tc{]}}}}
\newcommand{\LB}{\textcolor{MathColor}{\ensuremath{\tc{[}B\tc{]}}}}
\newcommand{\LLA}{\textcolor{MathColor}{\ensuremath{\tc{[[}A\tc{]]}}}}
\newcommand{\LLB}{\textcolor{MathColor}{\ensuremath{\tc{[[}B\tc{]]}}}}
\newcommand{\lam}{\textcolor{CodeColor}{\ensuremath{\lambda}}}
\newcommand{\redu}{\textcolor{MathColor}{\ensuremath{\longrightarrow}}}
\newcommand{\Has}{\textit{Haskell}\,}

\newcommand{\ack}{I would like to thank Tai Meng for many corrections
and suggestions for improvements to these notes.  Of course, the remaining
errors are my fault.}


\newcommand{\NAT}{\textcolor{MathColor}{\ensuremath{\mathbb{N}}}}
\newcommand{\ZZ}{\textcolor{MathColor}{\ensuremath{\mathbb{Z}}}}
\newcommand{\ZP}{\textcolor{MathColor}{\ensuremath{\mathbb{Z}^{+}}}}
\newcommand{\QQ}{\textcolor{MathColor}{\ensuremath{\mathbb{Q}}}}
\newcommand{\QP}{\textcolor{MathColor}{\ensuremath{\mathbb{Q}^{+}}}}
\newcommand{\suc}{\tmi{succ}}
\newcommand{\Ha}{\tmi{Hask}}
\newcommand{\vs}{\vspace{0.3in}}
\newcommand{\CC}{\tm{\cal C}}
\newcommand{\CCC}{\tm{{\cal C}\,}}
\newcommand{\COP}{\tm{\cal C^\mathrm{op}}}
\newcommand{\DD}{\tm{\cal D}}
\newcommand{\DDD}{\tm{{\cal D}\,}}

\newcommand{\pb}{\textcolor{MathColor}{\ensuremath{+\!\!_{_{_{\bot}}}}}}

\newpagestyle{lectures}{\makebox[1in][l]{CMPT 481/731} \hfill Functional Programming
\hfill \makebox[1in][r]{\semester}}{\makebox[2in][l]{Natural Numbers}
\hfill F. Warren
Burton \hfill \makebox[2in][r]{\thepage}}

\pagestyle{lectures}

\begin{document}
\begin{slide}

\begin{center}
\begin{LARGE}
\textbf{Natural Numbers}
\end{LARGE}\\[3ex]
\end{center}

We will use natural numbers to illustrate several ideas that
will apply to \Has data types in general.

For the moment we will ignore that fact that each type in
\Has includes \bottom ~ as a possible value.

We note that it is only possible to define the natural numbers up to
isomorphism.

\newslide

\section{Peano's Axioms}

Let's consider a definition of \textit{Natural Numbers},
\NAT, based on Peano's axioms. 

\begin{enumerate}
\item \tm{0 \in \NAT}
\item \tm{m \in \NAT \Rightarrow \tmi{succ}(m) \in \NAT}.
\end{enumerate}

\newslide

Oops, we forgot to exclude negative numbers!

\begin{enumerate}
\item \tm{0 \in \NAT}
\item \tm{m \in \NAT \Rightarrow \suc(m) \in \NAT}.
\item \tm{m \in \NAT \Rightarrow \suc(m) \not= 0}
\end{enumerate}

\newslide

We also need one more axiom.

\begin{enumerate}
\item \tm{0 \in \NAT}
\item \tm{m \in \NAT \Rightarrow \suc(m) \in \NAT}.
\item \tm{m \in \NAT \Rightarrow \suc(m) \not= 0}
\item \tm{\suc(m) = \suc(n) \Rightarrow m = n}
\end{enumerate}

\newslide

Oh my gosh!  I just noticed that the nonnegative real numbers, the rational excluding
negative integers,
and lots of other sets, all satisfy these axioms.  I forgot the induction
axiom!

\begin{enumerate}
\item \tm{0 \in \NAT}
\item \tm{m \in \NAT \Rightarrow \suc(m) \in \NAT}.
\item \tm{m \in \NAT \Rightarrow \suc(m) \not= 0}
\item \tm{\suc(m) = \suc(n) \Rightarrow m = n}
\item For any predicate \tm{P} in the first order predicate calculus,
\tm{(P(0) \wedge (P(m) \Rightarrow P(\suc(m)))) \Rightarrow
(\forall n \in \NAT.P(n))}
\end{enumerate}

The induction axiom insures that \NAT \, is the smallest set that satisfies
the above axioms.

\newslide

While you were looking at the previous page, I remembered that there are
some completeness issues with the axioms for number theory.  The axioms
on the previous page are satisfied by an infinite number of sets, most of
which are a bit weird.

We have to allow \tm{P} to be any predicate.

\begin{enumerate}
\item \tm{0 \in \NAT}
\item \tm{m \in \NAT \Rightarrow \suc(m) \in \NAT}.
\item \tm{m \in \NAT \Rightarrow \suc(m) \not= 0}
\item \tm{\suc(m) = \suc(n) \Rightarrow m = n}
\item For any predicate \tm{P},
\tm{(P(0) \wedge (P(m) \Rightarrow P(\suc(m)))) \Rightarrow
(\forall n \in \NAT.P(n))}
\end{enumerate}

\newslide

Since we have no way of expressing more than a countable subset of the
predicates on \NAT, perhaps it is better to restate the final axiom in terms
of sets.

\begin{enumerate}
\item \tm{0 \in \NAT}
\item \tm{m \in \NAT \Rightarrow \suc(m) \in \NAT}.
\item \tm{m \in \NAT \Rightarrow \suc(m) \not= 0}
\item \tm{\suc(m) = \suc(n) \Rightarrow m = n}
\item \tm{(X \subseteq \NAT \wedge 0 \in X
\wedge (m \in X \Rightarrow \suc(m) \in X))
\Rightarrow X = \NAT}
\end{enumerate}

\newslide

Are you really sure we finally have a correct definition
for the natural numbers?

Are you even sure I finally managed to get my parentheses
balanced correctly?

\newslide

\section{Natural Numbers defined using Category Theory}

Let's try another approach.

Consider a category, \tmi{US} (for unary system), where

\begin{description}
\item[An Object] is any triple \tm{(S,a,f)} such that
\begin{itemize}
\item \tm{S} is a set,
\item \tm{a \in s}, and
\item \tm{f: S \rightarrow S} is a function.
\end{itemize}
\item[An Arrow] is any homomorphism between objects. 
\end{description}

The desired unary system,
\tm{(\NAT, 0, \suc)},
is an initial object in this category.

(An initial object is an object such that for any object, \tm{A}, there
is exactly one arrow from the initial object to \tm{A}.  In particular,
there is exactly one arrow from the initial object to itself, which in this case is the identity function.)

\newslide

We will define category, object, arrow and initial object later.
Initial objects are unique up to isomorphism, when they exist,
and in cases like this initial objects always exist.

For now, all we need to understand about this definition is that
the natural numbers is a unary system such that for another
unary system there is exactly one
homomorphism from the natural numbers to the other unary system.

If \tm{(S,a,f)} and \tm{(T,b,g)} are any two unary systems, then a homomorphism,
\tm{h}, is a structure preserving function from \tm{S} to \tm{T}.  That is
\begin{codenott}
\tm{h: S \rarrow T}\\
\tm{h(a) = b}\\
\tm{\forall s \in S . ~ h(f(s)) = g(h(s))}
\end{codenott}

Later we will give a more general definition of homomorphism that will
apply to a variety of different structures.

\newslide

\label{p-not-induction}

To see that \tm{(\NAT, 0, \suc)} is an initial object, it is sufficient
to show that for any other object, \tm{(S,a,f)}, in \tmi{US}
there is a unique homomorphism from \tm{(\NAT, 0, \suc)} to
\tm{(S,a,f)}.

Clearly, \tm{h} defined by

\begin{codenott}

\tm{h(0) = a}\\
\tm{\forall n \in \NAT.h(\suc(n)) = f(h(n))}
\end{codenott}

is the unique homomorphism.

This is clearly the case because of our intuitive understanding of the
natural numbers.  We are actually using the requirement that the
above holds to define the natural numbers.

\newslide

Quiz:  \color{QuoteColor}
For each of the following unary systems,
(i) find the unique homomorphism from \tm{(\NAT, 0, \suc)} to
the system, (ii) determine whether there are zero, one or more than one homomorphisms
from the unary system to \tm{(\NAT, 0, \suc)},
(iii) find either two or all homomorphisms from the unary system to
\tm{(\NAT, 0, \suc)}, and in the case of a unique homomorphism to
\tm{(\NAT, 0, \suc)~}determine if the homomorphism is an isomorphism.

\begin{enumerate}
\item \tm{(\ZZ, 0, succ)}
\item \tm{(\ZP, 1, succ)}
\item \tm{(}Strings over \tm{\{}\tmtt{'}\tm{a}\tmtt{'}\tm{\},}
\tmtt{'}\tm{a}\tmtt{'}\tm{, (+}\tmtt{'}\tm{a}\tmtt{'}\tm{))} where \tm{+} means concatenate.
% \item \tm{(\QQ, 0, succ)}
\item \tm{(\QP, 0, succ)}, where \tm{\mathit{succ} ~ x = x + 1}
\item \tm{(\{True, False\}, False, \neg)}
\item \tm{(\{0,1,2,3,4\}, 0, succ_{\!\pmod{5}})}
\item \tm{(\{0,1,2,3,4\}, 3, succ_{\!\pmod{5}})}
\end{enumerate}
\color{black}\rule{1ex}{0ex}

\newslide

Quiz:  \textcolor{QuoteColor}{
\begin{enumerate}
\item Prove that if \tm{h_{1}: S_{1} \rarrow S_{2}} is a homomorphism from
\tm{(S_{1},a_{1},f_{1})} to \tm{(S_{2},a_{2},f_{2})} and
\tm{h_{2}: S_{2} \rarrow S_{3}} is a homomorphism from
\tm{(S_{2},a_{2},f_{2})} to \tm{(S_{3},a_{3},f_{3})} then
\tm{h_{2} \circ h_{1}} is also a homomorphism.
\item Prove that if for each unary system
there is a unique homomorphism from
\tm{(S,a,f)} to that unary system, then \tm{(S,a,f)} is isomorphic to
\tm{(\NAT,0,succ)}.
\end{enumerate}}

Note: \tm{(S,a,f)} and \tm{(\NAT,0,succ)} are isomorphic unary systems if
there exist homomorphisms \tm{g: S \rarrow \NAT} and
\tm{h: \NAT \rarrow S} such that \tm{g \circ h = id_{\NAT}}
and \tm{h \circ g = id_{S}}.

\newslide

\section{Haskell \tc{Nat}s}

With \textit{Bird's} definition of \tc{Nat}s in Haskell (ignoring the
\tc{deriving} stuff),

\begin{code}
data Nat = Zero | Succ Nat
\end{code}

we have an additional constructor, \bottom.

Everything we have done so far carries over with minimal change.  Instead of
a unary system we have \tc{(Nat, Zero, \bottom, Succ)}.

A homomorphism
from \tm{(\tc{Nat}, \tc{Zero}, \bottom_{\tc{Nat}}, \tc{Succ})}
to \tm{(S, a, \bottom_{S}, f)} is a function \tm{h: \tc{Nat} \rarrow S}
such that

\begin{codenott}
\tm{h ~ \tc{Zero} = a}\\
\tm{h ~ \bottom_{\tc{Nat}} = \bottom_{S}}\\
\tm{\forall n \in \tc{Nat} . h ~ (\tc{Succ} ~ n) = f (h ~ n)}
\end{codenott}

\newpage

We will usually refer to \bottom~ without using a subscript, even though each
domain (set of values corresponding to a type --- more or less) has a
different \bottom.  Sometimes we will skip explicitly mentioning \bottom,
since the impact of adding \bottom ~ is straight forward.

The second condition on the previous page,
\tm{h ~ \bottom = \bottom}, will usually be stated as
``\tm{h} is strict''.

\newpage

\section{Catamorphisms}

The homomorphisms from \tc{Nat}, or more generally homomorphisms from
types defined by \Has \tc{data} statements, are called
\definition{catamorphisms}.

The catamorphisms for a type are exactly the functions that can be defined
using the \tm{\mathit{fold}} function for the type.

\newslide

The \tm{\mathit{fold}} functions for \tc{Nat}'s (given in \textit{Bird}) is

\begin{code}
foldn :: (a -> a) -> a -> Nat\\
foldn h c Zero = c\\
foldn h c (Succ n) = h (foldn h c n)
\end{code}

Recall that \tm{g} is a homomorphism
from \tm{(\tc{Nat}, \tc{Zero}, \bottom, \tc{Succ})}
to \tm{(S, a, \bottom, f)} iff

\begin{codenott}
\tm{g} ~ \color{black}{is strict}\\
\tm{g ~ \tc{Zero} = a}\\
\tm{\forall n \in \tc{Nat} . g ~ (\tc{Succ} ~ n) = f (g ~ n)}
\end{codenott}

in which case 

\begin{codenott}
\tm{g = \tc{foldn} ~ f ~ a}
\end{codenott}

The third condition above,
\tm{\forall n \in \tc{Nat} . g ~ (\tc{Succ} ~ n) = f (g ~ n)},
can be written more compactly as \tm{g \circ \tc{Succ} = f \circ g}.

\newslide

Another way to look at 

\begin{code}
foldn :: (a -> a) -> a -> Nat\\
foldn h c Zero = c\\
foldn h c (Succ n) = h (foldn h c n)
\end{code}

is that ~ \tc{foldn h c n} ~ constructs a new value by replacing the
\tc{Zero} constructor in \tc{n} with \tc{c} and each \tc{Succ}
constructor with \tc{h}.

For example

\begin{code}
foldn h c (Succ (Succ (Succ Zero))) = h (h (h c))
\end{code}

\newpage

\subsection{Fusion}

Suppose

\begin{itemize}
\item \tc{foldn g a} ~ is a homomorphism from \tc{(Nat,Zero,\bottom,Succ)}
to \tc{(\tm{S},a,\bottom,g)},
\item \tc{foldn h b} ~ is a homomorphism from \tc{(Nat,Zero,\bottom,Succ)}
to \tc{(\tm{T},b,\bottom,h)}, and
\item \tc{f} ~ is a homomorphism from \tc{(\tm{S},a,\bottom,g)}
to \tc{(\tm{T},b,\bottom,h)}.
\end{itemize}

Since the composition of two homomorphisms is a homomorphism,
and there is exactly one homomorphism from
\newline
\tc{(Nat,Zero,\bottom,Succ)}
to \tc{(\tm{T},b,\bottom,h)}, it must be the case that:

\begin{code}
f . foldn g a = foldn h b
\end{code}

\newpage

The result on the previous page is the \definition{fusion Law for \tc{Nat}}.

In the book, the condition that \tc{f} is a homomorphism is stated as

\begin{codenott}
\tc{f} \color{black}{is strict}\\
\tc{f a = b}\\
\tc{f.g = h.f}
\end{codenott}

It may help to note that

\begin{code}
Zero: Nat\\
Succ: Nat -> Nat\\[1.5ex]
a: \tm{S}\\
g: \tm{S} -> \tm{S}\\[1.5ex]
b: \tm{T}\\
h: \tm{T} -> \tm{T}\\[1.5ex]
f: \tm{S} -> \tm{T}\\
\end{code}

\newpage

\subsection{Induction}

In \textit{Bird}, the fusion law for \tc{Nat}s was proved using induction.

We will reverse this process and show, using the fusion law for \tc{Nat}s,
that proof by induction is valid on \NAT.

Before going into detail, let us consider why we can prove the validity
of proof by induction, while mathematicians regard induction as an axiom.

Recall that on
page \pageref{p-not-induction} we
\textbf{\textit{defined}} the natural numbers to
be the initial object of the category of unary systems.  It turns out that this
is equivalent to the induction axiom.

\newpage

We will break our proof into a number of small lemmas so that each lemma
and its proof can be written on a single page.

We will use the following definitions throughout the lemmas and
the theorem that follows.

\begin{code}
f1 n     = (n, True)\\
f2 n     = (n, p n)\\
g        = Succ\\
h (n, q) = (Succ n, q || p (Succ n))\\
a        = Zero\\
b1       = f1 a = (Zero, True)\\
b2       = f2 a = (Zero, p Zero)
\end{code}

While our work is done using the notation of \textit{Haskell}, we note that the
results will still hold even if \tc{p} is not a computable predicate.

\newpage

\paragraph{\red{Important:}}
In our consideration of natural numbers, we will restrict our attention
to \NAT.  In particular, we
will not allow numbers to be \bottom ~ or partial numbers, or
infinite.  This seems
reasonable, since induction is usually expressed in terms of \NAT ~ rather
than \tc{Nat}.

Furthermore, we will assume that \tc{p} is a total predicate (i.e.
\tc{(p n)} can never be \bottom).  This seems reasonable, since we are
not restricting \tc{p} to be a computable predicate.

%We could easily extends these results to \tc{Nat}s, by making \tc{f1} and
%\tc{f2} strict (which is not necessary for these to be homomorphisms on
%\NAT) and requiring that \tc{(p \bottom)} be true.

\newpage

%:lemma1

\begin{lemma} \label{lemma-a}
\tc{foldn Succ Zero = id}
\end{lemma}
\begin{proof}
The function ~ \tc{foldn Succ Zero} ~ is a homomorphism from \tc{\NAT} to
\tc{\NAT}.  The identity function is also a homomorphism from \tc{\NAT} to
\tc{\NAT}.  Since there can be only one such homomorphism, the result holds.
\end{proof}

Notice that we avoided using induction, since that would result in a
circular argument.

\newpage

%:lemma2

\begin{lemma} \label{lemma-b}
If ~ \tc{p  n \tmcc{\Rightarrow} p (Succ n)}
\newline
then ~ \tc{p (Succ n) = (p n || p  (Succ n))}
\end{lemma}
\begin{proof}
If ~ \tc{p n = True} ~ then
\begin{codenott}
\tc{p (Succ n)}\\
\reason{assumptions: ~ \tc{p n = True} ~ and ~ \tc{p n \tm{\Rightarrow} p (Succ n)}}\\
\tc{True}\\
\reason{algebra of logic}\\
\tc{True || p (Succ n)}\\
\reason{asumption: ~ \tc{p n = True}}\\
\tc{p n || p (Succ n)}
\end{codenott}

Otherwise (\tc{p n = False}), by a similar argument

\begin{codenott}
\tc{p (Succ n) = False || p (Succ n)}\\
\tc{  = p n || p (Succ n)}
\end{codenott}

\end{proof}

\newpage

If we allow \tc{~p n~} to be \bottom ~ then \textit{Lemma 2} will
no longer hold.

However, we can define a predicate \tm{q}, which is not computable, by

\color{MathColor}
\[q ~ n \equiv \left\{
\begin{array}{ll}
\tc{True}, & \tmr{if} ~ \tc{p n} \equiv \tc{True}\\
\tc{False}, & \tmr{if} ~ \tc{p n} \equiv \tc{False}\\
\tc{False}, & \tmr{if} ~ \tc{p n} \equiv \bottom
\end{array}
\right.\]
\color{black}

Now, if \tm{\forall n \in \NAT . q ~ n} then
\tm{\forall n \in \NAT . p ~ n}.

\newpage

%:lemma3

\begin{lemma} \label{lemma-e}
\tc{f1.g = h.f1}
\end{lemma}
\begin{proof} For any ~ \tc{n}\tm{\in\NAT}

\begin{code}
(f1.g) n\\
\reason{plug and grind}\\
(Succ n, True)\\
\reason{algebra of logic}\\
(Succ n, True || p (Succ n))\\
\reason{plug and grind}\\
(h.f1) n
\end{code}

so the result holds by extensionality.
\end{proof}

\newpage

%:lemma4

\begin{lemma}
% If ~ \tc{(p Zero) ~ then ~ 
\tc{f1 = foldn h (Zero, True)}
\end{lemma}
\begin{proof}
\begin{code}
f1\\
\reason{Lemma 1}\\
f1 . foldn Succ Zero\\
\reason{fusion; Lemma 3}\\
foldn h (f1 Zero)\\
\reason{plug and grind}\\
foldn h (Zero, True)
\end{code}

The fusion step uses:
\begin{code}
\tc{f1.foldn Succ Zero = foldn h (f1 Zero)}
\end{code}
\end{proof}


\newpage

%:lemma5

\begin{lemma} \label{lemma-f}
If ~ \tc{p  n \tm{\Rightarrow} p (Succ  n)} ~
then ~
\tc{f2.g = h.f2}
\end{lemma}
\begin{proof} For any ~ \tc{n}\tm{\in\NAT}

\begin{code}
(f2.g) n\\
\reason{plug and grind}\\
(Succ n, p (Succ n))\\
\reason{Lemma 2}\\
(Succ n, p n || p (Succ n))\\
\reason{plug and grind}\\
(h.f2) n\\
\end{code}

so the result holds by extensionality.
\end{proof}

\newpage

%:lemma6

\begin{lemma} \label{lemma-c}
If ~  \tc{(p Zero}) \tm{\wedge}
\tc{(p n \tm{\Rightarrow} p (Succ n))}
\newline
then ~ \tc{f2 = foldn h (Zero, True)}
%\tc{\tmcc{\backslash}n -> (n, p n) = foldn h (Zero, True)}
%\newline
%where \tc{h (n, q) = (Succ n, q || p (Succ n))}
\end{lemma}
\begin{proof}
\begin{code}
f2\\
\reason{Lemma 1}\\
f2 . foldn Succ Zero\\
\reason{fusion; Lemma 5}\\
foldn h (f2 Zero)\\
\reason{\tc{p Zero} is True}\\
foldn h (Zero, True)
\end{code}

The fusion step uses:
\begin{code}
\tc{f2.foldn Succ Zero = foldn h (f2 Zero)}
\end{code}
\end{proof}

\newpage

%In the fusion step we have:

%\begin{code}
%f = \tmcc{\backslash}n -> (n, True)\\
%g = Succ\\
%a = Zero\\
%h = h\\
%b = (Zero, True)\\
%\end{code}

%\newpage

\begin{theorem}[Induction] 
If ~  \tc{(p Zero}) \tm{\wedge}
\tc{(p n \tm{\Rightarrow} p (Succ n))}
\newline
then ~ \tm{\forall \tc{n} \tm{\in \NAT.}\tc{p n = True}}
\end{theorem}
\begin{proof}
On \NAT ~ (but not \tc{Nat}), by \textit{Lemma 4} and \textit{Lemma 6},

\begin{codenott}
\tc{f1 = f2}
\end{codenott}

so 

\begin{codenott}
\tc{snd.f1 = snd.f2}
\end{codenott}

Since

\begin{code}
(snd.fl) n = True\\
(snd.f2) n = p n
\end{code}

We have the desired result, that for ~
\tm{\forall}\tc{n} \tm{\in \NAT.}\tc{p n = True}.

\end{proof}

\newpage

We will now demonstrate the use of fusion to do the classic undergraduate
problem of showing
\[\textcolor{MathColor}{\sum_{i=0}^{n} i = \frac{n(n+1)}{2}}\]

We will use \tc{Integer}s to represent \NAT s, with the following fold function.

\begin{code}
foldi :: (a -> a) -> a -> Integer -> a\\
foldi g a 0 = a\\
foldi g a (n+1) = g (foldi g a n)
\end{code}

That is, we will use the concrete type \tc{Integer} to represent the
abstract type \NAT.  Negative \tc{Integer}s will not be used, and
\tc{foldi} is the concrete implemetation of an abstract
\tmtt{fold\NAT} function.  If we used \tc{Nat}s to represent \NAT s,
then we would need to ignore partial and infinite \tc{Nat}s, and use
\tc{foldn} to implement \tmtt{fold\NAT}.

\newpage

To recursively define

\begin{codenott}
\tm{h ~ n = (n,~ \sum_{i=0}^{n} i)}
\end{codenott}

we can define and use \tc{g}

\begin{code}
g :: (Integer,Integer) -> (Integer,Integer)\\
g (a, b) = (a+1, b + (a+1))
\end{code}

so

\begin{codenott}
\tc{h (n+1)} \= \tc{= g (h n) =}
\= \tm{= \tc{g} ~ (n,~ \sum_{i=0}^{n} i)} \= \kill
\tc{h 0} \> \tc{= (0, 0)} \> \tm{= (0,~ \sum_{i=0}^{0} i)}\\
\tc{h (n+1)} \> = \tc{g (h n)} \> \tm{= \tc{g} ~ (n,~ \sum_{i=0}^{n} i)}
\> \tm{= (n+1,~ \sum_{i=0}^{n+1} i)}
\end{codenott}

or

\begin{codenott}
\tc{h n = foldi g (0, 0) n} \tm{=(n, \sum_{i=0}^{n} i)}
\end{codenott}

recalling that

\begin{code}
foldi :: (a -> a) -> a -> Integer -> a\\
foldi g a 0 = a\\
foldi g a (n+1) = g (foldi g a n)
\end{code}

\newpage

% Define
% 
% \begin{code}
% g :: (Integer,Integer) -> (Integer,Integer)\\
% g (a, b) = (a+1, b + (a+1))
% \end{code}
% 
% so
% 
% \begin{codenott}
% \tc{g (0, 0) = (1, 1)}\\
% \tc{g (n, }\tm{\sum_{i=0}^{n} i}\tc{) = (n+1, } \tm{\sum_{i=0}^{n+1} i} \tc{)}
% \end{codenott}
% 
% and therefore
% 
% \begin{codenott}
% \tc{h n = foldi g (0, 0) n} \tm{=(n, \sum_{i=0}^{n} i)}
% \end{codenott}
 
We now define

\begin{code}
f :: Integer -> (Integer,Integer)\\
f n = (n, n*(n+1) `div` 2)
\end{code}

and use fusion to show that ~ \tc{f = foldi g (0, 0)} ~ from which the
desired result follows immediately.

\newslide

%We will restrict our attention to \NAT, but by making \tc{f} strict the results
%can be extended to \tc{Nat}.

Since

\begin{code}
n*(n+1) `div` 2 + (n+1) = (n+1)*(n+2) `div` 2
\end{code}

we see that

\begin{code}
(g.f) n\\
\reason{plug and grind}\\
g (n, n*(n+1) `div` 2)\\
\reason{plug and grind}\\
(n+1, (n+1)*(n+2) `div` 2)\\
\reason{plug and grind}\\
(f.(+1)) n
\end{code}

or

\begin{code}
g.f = f.(+1)
\end{code}

\newslide

Now we have

\begin{code}
f\\
\reason{\tc{foldi (+1) 0 = id}}\\
f . foldi (+1) 0\\
\reason{fusion}\\
foldi g (0, 0)
\end{code}

In the fusion step we use
\begin{enumerate}
\item \tc{f 0 = (0, 0)}
\item \tc{g.f = f.(+1)}
\end{enumerate}

However, since we are working with \NAT s rather than \tc{Nat}s,
we do not have the requirement that \tc{f} be strict.

\newpage

What we have show is that both ~ \tc{f} ~ and ~
\tc{foldi g (0,0)} ~ are the unique homomorphism from ~ \tc{(\NAT,0,(+1))} ~
to ~ \tc{(\tm{\NAT \times \NAT}, (0,0), g)}~  and therefore are equal.

Since

\begin{codenott}
\tc{f n = (n, n*(n+1) `div` 2)}\\
\tc{foldi g (0, 0) n =} \tm{(n, ~ \sum_{i=0}^{n} i)}
\end{codenott}

we have

\[\textcolor{MathColor}{\sum_{i=0}^{n} i = \frac{n(n+1)}{2}}\]

% \newslide
% 
% We will restrict our attention to \NAT, but by making \tc{f} strict the results
% can be extended to \tc{Nat}.
% 
% Since
% 
% \begin{code}
% n*(n+1) `div` 2 + (n+1) = (n+1)*(n+2) `div` 2
% \end{code}
% 
% we see that
% 
% \begin{code}
% (g.f) n\\
% \reason{plug and grind}\\
% g (n, n*(n+1) `div` 2)\\
% \reason{plug and grind}\\
% (n+1, (n+1)*(n+2) `div` 2)\\
% \reason{plug and grind}\\
% (f.(+1)) n
% \end{code}
% 
\newslide

In an induction proof we would have

\begin{description}
\item[Basic Step:] \tc{f 0 = (0, 0)}
\item[Induction Step:] \tc{h (n+1) = g (h n) = g (f n)  = f (n+1)}
\end{description}

Notice how closely these correspond to our requirements for fusion to hold.

In the fusion step we use
\begin{enumerate}
\item \tc{f 0 = (0, 0)}
\item \tc{g.f = f.(+1)}
\end{enumerate}

\end{slide}

\end{document}

