 \documentclass[10pt]{report}
\usepackage{rotating,graphics,psfrag,epsfig,amssymb, amsmath}
\oddsidemargin=-0.25in
\topmargin= -0.5in
\textwidth=7.0in
\textheight=9in
\pagestyle{empty}
\sloppy

\title{Assignment 1}
\author{Paul Vrbik : 301056796 : CMPT 731 : Fall 2006}
\date{\today}


\begin{document}

\maketitle

\section*{Assignment 2}

\subsection*{Question 1.2.4}

This can be reduced in two ways:

\begin{enumerate}
\item 
\begin{verbatim}
=add(succ(pred(Zero),Zero)
=succ(add(pred(Zero),Zero))
=succ(pred(add(Zero,Zero)))
=succ(pred(Zero)
=Zero
\end{verbatim}

\item
\begin{verbatim}
=add(succ(pred(Zero)),Zero)
=add(Zero,Zero)_
=Zero
\end{verbatim}
\end{enumerate}

\subsection*{Question 1.3.2}

Suppose $f$ and $g$ are strict then we have that $f \bot =\bot$ and $g \bot =\bot$, 

\begin{align*}
{}& \Rightarrow h \bot = f (g \bot)=f \bot= \bot \\
{}& \Rightarrow h \bot = \bot \\
{}& \Rightarrow \text{h is strict}
\end{align*}

\subsection*{Question 1.4.3}

\begin{verbatim}
mylog :: Float -> Float -> Float
mylog x b = log(x)/log(b)
\end{verbatim}

\subsection*{Question 1.4.7}
\begin{verbatim}
uncurry :: (a->b->t) -> ((a,b)->t)
uncurry g (x,y) = g x y
\end{verbatim}

\subsection*{Question 1.6.2}

Defining $\mathtt{swap(x,y)=(y,x)}$ we verify:

\begin{verbatim}
flip(curry f) x y = (curry f) y x = f(y,x)
\end{verbatim}

\begin{verbatim}
curry(f.swap) x y = (f.swap)(x,y) = f(swap(x,y))=f(y,x)
\end{verbatim}

\subsection*{Question 2.4.1}

\begin{verbatim}
=cross(f,g).cross(h,k)
=cross(f,g).pair(h.fst,k.snd)        (by defn. of cross)
=pair(f.h.fst,g.k.snd)               (by property 3)
=cross(f.h,g.k)                      (by defn. of pair)
\end{verbatim}

\subsection*{Question 2.4.3}

\begin{verbatim}
age (d1,m1,y1) (d2,m2,y2) 
        | m1>m2                         = y1-y2
        | (d1>=d2) && (m1==m2)          = y1-y2
        | otherwise                     = (y1-y2)-1
\end{verbatim}

\subsection*{Question 2.5.1}
\begin{verbatim}
case2 (Left x) = 1
case2 (Right x) = 2
\end{verbatim}

\subsection*{Question 2.5.2}

\begin{verbatim}
=case(f,g).plus(h,k)
=case(f,g).case(Left.h,Right.k)		(by defn.)
=case(case(f,g).Left.h,case(f,g).Right.l)		(by prop. 3)
=case(f.h,g.k)	(by prop 1. and prop 2)
\end{verbatim}

\subsubsection*{Question 3.2.4}

Let $\Pi(p)$ be the proposition $\Pi(p) \Leftrightarrow (m+n)+p = m+(n+p)$ $\forall m,n \in \mathbb{N}$ where $p\in \mathbb{N}$.

\begin{description}
\item[Base:] $\Pi(0)$ \\

Utilizing the definition $x+0=x$ for $p=0$ we have

$$\text{LHS} = (m+n)+0 = (m+n) = m+n$$
$$\text{RHS} = m+(n+0)=m+(n)=m+n$$

So $\Pi(0)$ is true.

\item[Assumption:] 

We will assume $\Pi(p)$, that is,  $(m+n)+p=m+(n+p)$ $\forall m,n \in \mathbb{N}$.

\item[Induction step:] (Show $\Pi(P) \Rightarrow \Pi(suc(p))$) \\

	\begin{align*}
	{}& = (m+n)+suc(p) \\
	{}& = suc((m+n)+p) \text{ by defn. of +} \\
	{}& = suc(m+(n+p)) \text{ by assumption} \\
	{}& = m+suc(n+p) \text{ by defn. of +} \\
	{}& = m+(n+succ(p))
	\end{align*}

So we have that $\Pi(p) \Rightarrow (m+n)+suc(p) = m+(n+succ(p)) \Rightarrow \Pi(suc(p))$. So by induction we have that $\Pi(p)$ $\forall p \in \mathbb{N}$ as desired.

\end{description}


\end{document}