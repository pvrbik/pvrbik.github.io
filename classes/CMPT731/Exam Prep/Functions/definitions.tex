 \documentclass[10pt]{article}
\usepackage{rotating,graphics,psfrag,epsfig,amssymb,amsmath,color}
\oddsidemargin=-0.25in
\topmargin= -0.5in
\textwidth=7.0in
\textheight=9in
\pagestyle{empty}
\sloppy

\newcommand{\imp}{\textbf}
\newcommand{\dom}[1]{\textsf{dom} (#1)}
\newcommand{\cod}[1]{\textsf{cod}(#1)}
\newcommand{\arr}[1]{\mathcal{#1}\mathrm{-arrows}}
\newcommand{\obj}[1]{\mathcal{#1}\mathrm{-objects}}
\newcommand{\state}[1]{\item[#1] \textcolor{white}{.} \\ \textcolor{white}{.} \\ }
\newcommand{\Lstate}[1]{\item[#1] \nl}
\newcommand{\id}[1]{id_{#1}}
\newcommand{\nl}{\textcolor{white}{.}\ }
\newcommand{\func}[3]{#1 : #2 \rightarrow #3}
\newcommand{\CC}{\mathcal{C}}
\newcommand{\DD}{\mathcal{D}}
\newcommand{\point}[1]{\begin{itemize} \item[$\cdot$] #1 \end{itemize}}
\newcommand{\blank}{\textcolor{white}{-}}

\begin{document} 

\begin{description}

\Lstate{$>>=$}
\begin{verbatim}
(>>=) :: Monad f => f a -> (a -> f b) -> f b
x >>= h = join (fmap h x)
\end{verbatim}
For example, with the list monad, \texttt{>>=} will have the following definition.
\begin{verbatim}
(>>=) :: [a] -> (a -> [b]) -> [b]
x >>=  h = concat (map h x)
\end{verbatim}

\state{Arrow}
Is a homorphism between objects.

\state{Cat} Is the category of small categories which has categories as objects all functors between these as arrows.

\state{Catamorphisms}
Catamorphisms are homorphisms from $Nat$, or more generally homorphisms from types defined by Haskell \texttt{data} statements.

\point{The catamorphisms for a type are exactly the functions that can be defined using the \texttt{fold} function for the type.}

\item[Category] \textcolor{white}{.}

\begin{enumerate}
\item A collection of \imp{objects}.
\item A collection or \imp{arrows}.
\item Operations assigning to each arrow $f$

\begin{enumerate}
\item An object $\dom{f}$ called the \imp{domain} of $f$, and
\item An object $\cod{f}$ called the \imp{codomain} of $f$ often expressed by $f : \dom{f} \rightarrow \cod{f}$
\end{enumerate}

\item An associative operator $\circ$ assigning to each pair of arrows, $f$ and $g$, where $\dom{f}=\cod{g},$ a \imp{composite} arrow $$f \circ g : \dom{g} \rightarrow \cod{f}.$$

\item For each object $A$, an identity arrow, $\func{id_A}{A}{A}$ satisfying the law that for any arrow $\func{f}{A}{B}$ as $${id_B \circ f}={f}={f \circ id_A}.$$
\end{enumerate}

\state{Cocone}
A cocone is the dual of a cone.
\point{A cocone for a diagram $\DD$ in a cateogry $\CC$ is a $\obj{C}$, $A$, together with an arrow $\func{f_i}{B_i}{A}$ for each object $B_i$ in $\DD$, such that for any arrow $\func{G}{B_i}{B_j}$ in $\DD$ such that $f_i = f_j \circ q$.}

\state{Colimit}
A colimit is the dual of a limit.

\state{Cone} 
For a diagram $\DD$ in a category $\CC$ is a $\obj{C}$, $A$, together with an arrow $\func{f_i}{A}{B_i}$ for each object $B_i$ in $\DD$, such that for any arrow $\func{g}{B_i}{B_j}$ in $\DD$, the following holds
$$ g \circ f_i = f_j $$
We will use the notation ${\func{f_i}{A}{B_i}}$ for the cone described above, and call the cone a $\DD$-cone to indicate that it is a cone for the diagram $\DD$.

\state{Constant Functor}
Let $\CC$ and $\DD$ be categories and let $A$ be an object of $\DD$. A constant functor $K_A$ is a functor $\func{K_A}{\CC}{\DD}$ that maps every object of $\CC$ to the object $A$ and every arrow of $\CC$ to $id_A$.

\state{Diagram}
A diagram in category $\CC$ is a collection of objects in $\CC$ together with some (or all or none) of the arrows between those objects.

\point{A diagram \imp{commutes} if whenever there are two (or more) distinct paths through the diagram from some object $A$ to some (possible other) object $B$, the composition of the arrows along the other path(s).}

\state{Dual}
The dual of a category $\mathcal{C}$ is identical to $\mathcal{C}$ but all the arrows are reversed. This is sometimes denoted $\mathcal{C}^{op}$.

\state{Endofunctor}
A functor from a category $\CC$ to the same category $\CC$.

\state{F-Algebra}
Let $\CC$ be a category and $\func{F}{\CC}{\CC}$ be an endofunctor. An $F$-algebra is a pair $(A,a)$ where $A$ is a $\obj{C}$ and $\func{a}{F A}{A}$ is a $\arr{C}$. The object $A$ is called the \imp{carrier} of the $F$-algebra.


\state{Final Object}
A final object (or \emph{terminal object}) in a category $\CC$ is an object $1$ such that for any object $A$ in $\CC$ there is exactly one arrow $\CC$ from $A$ to $1$.

\state{Functors}
A \imp{functor} $F$ from category $\mathcal{C}$ to $\mathcal{D}$ is a pair of functions.

\begin{enumerate}
\item $\func{F_0}{\obj{C}}{\obj{D}}$
\item $\func{F_1}{\arr{C}}{\arr{D}}$
\end{enumerate}
such that
\begin{enumerate}
\item If $f$ is an arrow in $\mathcal{C}$ and $\func{f}{A}{B}$ then $\func{F_1(f)}{F_0 (A)}{F_0(B)}$
\item $F_1(id_{a})=id_{F_0(A)}$, and
\item $F_1 (f \circ g) = F_1 (f) \circ F_1 (g)$ whenever $f \circ g$ defined.
\end{enumerate}


\state{Functors - HASK}
A typical definition of a functor in haskell:
\begin{description}
\item[Object Functor] \nl
\begin{verbatim}
data NAME type =
         Base | Constructor1 type | Consrtuctor2 type . . .
\end{verbatim}
\item[Function Functor] \nl
\begin{verbatim}
instance Functor Name where
         fmap f BASE = something
         fmap f (Constructor x) = Constructor (f x)
\end{verbatim}
\end{description}


\state{HASK}
In Hask, \imp{objects are Haskell types}, so the object part of a functor, $F_0$, is a function from types to types.

\point{Therefore, a type constructor is suitable for use as the object function of a functor. 
For example $$ F_0 A = [A] $$ is suitable. In this case $$F_1 = \textsf{map}$$
completes the definition of a functor.}

\state{Identity Functor}
The identity functor, $Id$, is an endofunctor that maps every object to itself and every arrow to itself. The identity functor for a category is just the identity arrow for that category viewed as an object in the category $Cat$.

\state{Initial Object}
An initial object in a category $\CC$ is an object $0$ such that for any object $A$ in $\CC$ there is exactly one arrow in $\CC$ from $0$ to $A$.

\point{Since 0 is an object in $\CC$, the only arrow from $0$ to itself is the identity arrow.}

\state{Isomorphic}
We say that two object $A$ and $B$ are isomorphic if and only if there are arrows
$$\func{f}{A}{B}$$ $$\func{g}{B}{A}$$ such that $$g \circ f = id_A$$ $$f \circ g = id_B$$

\point{Any two initial object in a category are isomorphic. The same applies to final objects.}

\state{Limit}
A limit for a diagram $\DD$ (when one exists) is a $\DD$-cone ${\func{f_i}{A}{B_i}}$ with the property that for any $\DD$-cone $\{\func{f^\prime_i}{A^\prime}{B_i}\}$ there is exactly one arrow $\func{A^\prime}{A^\prime}{A}$
$$ \rm{insert-picture-on-page-35} $$
commutes for every object $B_i$ in $\DD$,
\point{That is, a limit is a final object in a category where the objects are the cones over a given diagram}

\state{Monads}
A monad is a functor, $F$ with two natural transformations,
\begin{verbatim}
return :: a -> F a
return : Id -> F

join :: F (F a) -> F a
join : F x F -> F
\end{verbatim}

such that the following laws are satisfied

\begin{verbatim}
join.return = id
join.fmap return = id
join.fmap join = join.join
\end{verbatim}

\state{Monads - HASK}
Monads in Hask are implemented as:
\begin{verbatim}
return = wrap
join = concat
fmap = map
\end{verbatim}

\state{Natural Transformation}
Let $\CC$ and $\DD$ be two categories. Let $F$ and $G$ be functors from $\CC$ to $\DD$.

We write $\func{\tau}{F}{G}$ to mean that $\tau$ is a natural transformation from $F$ to $G$.

A natural transformation from $F$ to $G$ is an assignment $\tau$ that provides, for each $\obj{C}$ $A$, a $\arr{D}$, $\func{\tau_A}{F(A)}{G(A)}$ such that for any $\arr{C}$ $\func{f}{A}{B}$,
$$ \tau_B \circ \tt{fmap}_F(f)= \tt{fmap}_G (f) \circ \tau_A$$

\state{Object}
Is any triple $(S,a,f)$ such that
\point{$S$ is a set}
\point{$a \in s$}
\point{$\func{f}{S}{S}$}

\state{Product}
A product of objects $A$ and $B$ is an object $A \times B$ together with two arrows $\func{\pi_1}{A \times B}{A}$ and $\func{\pi_2}{A \times B}{B}$,
$$ \rm{insert-picture-on-page-41} $$
such that for any object $\CC$ with arrows $\func{a}{C}{A}$ and $\func{b}{C}{B}$.

\state{Small Category} Is a category where the collection of objects is a set and the collection of arrows is a set.

\point{Any statement that is true of an arbitrary category $\CC$ is also true of $\mathcal{C}^{op}$.}


\state{Strict}
A function is called strict if $f \bot \equiv \bot$

\point{For example $\texttt{const 5}$ is not strict since $\texttt{const 5 } \bot \not\equiv 5$.}



\end{description}

\end{document}