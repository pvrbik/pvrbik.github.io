\documentclass[12pt]{report}

\usepackage{rotating,graphics,psfrag,epsfig,amssymb,amsmath, subfigure, amsxtra, amsthm, color}
\oddsidemargin=-0.25in
\topmargin= -0.5in
\textwidth=7.0in
\textheight=9in
\pagestyle{plain}

%\sloppy

\theoremstyle{plain}
\newtheorem{theorem}{Theorem}
\newtheorem{corollary}{Corollary}
\newtheorem{lemma}{Lemma}
\newtheorem{proposition}{Proposition}
\newtheorem{conjecture}{Conjecture}

\theoremstyle{definition}
\newtheorem{definition}{Definition}
\newtheorem{example}{Example}
\newtheorem{remark}{Remark}
\newtheorem{question}{Question}
\newtheorem{problem}{Problem}


\renewcommand{\implies}{\Rightarrow}
\newcommand{\imp}{\textbf}
\newcommand{\dom}[1]{\textsf{dom} (#1)}
\newcommand{\cod}[1]{\textsf{cod}(#1)}
\newcommand{\arr}[1]{\mathcal{#1}\mathrm{-arrows}}
\newcommand{\obj}[1]{\mathcal{#1}\mathrm{-objects}}
\newcommand{\state}[1]{\item[#1] \textcolor{white}{.} \\ \textcolor{white}{.} \\ }
\newcommand{\Lstate}[1]{\item[#1] \nl}
\newcommand{\id}[1]{id_{#1}}
\newcommand{\nl}{\textcolor{white}{.}\ }
\newcommand{\func}[3]{#1 : #2 \rightarrow #3}
\newcommand{\CC}{\mathbb{C}}
\newcommand{\DD}{\mathcal{D}}
\newcommand{\point}[1]{\begin{itemize} \item[$\cdot$] #1 \end{itemize}}
\newcommand{\blank}{\textcolor{white}{-}}
\newcommand{\fieldt}{\mathbb{Z}_q[t]}
\newcommand{\fieldtx}{\mathbb{Z}_q[t][x]}
\newcommand{\impl}{\Rightarrow}
\newcommand{\eps}{\varepsilon}
\newcommand{\brac}[1]{\left( #1 \right)}
\newcommand{\Res}[1]{\textrm{Res}_{#1} }

\newcommand{\x}{\boldsymbol{x}}
\renewcommand{\H}{\boldsymbol{H}}
\newcommand{\zero}{\boldsymbol{0}}
\newcommand{\z}{\boldsymbol{z}}
\newcommand{\F}{\boldsymbol{F}}
\newcommand{\p}{\boldsymbol{p}}
\newcommand{\q}{\boldsymbol{q}}
\newcommand{\alp}{{\boldsymbol{\alpha}}}
\newcommand{\bet}{{\boldsymbol{\beta}}}

\newcommand{\M}{\mathbf{M}}
\newcommand{\A}{\mathbf{A}}
\newcommand{\B}{\mathbf{B}}
\newcommand{\C}{\mathbf{C}}
\newcommand{\X}{\mathbf{X}}
\newcommand{\0}{\mathbf{0}}
\newcommand{\II}{\mathbf{I}}
\newcommand{\ZZ}{\mathbb{Z}}
\newcommand{\QQ}{\mathbb{Q}}

\newcommand{\eval}[2]{ \left.#1\right |_{#2}}
\newcommand{\xinit}{x_{\text{init}}}
\newcommand{\idnty}{\text{Id}}

\newcommand{\pdiff}[2]{\frac{\partial #1}{\partial #2}}

\newcommand{\FF}{\mathbf{F}}
\newcommand{\JJ}{\mathbf{J}}
\newcommand{\XX}{\mathbf{X}}

\newcommand{\Xinit}{\XX_{\text{init}}}
\newcommand{\Ginit}{G_{\text{init}}}

\renewcommand{\mod}{\text{ mod }}
\newcommand{\Mod}{\hspace{2mm}\mod}

\usepackage{algorithm}
\usepackage{algorithmic}
\usepackage{subfigure}
\usepackage{tikz}

\usepackage{listings}
\lstset{language=Java}

\newcommand{\Ans}[1]{\hfill {\Huge\begin{tabular}{|c|} \hline \hfill #1\\ \hline \end{tabular}} }

\newcommand{\AnsBox}[1]{
\hspace{-1cm}\fbox{
\begin{minipage}[b][#1][c]{17.6cm}
\hfill
\end{minipage}}
}

\begin{document}

%%%TITLE PAGE%%%
{\large
\begin{center}
{\it UNIVERSITY OF WESTERN ONTARIO}
\end{center}
\begin{center}
\bf{Computer Science 1026b, Spring 2010}\\
\large{\bf{Computer Science Fundamentals I}}\\
{\large{\it MIDTERM TEST }}\\
\end{center}
}

\vspace{8cm}
\begin{center}{\Huge{Solutions}}\end{center}
\clearpage
%%%END TITLE PAGE%%%

\section*{Very Short Answers}
Each answer in this section is worth \emph{1 point} for a total of \emph{10 points}.
\subsection*{Question 1}
\emph{The following questions contain valid Java code and will provide output (without error) if entered into 
the DrJava interactions pane.} 


\renewcommand{\labelenumi}{\alph{enumi}.}
\begin{enumerate}
\item Suppose we have defined
\begin{verbatim}
int n = 4;
double x = 2.5;
\end{verbatim}
then the value of the expression \texttt{5 * x - n / 5} is: \Ans{12.5}

\item Suppose we have defined
\begin{verbatim}
int num1 = 50, num2 = 50;
\end{verbatim}
then the value of the expression \texttt{(num1 == num2)} is: \Ans{true}

\item After the following code segment is executed
\begin{verbatim}
int x = 3; 
x = x + x; 
x = x + x;
\end{verbatim}
the value of \texttt{x} is: \Ans{12}

\item After the following code segment is executed
\begin{verbatim}
int x = 3, y = 5;
x = y;
y = x;
\end{verbatim}
the value of \texttt{y} is: \Ans{5}

\item If we declare the array
\begin{verbatim}
int [] intArr = {2,4,6,8,10};
\end{verbatim}
then the value stored in \texttt{intArr[1]} is: \Ans{4}

\item After the following code segment is executed
\begin{verbatim}
int sum = 0;
int count = 1;
while (count < 5) {
    sum = sum + count;
    count = count + 2;
}
\end{verbatim}
the value of \texttt{count} is: \Ans{5}

\item After the following code segment is executed
\begin{verbatim}
bool x = (true && false) || (true && true) || false;
\end{verbatim}
the value of \texttt{x} is: \Ans{true}

\item After the following code segment is executed
\begin{verbatim}
int x = (3+3*2) % 6;
\end{verbatim}
the value of \texttt{x} is: \Ans{3}

\item After the following code segment is executed
\begin{verbatim}
int [] xs = { 1, 2, 3, 4, 5, 6, 7 };
System.out.println(xs[xs.length]);
\end{verbatim}
what is printed? (Your solution doesn't need to be precise). \Ans{\large{OutOfBounds}}

\item After the following code segment is executed
\begin{verbatim}
int n = 40; 
while ((n % 2) == 0) {
   n = n / 2;
}
\end{verbatim}
the value of \texttt{n} is: \Ans{5}

\end{enumerate}

%%%SHORT ANSWER%%%
\clearpage
\section*{Short Answers}
Each solution in this section is worth \emph{4 points} for a total of \emph{20 points}.
\subsection*{Question 2}
\begin{enumerate}
\item What will be the output of the following code segment?
\begin{verbatim}
for (int row = 1; row <=3; row ++) {
    for (int count = 1; count <= (4 - row); count++) {
        System.out.print("*");
    }
    System.out.println();
}
\end{verbatim}
\item[\textsc{Solution.}]
\begin{verbatim}
  ***
  **
  *
\end{verbatim}
\vspace{2.5cm}
\item What happens when the following code segment is executed? Why?
\begin{verbatim}
int sum = 0;
int count = 5;
while (count > 1) {
    sum = sum + count;
    count = count + 2;
}
System.out.println(sum);
\end{verbatim}
\vspace{.5cm}
\item[\textsc{Solution.}] This will cause an infinite loop. The value of \texttt{count}, starting at five, is \emph{increased} at every iteration. Thus count will \emph{always} be greater than one and the loop will neve terminate as a result of this.

\clearpage

\item In the space provided, write a \texttt{for} loop that is equivalent to the following \texttt{while} loop. More precisely, 
after executing this for loop the value
in {\tt sum} will be the same as it is after executing the {\tt while} loop
below.


\begin{verbatim}
int sum = 0;
int i = 3;
while (i < 100) {
    sum = sum + i;
    i = i + 3;
}
\end{verbatim}

\item[\textsc{Solution.}] 
\begin{verbatim}
    int sum = 0, i;

    for (i=3; i<100; i=i+3) {
        sum = sum + i;
    }
\end{verbatim}


%\item Write a method with the header \texttt{public void aboutFace()} for inclusion in the \texttt{Turtle} class 
%that causes a turtle to ``turn around''; that is, to face in the opposite direction it is currently pointing.
\vspace{1.5cm}
\item Suppose \texttt{Yertle} and \texttt{Franklin} are two Turtle objects in the same World. Provide a code segment that makes the two turtles face each other.
\vspace{.5cm}
\item[\textsc{Solution.}]
\begin{verbatim}
    Yertle.turnToFace(Franklin);
    Franklin.turnToFace(Yertle);
\end{verbatim}

\vspace{1.5cm}
\item Write a code segment that creates a 400px by 300px (width by height) Picture object with all pixels set to red. Assume that this code will be part of a
{\tt main} program and \emph{not} in the Picture class.
For the red color, use {\tt Color.RED} from the class {\tt java.awt.Color}.
\vspace{.5cm}
\item[\textsc{Solution.}]
\begin{verbatim}
    Picture MyPic = new Picture(400, 300);
    Pixel[] PixelArray = MyPic.getPixels();
        
    for (int i=0; i<PixelArray.length; i++) {
       PixelArray[i].setColor( java.awt.Color.RED ); 
    }
\end{verbatim}


\end{enumerate}

\clearpage
\section*{Long Answers}
Each solution in this section is worth \emph{10 points} for a total of \emph{20 points}.
\subsection*{Question 3}
\begin{enumerate}
\item Write an algorithm in pseudo-code (i.e. code mixed with English) which prints the following pattern when given \texttt{n}. Your method should work for any value of $n$ and not only on the examples given below. 
Note that for a given \texttt{n}, the printed pattern has \texttt{n+1} rows
and \texttt{n} columns.

\begin{center}
\begin{tabular}{l  c l c l }
$n=2$ 		&\hspace{50px}\hfill	& $n=3$ 		&\hspace{50px}\hfill& $n=4$ \\
\hline
\texttt{xx}		&&\texttt{xxx}	&& \texttt{xxxx} \\
\texttt{xo}		&&\texttt{xxo}	&& \texttt{xxxo} \\
\texttt{oo}		&&\texttt{xoo}	&& \texttt{xxoo} \\
			&&\texttt{ooo}	&& \texttt{xooo} \\
			&&			&& \texttt{oooo} \\
\hline
\end{tabular}
\end{center}

\item[\textsc{Solution}] \blank

for $i$ from $0$ to $n$ do\\
....print $n-i$ many \texttt{x}'s\\
....print $i$ many \texttt{o}'s\\
....print a new line\\
end do
\vspace{1cm}

\item Write a java class method \texttt{static void PrintPattern (int n) } that prints the same pattern as in Part a.

\item[\textsc{Solution}]
\begin{verbatim}
    public static void PrintPattern (int n) {
        for (int i=0; i<=n; i++) {
            for (int j=0; j<n-i; j++) {
                System.out.print("x"); 
            }
            for (int j=0; j<i; j++) {
                System.out.print("o"); 
            }
            System.out.println();
        }
    }
\end{verbatim}

\end{enumerate}

\clearpage
\subsection*{Question 4}
Write a java object method \texttt{void nSidedPolygon (int n, int s)} for inclusion in the Turtle class that takes as input $n$ and $s$ and draws an $n$-sided polygon with side length $s$. 
You may assume that drawing these polygons will never cause  your turtle
to go out of its world boundaries. 
The ``exterior angle'' will be ${360^{\circ}}/{n}$ as illustrated for $n=3$ and $n=6$ in the figure below. \\
\begin{center}
% !TEX root = OurMid.tex
\begin{tikzpicture}
% The graphic
%\draw[style=help lines,step=1cm] (0,0.5) grid (16,5.5);

\draw (1,1) -- (3,4);
\draw (3,4) -- (5,1);
\draw (5,1) -- (1,1);

\draw (5.5,1) -- (6,1);
\draw (5,1) -- (5.5,1) arc(0:120:5mm);
\draw (5,2) node[right=-15pt] {$\frac{360^{\circ}}{3} = 120^{\circ}$};

%\draw (3,1) node[below] {$s$};
\draw (2,2.75) node[left] {$s$};
%\draw (4.5,2.75) node[left] {$s$};

%2.23606798
\draw (10,1) -- (12.23606798,1);
\draw (8.76393202,3) -- (10,1);
\draw (13.23606798, 3) -- (12.23606798,1);
\draw (13.23606798, 3) -- (12.23606798,5);
\draw (10,5) -- (8.76393202,3);
\draw (10,5) -- (12.23606798,5);

\draw (12.23606798,1) -- (13.5,1) arc(0:60:13.4mm);
\draw (13.5,1) -- (14,1);
\draw (14,2) node[right=-25pt] {$\frac{360^{\circ}}{6} = 60^{\circ}$};
\draw (9,4) node {$s$};


\end{tikzpicture}
\\
\end{center}

\begin{enumerate}
\item[\textsc{Solution.}] \blank
\begin{verbatim}
    public void nSidedPolygon (int n, int s) {
        for (int i=0; i<n; i++) {
            this.forward(s);
            this.turn(360 / n);
        }
    } 
\end{verbatim}

\end{enumerate}



\end{document}
