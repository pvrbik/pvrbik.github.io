\documentclass[12pt]{report}
\usepackage{fancyvrb}

\newcommand{\rem}{\text{ rem }}
\newcommand{\quo}{\text{ div }}
\renewcommand{\div}{\text{ div }}

\usepackage{rotating,graphics,psfrag,epsfig,amssymb,amsmath, subfigure, amsxtra, amsthm, color}
\usepackage{fancyvrb}
\oddsidemargin=-0.25in
\topmargin= -0.5in
\textwidth=7.0in
\textheight=9in

\usepackage{listings}
\lstdefinelanguage{maple}
	{morekeywords={true, false, try, catch, return, break, error, 
	               module, export, local, option, in, use,
                 and, or, not, xor, xnor,
                 if, then, elif, else, fi,
                 while, for, from, by, to, do, od,
                 proc, nargs, local, global, end, NULL}}
\lstset{language=maple,numbers=left,basicstyle=\footnotesize,numberstyle=\tiny, breakatwhitespace=false,frame=single,morecomment=[l]{\#},stringstyle=\ttfamily}

\usepackage[usenames,dvipsnames]{xcolor}
\definecolor{MapleRed}{rgb}{1,0,0}
\definecolor{MapleBlue}{rgb}{0,0,1}
\definecolor{MaplePink}{rgb}{1,0,1}

%\definecolor{MapleRed}{rgb}{0,0,0}
%\definecolor{MapleBlue}{rgb}{0,0,0}

\renewcommand{\arraystretch}{1.25}

%\pagestyle{headings}

%\sloppy

\theoremstyle{plain}
\newtheorem{theorem}{Theorem}
\newtheorem{corollary}{Corollary}
\newtheorem{lemma}{Lemma}
\newtheorem{proposition}{Proposition}
\newtheorem{conjecture}{Conjecture}

\theoremstyle{definition}
\newtheorem{definition}{Definition}
\newtheorem{example}{Example}
\newtheorem{remark}{Remark}
\newtheorem{question}{Question}
\newtheorem{problem}{Problem}


\renewcommand{\implies}{\Rightarrow}
\newcommand{\imp}{\textbf}
\newcommand{\dom}[1]{\textsf{dom} (#1)}
\newcommand{\cod}[1]{\textsf{cod}(#1)}
\newcommand{\arr}[1]{\mathcal{#1}\mathrm{-arrows}}
\newcommand{\obj}[1]{\mathcal{#1}\mathrm{-objects}}
\newcommand{\state}[1]{\item[#1] \textcolor{white}{.} \\ \textcolor{white}{.} \\ }
\newcommand{\Lstate}[1]{\item[#1] \nl}
\newcommand{\id}[1]{id_{#1}}
\newcommand{\nl}{\textcolor{white}{.}\ }
\newcommand{\func}[3]{#1 : #2 \rightarrow #3}
\newcommand{\CC}{\mathbb{C}}
\newcommand{\DD}{\mathcal{D}}
\newcommand{\point}[1]{\begin{itemize} \item[$\cdot$] #1 \end{itemize}}
\newcommand{\blank}{\textcolor{white}{-}}
\newcommand{\fieldt}{\mathbb{Z}_q[t]}
\newcommand{\fieldtx}{\mathbb{Z}_q[t][x]}
\newcommand{\impl}{\Rightarrow}
\newcommand{\eps}{\varepsilon}
\newcommand{\brac}[1]{\left( #1 \right)}
\newcommand{\Res}[1]{\textrm{Res}_{#1} }

\newcommand{\x}{\boldsymbol{x}}
\renewcommand{\H}{\boldsymbol{H}}
\newcommand{\zero}{\boldsymbol{0}}
\newcommand{\z}{\boldsymbol{z}}
\newcommand{\F}{\boldsymbol{F}}
\newcommand{\p}{\boldsymbol{p}}
\newcommand{\q}{\boldsymbol{q}}
\newcommand{\alp}{{\boldsymbol{\alpha}}}
\newcommand{\bet}{{\boldsymbol{\beta}}}

\newcommand{\V}{\mathbf{V}}
\newcommand{\W}{\mathbf{W}}
\newcommand{\M}{\mathbf{M}}
\newcommand{\A}{\mathbf{A}}
\newcommand{\B}{\mathbf{B}}
\newcommand{\C}{\mathbf{C}}
\newcommand{\X}{\mathbf{X}}
\newcommand{\0}{\mathbf{0}}
\newcommand{\II}{\mathbf{I}}
\newcommand{\ZZ}{\mathbb{Z}}
\newcommand{\QQ}{\mathbb{Q}}

\newcommand{\eval}[2]{ \left.#1\right |_{#2}}
\newcommand{\xinit}{x_{\text{init}}}
\newcommand{\idnty}{\text{Id}}

\newcommand{\pdiff}[2]{\frac{\partial #1}{\partial #2}}

\newcommand{\FF}{\mathbf{F}}
\newcommand{\JJ}{\mathbf{J}}
\newcommand{\XX}{\mathbf{X}}

\newcommand{\Xinit}{\XX_{\text{init}}}
\newcommand{\Ginit}{G_{\text{init}}}

\renewcommand{\mod}{\text{ mod }}
\newcommand{\Mod}{\hspace{2mm}\mod}

\newcommand{\rem}{\text{ rem }}
\newcommand{\quo}{\text{ quo }}
\renewcommand{\div}{\text{ div }}
\newcommand{\MM}{\textsf{M}}

\def\MapleInput#1{\noindent{\tt{[}\color{MapleRed}{\tt{>}\tt{#1}}}}
\def\MapleOutput#1{{\begin{center} \emph{\color{MapleBlue}{#1}} \end{center}}}
\def\MapleWarning#1{\noindent{{\small {\tt \color{MaplePink}{#1} }}}}


\DefineVerbatimEnvironment% 
{MapleInputs}{Verbatim}{formatcom=\color{MapleRed}}

\renewcommand{\FancyVerbFormatLine}[1]{\makebox[0.43cm][l]{\color{black}{[}\color{MapleRed}{>}}#1}

\newcommand{\PutEnd}{\begin{center}
\vspace{\fill}\ \newline
{\tiny \rm $ $Typeset by: Paul Vrbik$ $ }
{\tiny \rm $ $Compiled on \today$ $ }
\end{center}
}

\DefineVerbatimEnvironment% 
{Maple}{Verbatim} {numbers=left,numbersep=2mm,frame=lines,framerule=0.8mm}

\DefineVerbatimEnvironment% 
{Output}{Verbatim} {frame=lines,framerule=0.0mm}

\title{Assignment 1 \\ CS 9566A}
\author{Paul Vrbik \\ 250389673}
\date{\today}

\begin{document}
\maketitle

\begin{titlepage}
\newpage
\mbox{}
\newpage
\end{titlepage}

\subsection*{Question 1 - Karatsuba's algorithm}

\begin{Maple}
karatsuba:=proc(f,g)
local df,dg,n,x,m,F0,G0,F1,G1,b,c,a;
    
printf("Call karatsuba( %a , %a )\n",f,g);
    
    if nops(indets(f) union indets(g)) > 1 then error "univariates only"; end if;
    
    df,dg:=degree(f),degree(g);
    
    n:=max(df,dg);
    
    if n<1 then
        return f*g;
    end if;
    
    n:=2^(trunc(log[2](n))+1);
    
    x:=indets(f) union indets(g); #assuming = indet(g)
    x:=x[1];
    m:=n/2;
    
    F0:=rem(f,x^m,x,'F1');
    G0:=rem(g,x^m,x,'G1');
    
    a:=procname(F0,G0);
    b:=procname(F1,G1);
    c:=procname(F0+F1,G0+G1);
    
    return b*x^n+(c-a-b)*x^m+a;
end proc:
\end{Maple}

\begin{Output}
> f:=1-2*x^3+3*x^3;
                                         3
                               f := 1 + x

> g:=1-x-2*x^2-x^3;
                                          2    3
                          g := 1 - x - 2 x  - x

> h:=karatsuba(f,g);
Call karatsuba( 1+x^3 , 1-x-2*x^2-x^3 )
Call karatsuba( 1 , 1-x )
Call karatsuba( 1 , 1 )
Call karatsuba( 0 , -1 )
Call karatsuba( 1 , 0 )
Call karatsuba( x , -x-2 )
Call karatsuba( 0 , -2 )
Call karatsuba( 1 , -1 )
Call karatsuba( 1 , -3 )
Call karatsuba( 1+x , -1-2*x )
Call karatsuba( 1 , -1 )
Call karatsuba( 1 , -2 )
Call karatsuba( 2 , -3 )
                        2         4      2       2
                h := (-x  - 2 x) x  + (-x  - 2) x  + 1 - x

> expand(h - f*g);
                                    0
\end{Output}

\subsection*{Question 2 - Fourier Transform}
\subsubsection*{FFT}

\begin{Maple}
FFT:=proc(f,n)
local w;
    w:=exp(2*I*Pi/n);
    return h_FFT(f,n,w,indets(f)[1]);
end proc;

h_FFT:=proc(f,n,w,x) #naive implemenation
local df,Feven,Fodd,V,Vp;
printf("Call FFT( %a , %a ) with w=%a\n",f,n,w);
    
    if n=1 then return [f]; end if;
    
    df:=degree(f,x);
    Feven:=add( x^i*coeff(f,x,2*i),i=0..trunc(df/2));
    Fodd:=add( x^i*coeff(f,x,2*i+1),i=0..trunc(df/2));
    
    V:=procname(Feven,n/2,w^2,x);
    Vp:=procname(Fodd,n/2,w^2,x);
    
#   return [f(w^0),...,f(w^(n-1))];
    return [seq( V[i mod n/2 + 1] + w^i*Vp[i mod n/2 + 1], i=0..n-1 )];
end proc:
\end{Maple}

\begin{Output}
> P:=1-x-2*x^2-3*x^3;
                                         2      3
                         P := 1 - x - 2 x  - 3 x

> n:=4:
> A:=FFT(P,n);
Call FFT( 1-x-2*x^2-3*x^3 , 4 ) with w=I
Call FFT( 1-2*x , 2 ) with w=-1
Call FFT( 1 , 1 ) with w=1
Call FFT( -2 , 1 ) with w=1
Call FFT( -1-3*x , 2 ) with w=-1
Call FFT( -1 , 1 ) with w=1
Call FFT( -3 , 1 ) with w=1
                      A := [-5, 3 + 2 I, 3, 3 - 2 I]

#check
> w:=exp(2*I*Pi/n):
> B:=[ seq( eval(P,x=w^i), i=0..n-1) ]:
> map( evalc, A-B );
                               [0, 0, 0, 0]


\end{Output}

\subsubsection*{Inverse FFT}
\begin{Maple}
InvFFT:=proc(A,n)
local w,f,B;
    w:=exp(2*I*Pi/n);
    
    f:=add( A[i+1]*x^(i), i=0..nops(A)-1 );
    
    B:=FFT(f,n,1/w,x);
    B:=B/n;
    
    return evalc(add( B[i+1]*x^i, i=0..nops(B)-1 ));
    
end proc:
\end{Maple}

\begin{Output}
> InvFFT(B,n);
Call FFT( -5+(3+2*I)*x+3*x^2+(3-2*I)*x^3 , 4 ) with w=I
Call FFT( -5+3*x , 2 ) with w=-1
Call FFT( -5 , 1 ) with w=1
Call FFT( 3 , 1 ) with w=1
Call FFT( 3+2*I+(3-2*I)*x , 2 ) with w=-1
Call FFT( 3+2*I , 1 ) with w=1
Call FFT( 3-2*I , 1 ) with w=1
                                      2    3
                         1 - 3 x - 2 x  - x
\end{Output}

\subsection*{Question 3}
\begin{enumerate}
\item $S(n) = 3S(n/2) + n^{1.2}$. For the master theorem $a=3, b=2, c=1, k=1.2$ where $\log_2 3 = 1.58 > 1.2 = k$. Therefore $\mathbf{S(n) = O(n^{\textbf{log}_2 3})}$ as given by the master theorem.
\item $S(n) = 4S(n/2) + n^{1.75}$. For the master theorem $a=4, b=2, c=1, k=1.75$ where $\log_2 4 = 2 > 1.75 = k$. Therefore $\mathbf{S(n) = O(n^{\textbf{log}_2 4}) = O(n^2)}$ as given by the master theorem.
\item $S(n) = 3S(n/2) + n^{1.5}$. For the master theorem $a=3, b=2, c=1, k=1.5$ where $\log_2 3 = 1.58 > 1.5 = k$. Therefore $\mathbf{S(n) = O(n^{\textbf{log}_2 3})}$ as given by the master theorem.
\end{enumerate}


\subsection*{Question 4}

Let $T(1)=1$ and $T(n) = 3T(n/2) + \ell n$ with $n$ a power of two and $\ell$ constant.
\begin{align*}
T(2) &= 3T(n/2)+2 \ell = 3T(1) + 2 \ell = 3 + 2 \ell \\
T(4) &= 3T(4/2) + 4 \ell = 3T(2) + 4 \ell = 9 + 6 \ell + 4 \ell = 9 + 10 \ell \\
T(8) &= 3T(8/2) + 8 \ell = 3T(4)+8\ell = 27+30\ell+8\ell=27+38\ell
\end{align*}

Prove (by induction) that
\begin{equation}
T(2^n) = 3^n + 2(3^{n-1}+2 \cdot 3^{n-2} + \cdots + 2^{n-2} \cdot 3 + 2^{n-1}) \ell. \label{pr}
\end{equation}

The base case has already been given in the first part of this question. For the induction hypothesis assume
$$T(2^{n-1}) = 3^{n-1} + 2(3^{n-2}+2 \cdot 3^{n-3} + \cdots + 2^{n-3} \cdot 3 + 2^{n-2}) \ell$$
and proceed in the natural way:
\begin{align*}
T(2^n) &= 3T(2^n/2) + \ell 2^n & \text{by definition}& \\ 
&= 3T(2^{n-1}) + \ell 2^n & & \\
&= 3(3^{n-1} + 2(3^{n-2}+2 \cdot 3^{n-3} + \cdots + 2^{n-3} \cdot 3 + 2^{n-2}) \ell)+\ell2^n & \text{by assumption} & \\
&=3^n+2(3^{n-1}+2\cdot3^{n-2}+\cdots+2^{n-3}\cdot3^2+2^{n-2}\cdot3)\ell+\ell2^n & &\\
&=3^n + 2(3^{n-1}+2\cdot3^{n-2}+ \cdots + 2^{n-2}\cdot 3 + 2^{n-1})\ell. & &
\end{align*}
Therefore (\ref{pr}) has been proved.

It is clear that RHS of (1) is dominated by $3^n$ in the limit. $T(2^n) = O(3^n)$ is a direct consequence of this.

\subsection*{Question 5 - Uniqueness of Quotient and Remainder}
Suppose that we have $q,g,q^\prime$ and $r^\prime$ all generated by Euclidean division, satisfying $$ f = q \cdot g + r = q' \cdot g + r' $$ where $q \neq q'$ and $r \neq r'$. By the division algorithm neither $r$ or $r'$ has a term divisible by $\text{LT}{(g)}$. However, $(q-q') \cdot \text{LT}(g)$ does (any term of $(q-q^\prime)g_1$). Since $r - r'=(q - q') \cdot g$ it must also be the case that $r-r'$ has a term divisible by $\text{LT}(g)$. This is only possible when $r - r' = 0$ contradicting our assumption. Therefore $q$ and $r$ must be unique.

\subsection*{Question 6 - Division Rules}
\subsubsection{Addition}
Show: $$(A_1 \rem B) + (A_2 \rem B) = (A_1 + A_2 )\rem B.$$

\begin{proof}
By the division algorithm we have unique $q_1,q_2,r_1=A_1 \rem B$ and $r_2 = A_2 \rem B$ such that $A_1 = q_1 B + r_1$ and $A_2 = q_2 B + r_2$. Since adding $A_1$ and $A_2$ gives $$A_1 + A_2 = (q_1 + q_2)B + (r_1+r_2)$$
$(r_1 + r_2)$ must be the unique remainder when dividing $A_1 + A_2$ by $B$. That is, $(A_1+A_2) \rem B = r_1 + r_2 = (A_1 \rem B) + (A_2 \rem B)$. 
\end{proof}

\subsubsection{Multiplication}
Show: $$( (A_1 \rem B) \times (A_2 \rem B) ) \rem B = (A_1 \times A_2) \rem B.$$

\begin{proof}
By the division algorithm we have unique $q_1,q_2,r_1=A_1 \rem B$ and $r_2 = A_2 \rem B$ such that $A_1 = q_1 B + r_1$ and $A_2 = q_2 B + r_2$. Multiplying $A_1$ by $A_2$ gives
\begin{align*}
A_1 \times A_2 &= q_1q_2B^2 + r_2q_1B + r_1q_2B + r_1r_2\\
&= q_1q_2B^2 + r_2q_1B + r_1q_2B + (r_1r_2 \quo B) B + r_1r_2 \rem B \\
&= (q_1q_2B + r_2q_1 + r_1q_2 + r_1r_2 \quo B)B + r_1r_2 \rem B.
\end{align*}
By the division algorithm $r_1r_2 \rem B$ must be the unique remainder when dividing $A_1 \times A_2$ by $B$. That is, $(A_1 \times A_2) \rem B = r_1r_2 \rem B = ( (A_1 \rem B) \times (A_2 \rem B) ) \rem B$.
\end{proof}

\subsection*{Question 7 - Modular Multiplication}
First we compute
\begin{align*}
c_0 + c_1x &= (a_0+a_1x)(b_0+b_1x) \rem (x^2+2) \\
&= a_0b_0 + (a_0b_1 + a_1b_0)x + a_1b_1x^2 \rem(x^2+2) \\
&= (a_0b_0-2a_1b_1) + (a_0b_1 + a_1b_0)x
\end{align*}
and copy the trick of Karatsuba to do
\begin{align*}
c_0 &= a_0b_0 - a_1b_1 - a_1b_1\\
c_1 &= (a_0+a_1)(b_0+b_1) - a_0b_0 - a_1b_1  \\
&= (a_0+a_1)(b_0+b_1) +a_1b_1- c_0
\end{align*}
which establishes the required modular multiplication using only three products ($a_0b_0, a_1b_1$ and $(a_0+a_1)(b_0+b_1)$).

\subsection*{Question 8 - Alternative Quadratic Multiplication}

Let $f = f_0 + f_1 x + f_2x^2$ and $g= g_0 + g_1 x + g_2 x^2$ and 
\begin{align*}
h &= fg = h_0 + h_1x + h_2x^2 + h_3x^3 + h_4 x^4 \\
&= f_0 g_0 + (f_0 g_1 + f_1 g_0)x + (f_0 g_2 + f_1g_1 + f_2g_0)x^2 + (f_1g_2 + f_2g_1)x^3 + f_2g_2 x^4.
\end{align*}
We show:
\begin{align*}
H_0 &= F_0G_0 = f(0)g(0) = f_0 g_0 = h(0) \\
H_1 &= F_1G_1 = f(1)g(1) \\
&= f_0 g_0 + f_0 g_1 + f_1 g_0 + f_0 g_2 + f_1g_1 + f_2g_0 + f_1g_2 + f_2g_1 + f_2g_2\\  
&= h(1) \\
H_{-1} &= F_{-1}G_{-1} = f(-1)g(-1) \\
&= f_0 g_0 - (f_0 g_1 + f_1 g_0) + (f_0 g_2 + f_1g_1 + f_2g_0) - (f_1g_2 + f_2g_1) + f_2g_2\\ 
&= h(-1) \\
H_{x^2+2} &= F_{x^2+2}G_{x^2+2} \rem (x^2 + 2) \\
&= ( f_0-2f_2 + f_1x)(g_0-2g_2+g_1x) \rem (x^2+2)\\
&= f_0g_0 - 2f_0g_2 -2f_2g_0 +4f_2g_2 + (f_0g_1x-2f_2g_1+f_1g_0-2f_1g_2)x+f_1g_1x^2 \rem(x^2+2)\\
&= f_0g_0+(f_0g_1+f_1g_0)x -2 (f_0g_2 -f_1g_1 +f_2g_0) -2(f_1g_1+f_2g_1)x + 4f_2g_2 \\
&=h \rem (x^2+2).
\end{align*}


To recover $h$ from $H_0, H_1, H_{-1}, H_{x^2+2}$ we solve:
\begin{align*}
H_0 &= h_0 \\
H_1 &= h_0 + h_1 + h_2 + h_3 + h_4 \\
H_{-1} &= h_0 - h_1 + h_2 - h_3 + h_4 \\
H_{x^2+2} &= (h_0 - 2h_2 + 4h_4) + (h_1-2h_3)x = c_0 + c_1x
\end{align*}
which is equivalent to solving 
$$\left[\begin{array}{rrrrr}1 & 0 & 0 & 0 & 0 \\1 & 1 & 1 & 1 & 1 \\1 & -1 & 1 & -1 & 1 \\1 & 0 & -2 & 0 & 4 \\1 & 0 & 0 & -2 & 0\end{array}\right]\left[\begin{array}{c}h_0 \\h_1 \\h_2 \\h_3 \\h_4\end{array}\right]=\left[\begin{array}{l}H_0 \\H_1 \\H_{-1} \\c_0 \\c_1\end{array}\right]$$
and can be done with three divisions by two (done in constant time via bit shifting).

We require one multiplication (each) to get $H_0, H_1, H_{-1}$ and three to do $H_{x^2+2}$ (by Question 7.) \textbf{for a total of six multiplications}. 

To apply recursively we observe that any polynomial $f  = f_0 + f_1x + \cdots + f_nx^n \in \mathcal{R} [x]$ can be written as
\begin{align*}
F &= (f_0+f_1x+\cdots+f_{m-1}x^{m-1}) +(f_m+\cdots+f_{2m-1}x^{m-1})X+(f_{2m}+f_nx^{n-2m})X^2 \\
&= a_0 + a_1 X + a_2 X^2
\end{align*}
where $X = x^m$, $m = \lceil n / 3 \rceil$. So now we may use our multiplication scheme to multiply any two polynomials $f,g \in \mathcal{R}[x]$ by first representing them as quadratics polynomials (as above) and recursively applying the scheme to resolve the coefficient products (which will be polynomials of degree less than $m$). This recursion is guaranteed to terminate as the degrees of the coefficients at each step form a strictly descending chain.

Suppose that $M(n)$ is the number of products required to multiply two polynomials of degree less than $n$. For this algorithm, at each step, we do six multiplications of polynomials one third of the original degree. More explicitly we have $M(n) = 6M(n/3)$. We can use the master theorem which implies, in this case, that \textbf{the complexity of this scheme is} $\mathbf{O(n^{log_3{6}})}$.

\subsection*{Question 9}
$$ \text{time to complete }< (20\text{ min})\times(8 \text{ questions}) = 2.7 \text{ hours}$$

\end{document}