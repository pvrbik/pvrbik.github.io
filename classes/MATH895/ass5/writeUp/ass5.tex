 \documentclass[10pt]{report}
\usepackage{rotating,graphics,psfrag,epsfig,amssymb,amsmath, subfigure, color}
\oddsidemargin=-0.25in
\topmargin= -0.5in
\textwidth=7.0in
\textheight=9in
\pagestyle{empty}
\sloppy

\newcommand{\imp}{\textbf}
\newcommand{\dom}[1]{\textsf{dom} (#1)}
\newcommand{\cod}[1]{\textsf{cod}(#1)}
\newcommand{\arr}[1]{\mathcal{#1}\mathrm{-arrows}}
\newcommand{\obj}[1]{\mathcal{#1}\mathrm{-objects}}
\newcommand{\state}[1]{\item[#1] \textcolor{white}{.} \\ \textcolor{white}{.} \\ }
\newcommand{\Lstate}[1]{\item[#1] \nl}
\newcommand{\id}[1]{id_{#1}}
\newcommand{\nl}{\textcolor{white}{.}\ }
\newcommand{\func}[3]{#1 : #2 \rightarrow #3} 
\newcommand{\CC}{\mathcal{C}}
\newcommand{\DD}{\mathcal{D}}
\newcommand{\point}[1]{\begin{itemize} \item[$\cdot$] #1 \end{itemize}}
\newcommand{\blank}{\textcolor{white}{-}}
\newcommand{\fieldt}{\mathbb{Z}_q[t]}
\newcommand{\fieldtx}{\mathbb{Z}_q[t][x]}
\newcommand{\impl}{\Rightarrow}
\newcommand{\eps}{\varepsilon}
\newcommand{\brac}[1]{\left( #1 \right)}
\newcommand{\Res}[1]{\textrm{Res}_{#1} }
\renewcommand{\deg}[1]{\texttt{degree}(#1)}
\renewcommand{\implies}{\Rightarrow}
\newcommand{\lcm}[1]{\texttt{lcm}\brac{#1}}
\renewcommand{\gcd}[1]{\texttt{gcd}\brac{#1}}

\title{Assignment 5}
\author{Paul Vrbik : 301056796 : MATH 895 : Summer 2007}
\date{\today}

\begin{document}

\maketitle

%QUESTION 1
\section*{Question 1: Minimal Polynomials}

\subsection*{(i)}

Let $\alpha$ be algebraic over $\mathbb{C}$. Let $m(z) \in \mathbb{Q}[z]$ be a non-zero monic polynomial of minimal degree such that $m(\alpha) = 0$. Prove that $m(z)$ is irreducible over $\mathbb{Q}$ and unique.

\begin{description}
\state{Uniqueness} 
Suppose there are two monic polynomials $m(z)$ and $n(z) \in \mathbb{Q}[z]$ of minimal degree ($\deg{m}=\deg{n}$) where $m(\alpha)=n(\alpha)=0$. Consider the polynomial $f(z)=m(z)-n(z)$ where $f(\alpha)=m(\alpha)-n(\alpha)=0$. Since $m(z)$ and $n(z)$ are monic and of the same degree their leading terms will cancel when taking the difference. This implies that $\deg{f} < \deg{m}$ and it follows that $m(z)$ is not minimal ($f$ is), a contradiction. Therefore $m(z)$ is unique.$\hfill \Box$

\state{Irreducibility}
Suppose $m(z)=f(z) \cdot g(z)$ where $f,g \in \mathbb{Q}[z]$, $\deg{f} \neq 0$ and $\deg{g} \neq 0$. We have $m(\alpha)=0 \implies f(\alpha) \cdot g(\alpha) = 0 \implies f(\alpha)=0 \vee g(\alpha)=0$. Without loss of generality suppose $f(\alpha)=0$. Since $\deg{m}=\deg{f}+\deg{g}$ we have that $\deg{f}<\deg{m}$ which means that $m(z)$ is not minimal ($f$ is), a contradiction. Therefore $m(z)$ is irreducible.$\hfill \Box$

\end{description}

\subsection*{(ii)}

I define the maple functions:
\begin{verbatim}
> r:=(ma,mb)->resultant(ma,resultant(mb,z-x-y,y),x):
> check:=(f,x)->expand(eval(f,z=x)):
\end{verbatim}


\begin{description}
\item[\underline{$\alpha = 1 + \sqrt{2}$}] \blank 
\begin{verbatim}
> c:=1+sqrt(2):
> 
> ma:=x-1:
> mb:=y^2-2:
> 
> mc:=sort(r(ma,mb));
                                        2
                                 mc := z  - 2 z - 1

> check(mc,c);
                                         0

> irreduc(mc);
                                        true
\end{verbatim}


\item[\underline{$\alpha = 1 + \sqrt{2} + \sqrt[4]{2}$}] \blank
\begin{verbatim}
> c:=1+2^(1/2)+2^(1/4):
> 
> ma:=eval(mc,z=x):
> mb:=y^4-2:
> 
> mc:=sort(r(ma,mb));
               8      7       6      5       4       3       2
        mc := z  - 8 z  + 20 z  - 8 z  - 30 z  + 24 z  - 52 z  + 120 z - 63

> check(mc,c);
                                         0

> 
> mc1,mc2:=op(factor(mc));
                     4      3      2              4      3      2
        mc1, mc2 := z  - 4 z  + 2 z  + 12 z - 9, z  - 4 z  + 2 z  - 4 z + 7

> check(mc1,c); #non-zero so mc2 must be the minimal poly.
                                    1/2       (1/4)
                                16 2    + 16 2

> check(mc2,c);
                                         0

> 
> mc:=mc2:
> irreduc(mc);
                                        true
\end{verbatim}

\item[\underline{$\alpha = \sqrt 2 + \sqrt 3 + \sqrt 5$}] \blank
\begin{verbatim}
> c:=sqrt(2)+sqrt(3)+sqrt(5):
> 
#We have that the minimal polynomial for sqrt(2)+sqrt(3) is
> ma:=x^4-10*x^2+1:
> mb:=y^2-5:
> 
> mc:=r(ma,mb);
                                6    8        4              2
                     mc := -40 z  + z  + 352 z  + 576 - 960 z

> check(mc,c);
                                         0

> irreduc(mc);
                                        true
\end{verbatim}
\end{description}

\newpage
%QUESTION 2
\section*{Question 2: Cyclotomic Polynomials}

\subsection*{(i)}

\begin{verbatim}
> T:=table():
> T[1]:={x-1}:
> all:=T[1]:
> for n from 2 to 12 do
>     fs:=factor(x^n-1):
>     T[n]:={op(fs)} minus all:
>     all:=all union T[n]:
> end do:
> for i from 1 to 12 do
>     printf("%d  %a \n", i, sort(op(T[i]))):
> end do: 

n  \phi_n(x)
1  x-1 
2  x+1 
3  x^2+x+1 
4  x^2+1 
5  x^4+x^3+x^2+x+1 
6  x^2-x+1 
7  x^6+x^5+x^4+x^3+x^2+x+1 
8  x^4+1 
9  x^6+x^3+1 
10  x^4-x^3+x^2-x+1 
11  x^10+x^9+x^8+x^7+x^6+x^5+x^4+x^3+x^2+x+1 
12  x^4-x^2+1 
\end{verbatim}

\subsection*{(ii)}

Denoting $\zeta_n$ to be the primitive nth root of unity we define $$\Phi_n(x) = \prod_{\gcd{m,n}=1} (x-\zeta_n^m)$$ and show that
$$\Phi_n(x) = \frac {x^n-1} {\lcm{x^m-1\Big | m|n,m\neq n}}$$  where $x^n-1 = \prod_{m=1}^n (x-\zeta_n^m) $

\begin{description}
\state{Claim}
$$ \lcm{x^m-1 \Big | m|n,m\neq n} = \prod_{\gcd{m,n} \neq 1} (x-\zeta_n^m)$$

\point{Since I am lazy I will write $\lcm{-}$ and $\prod (-)$ instead.}

\state{Proof of claim}
Consider any $m$ such that $m|n$, $m \neq n  \implies \gcd{m,n} \neq 1$ it follows
\begin{align*}
x^m-1 	{}& = (x-\zeta_{n/m}) \cdots (x- \zeta^k_{n/m}) \cdots (x-\zeta_{n/m}^{n/m}) \\
		{}& = (x-\zeta_n^m) \cdots (x-\zeta_n^{m \cdot k}) \cdots (x-\zeta_n^{n})
\end{align*}

For any $k$ we have  $\gcd{m\cdot k , n} \neq 1 \implies (x-\zeta_n^{m\cdot k}) \big | \prod (-)$ and it follows that for any $m$ where $m|n$ and $m \neq n$, $$\label{P1} (x^m-1) \big | \prod(-) \implies \lcm{-} \big | \prod(-).$$

For any $m$ such that $\gcd{m,n} = g \neq 1$ we have $\bar m$ such that $g \cdot \bar m = m$. Giving $(x-\zeta_n^m)=(x-\zeta_n^{g\cdot \bar m})=(x-\zeta_{n/g}^{\bar m}) \big | (x^{n/g}-1)$. Since  $\frac n g \big | n \implies (x^{n/g}-1) \big | \lcm{-} \implies (x-\zeta_n^m) \big | \lcm{-}$ it follows that $$\prod (-) \Big |  \lcm{-}. $$

Collectively these two conclusions imply the claim.

\state{Proof of main result}
$$ \frac {x^n-1} {\lcm{x^m-1\Big | m|n,m\neq n}} = \frac {\prod_{m=1}^n (x-\zeta_n^m)} {\prod_{\gcd{m,n} \neq 1} (x-\zeta_n^m)} = \prod_{\gcd{m,n}=1} (x-\zeta_n^m) = \Phi_n(x)$$
$\hfill \Box$

\state{Justification of algorithm}
The algorithm that implements this result uses $\lcm{x^{n/p}-1 \Big | p \text{ is a prime divisor of } n}$ instead. We observe for any $m$ such that $m|n$, $m \neq n$ that
$$(x^m-1) = \prod_{\bar k=1}^m (x-\zeta_m^{\bar k})=\prod_{\bar k =1}^m (x-\zeta_n^{\frac {\bar k n} {m}})$$ 

where there is some prime divisor $p$ of $n$ such that $1 \leq \frac {\bar k n} {m p} \leq \frac {mn} {mp} = \frac {n} {p}$ and choosing $k= \frac {\bar k n} {m p}$ we have that $(x-\zeta_n^{\frac {\bar k n} {m}})=(x-\zeta_n^{kp})=(x-\zeta_{n/p}^k)$.

This means $\forall m \brac{m|n,m\neq n}$, $\forall \bar k \brac{1 \leq \bar k \leq m}$, $\exists p$, $p$ a prime divisor of $n$ such that 
$(x-\zeta_n^{\frac {\bar k n} {m}}) \Big | \prod_{k=1}^{n/p} (x-\zeta_n^{k p})$. So it suffices to only use prime divisors of $n$.

\end{description}

\subsection*{Code}
\begin{verbatim}
getCyc:=proc(n)
local ds, ps, L:
    
    if n=1 then
        x-1:
    elif isprime(n) then
        add(x^i,i=0..n-1);
    else
        ps:=select(`isprime`,numtheory[divisors](n));
        ds:=seq(n/p, p in ps):
        L:=lcm(seq(x^d-1,d in ds)):
        quo(x^n-1,L,x);
    end if:
    
end proc:
\end{verbatim}

\subsection*{Output}
\begin{verbatim}
> f:=0: i:=0:
> while norm(f,infinity)<3 do
>     i:=i+1:
>     f:=getCyc(i):
> end do:

> i;
                                        385

> quit:
\end{verbatim}

\newpage
%QUESTION 3
\section*{Question 3: Solving Linear Systems over Number Fields}

\subsection*{Code}
\begin{verbatim}
with(LinearAlgebra):

GE:=proc(A,b,m)
local n,B,L,k,i,j,piv:
#   cleaning up
    n:=(Dimensions(A))[1]:
    B:=<A|b>:
    B:=map(rem,B,m,e);
    L:=ilcm(seq(seq(denom(B[i,j]), i=1..n), j=1..n+1)):
    B:=B*L:
    
    for k to n do
        i:=k:
        
        while i<=n and B[i,k]=0 do
            i:=i+1:
        end do:
        
        if i>n then 
            error "A is singular":
        end if:
        
#       interchange rows
        for j from k to n+1 do
            B[i,j],B[k,j]:=B[k,j],B[i,j]:
        end do:
        
        gcdex(B[k,k],m,e,'piv'):
        
        for j from k to n+1 do
            B[k,j]:=rem(piv*B[k,j],m,e):
        end do:
        
        for i from k+1 to n do
            for j from k+1 to n+1 do
                B[i,j]:=rem(B[i,j]-B[i,k]*B[k,j],m,e):
            end do:
            B[i,k]:=0:
        end do:
    end do:
    
    for k from n by -1 to 1 do 
        for j from k-1 by -1 to 1 do
            B[j,n+1]:=rem(B[j,n+1]-B[j,k]*B[k,n+1],m,e):
            B[j,k]:=0;
        end do:
    end do:
        
    return Vector([seq(B[k,n+1],k=1..n)]):
end proc:
\end{verbatim}

\subsection*{Output}
\begin{verbatim}
> read "GE.mpl":
> 
> read "systems/sys49.txt":
> Dimensions(A)[1];
                                         49

> st:=time():
> x:=GE(A,b,M):
> time()-st;
                                       29.932

> convert(map(rem,A.x-b,M,e),set);
                                        {0}

> 
> read "systems/sys100.txt":
> Dimensions(A)[1];
                                        100

> st:=time():
> x:=GE(A,b,M):
> time()-st;
                                      672.371

> convert(map(rem,A.x-b,M,e),set);
                                        {0}

> 
> read "systems/sys196.txt":
> Dimensions(A)[1];
                                        196

> st:=time():
> x:=GE(A,b,M):
> time()-st;
                                      1994.514

> convert(map(rem,A.x-b,M,e),set);

                                        {0}
\end{verbatim}

\newpage
\section*{Question 4: A Modular Algorithm}

\subsection*{Code}
\begin{verbatim}
MGE:=proc(inA,inb,m,k)
local n,A,L,b,p,P,x,badprime,betas,xpbs,beta,Apb,bpb,xpb,xp,i,xx,pass:
    
    n:=LinearAlgebra[Dimension](inb):
    
    #	cleaning up
    A:=map(rem,inA,m,e):
    L:=ilcm(seq(seq(denom(A[i,j]),i=1..n),j=1..n)):
    A:=A*L:
    b:=inb*L:
    
    #	p:=2^30:
    p:=2^60:
    P:=1:	
    x:=Vector(1..n,0):
    
    while true do
    
        p:=nextprime(p):
        
        while not(1 = p mod k) do
            p:=nextprime(p):
        end do:			
        
        printf("p=%d \n", p);
        
        badprime:=false:
        
        betas:=map(x->x[1],Roots(m) mod p):
        
        xpbs:=NULL:	
        
        for beta in betas do
            Apb:=eval(A,e=beta) mod p:
            bpb:=eval(b,e=beta) mod p:
            xpb:=Linsolve(Apb,bpb) mod p:
            
            #a bad prime/eval point is one that makes A go singluar.
            if type(xpb,function) then
                badprime:=true:
                break:
            end if:
            
            xpbs:=xpbs,xpb:
        end do:
        
        if not(badprime) then
            xp:=Vector(1..n,0):
        
            for i from 1 to n do
                xp[i]:=Interp(betas,[seq(v[i],v in xpbs)],e) mod p: 
            end do:
            
            x:=chrem([x,xp],[P,p]):
            P:=P*p:
            
            xx:=Vector(1..n,0):
            pass:=true:
            
            for i from 1 to n do
                xx[i]:=iratrecon(x[i],P):
                
                if xx[i]=FAIL then
                    pass:=false:
                    break:
                end if:
            end do:
            
            if pass and convert(map(rem,A.xx-b,m,e),set)={0} then
                return xx:
            end if:
            
        end if:
        
    end do:
    
end proc:

\end{verbatim}

\subsection*{Output}
\begin{verbatim}
> read "MGE.mpl":
 
> read "systems/sys49.txt":
> k:=5:
> st:=time():
> x:=MGE(A,b,M,k):
p=1152921504606847081 

> time()-st;
                                       2.572

> c:=map(rem,(A.x-b),M,e):
> convert(c,set);
                                        {0}

> read "systems/sys100.txt":
> k:=24:
> st:=time():
> x:=MGE(A,b,M,k):
p=1152921504606847009 

> time()-st;
                                       31.929

> c:=map(rem,(A.x-b),M,e):
> convert(c,set);
                                        {0}

> read "systems/sys196.txt":
> k:=3:
> st:=time():
> x:=MGE(A,b,M,k):
p=1152921504606847009 
p=1152921504606847081 
p=1152921504606847123 
p=1152921504606847189 

> time()-st;
                                      326.183

> c:=map(rem,(A.x-b),M,e):
> convert(c,set);
                                        {0}

> quit:
\end{verbatim}


\end{document}