\documentclass[12pt]{report}

\usepackage{rotating,graphics,psfrag,epsfig,amssymb,amsmath, subfigure, amsxtra, amsthm, color}
\usepackage{fancyvrb}
\oddsidemargin=-0.25in
\topmargin= -0.5in
\textwidth=7.0in
\textheight=9in

\usepackage{listings}
\lstdefinelanguage{maple}
	{morekeywords={true, false, try, catch, return, break, error, 
	               module, export, local, option, in, use,
                 and, or, not, xor, xnor,
                 if, then, elif, else, fi,
                 while, for, from, by, to, do, od,
                 proc, nargs, local, global, end, NULL}}
\lstset{language=maple,numbers=left,basicstyle=\footnotesize,numberstyle=\tiny, breakatwhitespace=false,frame=single,morecomment=[l]{\#},stringstyle=\ttfamily}

\usepackage[usenames,dvipsnames]{xcolor}
\definecolor{MapleRed}{rgb}{1,0,0}
\definecolor{MapleBlue}{rgb}{0,0,1}
\definecolor{MaplePink}{rgb}{1,0,1}

%\definecolor{MapleRed}{rgb}{0,0,0}
%\definecolor{MapleBlue}{rgb}{0,0,0}

\renewcommand{\arraystretch}{1.25}

%\pagestyle{headings}

%\sloppy

\theoremstyle{plain}
\newtheorem{theorem}{Theorem}
\newtheorem{corollary}{Corollary}
\newtheorem{lemma}{Lemma}
\newtheorem{proposition}{Proposition}
\newtheorem{conjecture}{Conjecture}

\theoremstyle{definition}
\newtheorem{definition}{Definition}
\newtheorem{example}{Example}
\newtheorem{remark}{Remark}
\newtheorem{question}{Question}
\newtheorem{problem}{Problem}


\renewcommand{\implies}{\Rightarrow}
\newcommand{\imp}{\textbf}
\newcommand{\dom}[1]{\textsf{dom} (#1)}
\newcommand{\cod}[1]{\textsf{cod}(#1)}
\newcommand{\arr}[1]{\mathcal{#1}\mathrm{-arrows}}
\newcommand{\obj}[1]{\mathcal{#1}\mathrm{-objects}}
\newcommand{\state}[1]{\item[#1] \textcolor{white}{.} \\ \textcolor{white}{.} \\ }
\newcommand{\Lstate}[1]{\item[#1] \nl}
\newcommand{\id}[1]{id_{#1}}
\newcommand{\nl}{\textcolor{white}{.}\ }
\newcommand{\func}[3]{#1 : #2 \rightarrow #3}
\newcommand{\CC}{\mathbb{C}}
\newcommand{\DD}{\mathcal{D}}
\newcommand{\point}[1]{\begin{itemize} \item[$\cdot$] #1 \end{itemize}}
\newcommand{\blank}{\textcolor{white}{-}}
\newcommand{\fieldt}{\mathbb{Z}_q[t]}
\newcommand{\fieldtx}{\mathbb{Z}_q[t][x]}
\newcommand{\impl}{\Rightarrow}
\newcommand{\eps}{\varepsilon}
\newcommand{\brac}[1]{\left( #1 \right)}
\newcommand{\Res}[1]{\textrm{Res}_{#1} }

\newcommand{\x}{\boldsymbol{x}}
\renewcommand{\H}{\boldsymbol{H}}
\newcommand{\zero}{\boldsymbol{0}}
\newcommand{\z}{\boldsymbol{z}}
\newcommand{\F}{\boldsymbol{F}}
\newcommand{\p}{\boldsymbol{p}}
\newcommand{\q}{\boldsymbol{q}}
\newcommand{\alp}{{\boldsymbol{\alpha}}}
\newcommand{\bet}{{\boldsymbol{\beta}}}

\newcommand{\V}{\mathbf{V}}
\newcommand{\W}{\mathbf{W}}
\newcommand{\M}{\mathbf{M}}
\newcommand{\A}{\mathbf{A}}
\newcommand{\B}{\mathbf{B}}
\newcommand{\C}{\mathbf{C}}
\newcommand{\X}{\mathbf{X}}
\newcommand{\0}{\mathbf{0}}
\newcommand{\II}{\mathbf{I}}
\newcommand{\ZZ}{\mathbb{Z}}
\newcommand{\QQ}{\mathbb{Q}}

\newcommand{\eval}[2]{ \left.#1\right |_{#2}}
\newcommand{\xinit}{x_{\text{init}}}
\newcommand{\idnty}{\text{Id}}

\newcommand{\pdiff}[2]{\frac{\partial #1}{\partial #2}}

\newcommand{\FF}{\mathbf{F}}
\newcommand{\JJ}{\mathbf{J}}
\newcommand{\XX}{\mathbf{X}}

\newcommand{\Xinit}{\XX_{\text{init}}}
\newcommand{\Ginit}{G_{\text{init}}}

\renewcommand{\mod}{\text{ mod }}
\newcommand{\Mod}{\hspace{2mm}\mod}

\newcommand{\rem}{\text{ rem }}
\newcommand{\quo}{\text{ quo }}
\renewcommand{\div}{\text{ div }}
\newcommand{\MM}{\textsf{M}}

\def\MapleInput#1{\noindent{\tt{[}\color{MapleRed}{\tt{>}\tt{#1}}}}
\def\MapleOutput#1{{\begin{center} \emph{\color{MapleBlue}{#1}} \end{center}}}
\def\MapleWarning#1{\noindent{{\small {\tt \color{MaplePink}{#1} }}}}


\DefineVerbatimEnvironment% 
{MapleInputs}{Verbatim}{formatcom=\color{MapleRed}}

\renewcommand{\FancyVerbFormatLine}[1]{\makebox[0.43cm][l]{\color{black}{[}\color{MapleRed}{>}}#1}

\newcommand{\PutEnd}{\begin{center}
\vspace{\fill}\ \newline
{\tiny \rm $ $Typeset by: Paul Vrbik$ $ }
{\tiny \rm $ $Compiled on \today$ $ }
\end{center}
}


\usepackage{algorithm}
\usepackage{algorithmic}
\usepackage{subfigure}
\usepackage{tikz}

\usepackage{listings}
\lstset{language=Java}

\newcommand{\Ans}{\hfill {\Huge\begin{tabular}{|c|} \hline \hspace{2.5cm} \hfill \\ \hline \end{tabular}} }

\newcommand{\AnsBox}[1]{
\hspace{-1cm}\fbox{
\begin{minipage}[b][#1][c]{17.6cm}
\hfill
\end{minipage}}
}

\begin{document}

%%%TITLE PAGE%%%
{\large
\begin{center}
{\it UNIVERSITY OF WESTERN ONTARIO}
\end{center}
\begin{center}
\bf{Computer Science 1026b, Spring 2010}\\
\large{\bf{Computer Science Fundamentals I}}\\
{\large{\it MIDTERM TEST }}\\
9:30am - 11:20am, Thursday, March 4, 2010.
\end{center}

\begin{center}
\begin{tabular}{ r c}
Name (last, first): & \rule{2in}{0.01in} \\
Student Number: & \rule{2in}{0.01in}
\end{tabular}
\end{center}

{\normalsize
\vspace{20pt}
\begin{center}\textbf{Instructions}\end{center}
\begin{enumerate}
\item TURN OFF YOUR CELL PHONE.
\item Do not turn this page until instructed to do so.
\item Cheating (e.g. copying someone else's solutions, using prohibited material like your textbook and notes, 
or using electrical communication devices) will result in a zero grade and possible expulsion from the course.
\item We reserve the right to orally examine you regarding your solutions and then regrade accordingly.
\item Provide your solution in the box if a box is provided. All writing outside boxes will be ignored and as this is the case feel free to use the back of pages for rough work.
\item Only those exams written in pen are eligible for regrading.
%\item It is your responsibility to ensure that this exam booklet has \emph{six pages}.
\item Correct java code is preferable but pseudo-code (i.e. something resembling code) is 
acceptable and if appropriate will be accepted for partial marks.
\end{enumerate}
\vspace{30pt}
}

\begin{center}
{\Large
\begin{tabular}{||c|c|r||} \hline 
 1 & 10 &\hspace{10mm} \hfill \\ \hline 
 2 & 20 &       \\ \hline
 3 & 10 &      \\ \hline
4 &  10 &    \\ \hline
Total & 50 & \\ \hline 
\end{tabular}
}
\end{center}
}
\clearpage
%%%END TITLE PAGE%%%

\section*{Single Answer}
Each answer in this section is worth \emph{1 point} for a total of \emph{10 points}.
\subsection*{Question 1}
\emph{The following questions contain valid Java code and will provide output (without error) if entered into 
the DrJava interactions pane.} 


\renewcommand{\labelenumi}{\alph{enumi}.}
\begin{enumerate}
\item Suppose we have defined
\begin{verbatim}
int n = 4;
double x = 2.5;
\end{verbatim}
then the value of the expression \texttt{5 * x - n / 5} is: \Ans

\item Suppose we have defined
\begin{verbatim}
int num1 = 50, num2 = 50;
\end{verbatim}
then the value of the expression \texttt{(num1 == num2)} is: \Ans

\item After the following code segment is executed
\begin{verbatim}
int x = 3;
x = x + x;
x = x + x;
\end{verbatim}
the value of \texttt{x} is: \Ans

\item After the following code segment is executed
\begin{verbatim}
int x = 3, y = 5;
x = y;
y = x;
\end{verbatim}
the value \texttt{y} is: \Ans

\clearpage
\item If we declare the array
\begin{verbatim}
int [] intArr = {2,4,6,8,10};
\end{verbatim}
then the value stored in \texttt{intArr[1]} is: \Ans

\item After the following code segment is executed
\begin{verbatim}
int sum = 0;
int count = 1;
while (count < 5) {
    sum = sum + count;
    count = count + 2;
}
\end{verbatim}
the value of \texttt{count} is: \Ans

\item After the following code segment is executed
\begin{verbatim}
bool x = (true && false) || (true && true) || false
\end{verbatim}
the value of \texttt{x} is: \Ans

\item After the following code segment is executed
\begin{verbatim}
int x = (3+3*2) % 6
\end{verbatim}
the value of \texttt{x} is: \Ans

\item After the following code segment is executed
\begin{verbatim}
int [] xs = { 1, 2, 3, 4, 5, 6, 7 };
System.out.println(xs[xs.length]);
\end{verbatim}
what is printed? (Your solution doesn't have to be exact). \Ans

\item After the following code segment is executed
\begin{verbatim}
int x = (int) 2.3 > 2;
\end{verbatim}
the value of \texttt{x} is: \Ans

\end{enumerate}

%%%SHORT ANSWER%%%
\clearpage
\section*{Short Answer}
Each solution in this section is worth \emph{4 points} for a total of \emph{20 points}.
\subsection*{Question 2}
\begin{enumerate}
\item What will be the output of the following code segment?
\begin{verbatim}
for (int row = 1; row <=3; row ++) {
    for (int count = 1; count <= (4 - row); count++) {
        System.out.print("*");
    }
    System.out.println();
}
\end{verbatim}
\AnsBox{5cm}
\item What happens when the following code segment is executed? Why?
\begin{verbatim}
int sum = 0;
int count = 5;
while (count > 1) {
    sum = sum + count;
    count = count + 2;
}
System.out.println(sum);
\end{verbatim}
\AnsBox{5cm}
\clearpage

\item In the space provided write a \texttt{for} loop that is equivalent to the following \texttt{while} loop.
\begin{verbatim}
int sum = 0;
int i = 3;
while (i < 100) {
    sum = sum + i;
    i = i + 3;
}
\end{verbatim}
\AnsBox{8cm}


\item Suppose \texttt{Yertle} and \texttt{Franklin} are two Turtle objects in the same World. Provide a code segment that makes the two turtles face each other.

\AnsBox{8cm}

\item Write a code segment that creates a 400px by 300px (width by height) Picture object with all pixels set to red. Assume that this code will be part of the main program and \emph{not} in the Picture class.

\AnsBox{12cm}

\end{enumerate}

\clearpage
\section*{Long Answer}
Each solution in this section is worth \emph{10 points} for a total of \emph{20 points}.
\subsection*{Question 3}
\begin{enumerate}
\item Write an algorithm in pseudo-code (i.e. code mixed with english) which prints the following pattern when given \texttt{n}. Your method should work for any value of $n$ and not only on the examples given below.

\begin{center}
\begin{tabular}{l  c l c l }
$n=2$ 		&\hspace{50px}\hfill	& $n=3$ 		&\hspace{50px}\hfill& $n=4$ \\
\hline
\texttt{xx}		&&\texttt{xxx}	&& \texttt{xxxx} \\
\texttt{xo}		&&\texttt{xxo}	&& \texttt{xxxo} \\
\texttt{oo}		&&\texttt{xoo}	&& \texttt{xxoo} \\
			&&\texttt{ooo}	&& \texttt{xooo} \\
			&&			&& \texttt{oooo} \\
\hline
\end{tabular}
\end{center}

\AnsBox{13cm}

\item Write a java class method \texttt{static void PrintPattern (int n) } that prints the same pattern as in Part a.

\AnsBox{14cm}

\end{enumerate}

\clearpage
\subsection*{Question 4}
Write a java object method \texttt{void nSidedPolygoon (int n, int s)} for inclusion in the Turtle class that takes as input $n$ and $s$ and draws an $n$-sided polygon with side length $s$. You may assume your turtle never goes out of bounds. The ``exterior angle'' will be ${360^{\circ}}/{n}$ as illustrated for $n=3$ and $n=6$ in the figure below. \\
\begin{center}
% !TEX root = OurMid.tex
\begin{tikzpicture}
% The graphic
%\draw[style=help lines,step=1cm] (0,0.5) grid (16,5.5);

\draw (1,1) -- (3,4);
\draw (3,4) -- (5,1);
\draw (5,1) -- (1,1);

\draw (5.5,1) -- (6,1);
\draw (5,1) -- (5.5,1) arc(0:120:5mm);
\draw (5,2) node[right=-15pt] {$\frac{360^{\circ}}{3} = 120^{\circ}$};

%\draw (3,1) node[below] {$s$};
\draw (2,2.75) node[left] {$s$};
%\draw (4.5,2.75) node[left] {$s$};

%2.23606798
\draw (10,1) -- (12.23606798,1);
\draw (8.76393202,3) -- (10,1);
\draw (13.23606798, 3) -- (12.23606798,1);
\draw (13.23606798, 3) -- (12.23606798,5);
\draw (10,5) -- (8.76393202,3);
\draw (10,5) -- (12.23606798,5);

\draw (12.23606798,1) -- (13.5,1) arc(0:60:13.4mm);
\draw (13.5,1) -- (14,1);
\draw (14,2) node[right=-25pt] {$\frac{360^{\circ}}{6} = 60^{\circ}$};
\draw (9,4) node {$s$};


\end{tikzpicture}
\\
\end{center}

\AnsBox{14cm}

\clearpage
\begin{center}
\textbf{Rough Work}\\
\emph{Nothing on this page will be marked. If this page is removed your exam will not be graded.}
\end{center}

\end{document}