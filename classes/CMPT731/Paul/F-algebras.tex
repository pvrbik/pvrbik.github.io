\section*{F-Algebras}

Let \CC \, be a category and \tm{F: {\CC} \rightarrow {\CC}} \, be
an endofunctor.

An \definition{F-algebra} is a pair \tm{(A,a)} where \tm{A} is a
\CC-\tm{\unmath{object}} and \tm{a: F \, A \rightarrow A} is a
\CC-\tm{\unmath{arrow}}.  The object \tm{A} is called the
\definition{carrier} of the F-algebra.

\newslide

For example, let \tmi{Natural} be the set of natural numbers in the
category
\tmi{Set}.  We will now define an F-algebra on the carrier \tmi{Natural}
with two ``operators''.  The first operator is really a constant,
which can
be viewed as an operator that takes zero arguments.  This operator,
which we
will call \tmi{zero}, ``returns'' the natural number \tm{0}.  The
other
operator is a binary operator, \tmi{plus}, that returns the sum of
its two
arguments.  This gives us a group with an identity element and a
binary
operation.

This example could easily be extended to include more constants and
operators, such as the constant \tmi{one} and the binary operator
\tmi{times}, as well up operators of different arity, such as a
unary
operator, \tmi{negate}, and operators of arity three or more.

\newslide

Let \tm{F: F \, \tmi{Set} \rightarrow \tmi{Set}} be the functor
defined by \tm{F_{0} \, S = 1 + S \times S} and
\tm{F_{1} \, f = [\tmi{id}_{1},<f,f>]}.

We now define the arrow \tm{a: F \, A \rightarrow A} as follows.

\begin{codenott}
\tm{a (i_{1} (x)) = 0}\\
\tm{a (i_{2} (a,b)) = a + b}
\end{codenott}

\newslide

Let us revisit this example in \Has.

We will need to assume that we have a \textit{Haskell} type
\tcx{Natural}
that contains the natural numbers and nothing else.  This is not
actually
possible in \textit{Haskell}, because every type contains a $\bottom$
value,
but we can pretend.  Similarly, for a later example, we will assume
that
\textit{Haskell} contains a type \tcx{PosReal}, which includes all
positive
real numbers (so it has an uncountable number of values), does not
contain
$\bot$.  We will distinguish that fake types having no $\bottom$
element by
using a \tcx{special color (PineGreen)}.

\newslide

We start by defining the functor (type constructor) \tcx{F}.

\begin{code}
datatype \tcx{F} a = Ident \tcx{|} Op a a\\
\\
instance Functor \tcx{F} where\\
   fmap f Ident = Ident\\
   fmap f (Op x y) = Op (f x) (f y)
\end{code}

As the color suggests, we assume that no $\bottom$ element is added
by the
type constructor \tcx{F}\@.  We also create labels, \tc{Zero} and
\tc{Add}
for the two cases in the domain sum.

\newslide

We now define

\begin{code}
a :: \tcx{F Natural} -> \tcx{Natural}\\
a Ident = 0\\
a (Op x y) = x + y
\end{code}

Notice how a single function, \tc{a}, defined both (or in general
case, all) operations at once.

\newslide

Using the same functor, \tcx{F}, we also can define an F-algebra
over the
positive real numbers with a constant \tmi{one} and an operation
\tmi{times}.

\begin{code}
a :: \tcx{F PosReal} -> \tcx{PosReal}\\
a Ident = 1\\
a (Op x y) = x * y
\end{code}

\newslide

\begin{cdiag}
\obj(1,1)[B]{\tcx{F \, Real}}
\obj(1,11)[A]{\tcx{F \, Natural}}
\obj(21,1)[D]{\tcx{Real}}
\obj(21,11)[C]{\tcx{Natural}}
\mor{A}{B}{\tc{fmap h}}
\mor{C}{D}{\tc{h}}
\mor{A}{C}{\tc{a}}
\mor{B}{D}{\tc{b}}
\end{cdiag}
