\documentclass[12pt]{article}
\usepackage{fancyvrb,enumerate}
\usepackage{fancyvrb}
\usepackage{vaucanson-g}

\usepackage{rotating,graphics,psfrag,epsfig,amssymb,amsmath, subfigure, amsxtra, amsthm, color}
\usepackage{fancyvrb}
\oddsidemargin=-0.25in
\topmargin= -0.5in
\textwidth=7.0in
\textheight=9in

\usepackage{listings}
\lstdefinelanguage{maple}
	{morekeywords={true, false, try, catch, return, break, error, 
	               module, export, local, option, in, use,
                 and, or, not, xor, xnor,
                 if, then, elif, else, fi,
                 while, for, from, by, to, do, od,
                 proc, nargs, local, global, end, NULL}}
\lstset{language=maple,numbers=left,basicstyle=\footnotesize,numberstyle=\tiny, breakatwhitespace=false,frame=single,morecomment=[l]{\#},stringstyle=\ttfamily}

\usepackage[usenames,dvipsnames]{xcolor}
\definecolor{MapleRed}{rgb}{1,0,0}
\definecolor{MapleBlue}{rgb}{0,0,1}
\definecolor{MaplePink}{rgb}{1,0,1}

%\definecolor{MapleRed}{rgb}{0,0,0}
%\definecolor{MapleBlue}{rgb}{0,0,0}

\renewcommand{\arraystretch}{1.25}

%\pagestyle{headings}

%\sloppy

\theoremstyle{plain}
\newtheorem{theorem}{Theorem}
\newtheorem{corollary}{Corollary}
\newtheorem{lemma}{Lemma}
\newtheorem{proposition}{Proposition}
\newtheorem{conjecture}{Conjecture}

\theoremstyle{definition}
\newtheorem{definition}{Definition}
\newtheorem{example}{Example}
\newtheorem{remark}{Remark}
\newtheorem{question}{Question}
\newtheorem{problem}{Problem}


\renewcommand{\implies}{\Rightarrow}
\newcommand{\imp}{\textbf}
\newcommand{\dom}[1]{\textsf{dom} (#1)}
\newcommand{\cod}[1]{\textsf{cod}(#1)}
\newcommand{\arr}[1]{\mathcal{#1}\mathrm{-arrows}}
\newcommand{\obj}[1]{\mathcal{#1}\mathrm{-objects}}
\newcommand{\state}[1]{\item[#1] \textcolor{white}{.} \\ \textcolor{white}{.} \\ }
\newcommand{\Lstate}[1]{\item[#1] \nl}
\newcommand{\id}[1]{id_{#1}}
\newcommand{\nl}{\textcolor{white}{.}\ }
\newcommand{\func}[3]{#1 : #2 \rightarrow #3}
\newcommand{\CC}{\mathbb{C}}
\newcommand{\DD}{\mathcal{D}}
\newcommand{\point}[1]{\begin{itemize} \item[$\cdot$] #1 \end{itemize}}
\newcommand{\blank}{\textcolor{white}{-}}
\newcommand{\fieldt}{\mathbb{Z}_q[t]}
\newcommand{\fieldtx}{\mathbb{Z}_q[t][x]}
\newcommand{\impl}{\Rightarrow}
\newcommand{\eps}{\varepsilon}
\newcommand{\brac}[1]{\left( #1 \right)}
\newcommand{\Res}[1]{\textrm{Res}_{#1} }

\newcommand{\x}{\boldsymbol{x}}
\renewcommand{\H}{\boldsymbol{H}}
\newcommand{\zero}{\boldsymbol{0}}
\newcommand{\z}{\boldsymbol{z}}
\newcommand{\F}{\boldsymbol{F}}
\newcommand{\p}{\boldsymbol{p}}
\newcommand{\q}{\boldsymbol{q}}
\newcommand{\alp}{{\boldsymbol{\alpha}}}
\newcommand{\bet}{{\boldsymbol{\beta}}}

\newcommand{\V}{\mathbf{V}}
\newcommand{\W}{\mathbf{W}}
\newcommand{\M}{\mathbf{M}}
\newcommand{\A}{\mathbf{A}}
\newcommand{\B}{\mathbf{B}}
\newcommand{\C}{\mathbf{C}}
\newcommand{\X}{\mathbf{X}}
\newcommand{\0}{\mathbf{0}}
\newcommand{\II}{\mathbf{I}}
\newcommand{\ZZ}{\mathbb{Z}}
\newcommand{\QQ}{\mathbb{Q}}

\newcommand{\eval}[2]{ \left.#1\right |_{#2}}
\newcommand{\xinit}{x_{\text{init}}}
\newcommand{\idnty}{\text{Id}}

\newcommand{\pdiff}[2]{\frac{\partial #1}{\partial #2}}

\newcommand{\FF}{\mathbf{F}}
\newcommand{\JJ}{\mathbf{J}}
\newcommand{\XX}{\mathbf{X}}

\newcommand{\Xinit}{\XX_{\text{init}}}
\newcommand{\Ginit}{G_{\text{init}}}

\renewcommand{\mod}{\text{ mod }}
\newcommand{\Mod}{\hspace{2mm}\mod}

\newcommand{\rem}{\text{ rem }}
\newcommand{\quo}{\text{ quo }}
\renewcommand{\div}{\text{ div }}
\newcommand{\MM}{\textsf{M}}

\def\MapleInput#1{\noindent{\tt{[}\color{MapleRed}{\tt{>}\tt{#1}}}}
\def\MapleOutput#1{{\begin{center} \emph{\color{MapleBlue}{#1}} \end{center}}}
\def\MapleWarning#1{\noindent{{\small {\tt \color{MaplePink}{#1} }}}}


\DefineVerbatimEnvironment% 
{MapleInputs}{Verbatim}{formatcom=\color{MapleRed}}

\renewcommand{\FancyVerbFormatLine}[1]{\makebox[0.43cm][l]{\color{black}{[}\color{MapleRed}{>}}#1}

\newcommand{\PutEnd}{\begin{center}
\vspace{\fill}\ \newline
{\tiny \rm $ $Typeset by: Paul Vrbik$ $ }
{\tiny \rm $ $Compiled on \today$ $ }
\end{center}
}


\newcommand{\powerset}{\mathcal{P}}
\renewcommand{\iff}{\Leftrightarrow}

\title{CS 3331a - Assignment 2 - Solutions}
\author{Paul Vrbik}
\date{October 19, 2009}

\begin{document}

%FOR GRAPHS
\input{VCPref-mystyle} 
\ShowFrame %\ShowGrid 
%END FOR GRAPHS

\maketitle
\subsection*{Question 1- $18$ marks}
Let the language $L = \{ a^ib^i | i \geq 0 \}$.
\begin{enumerate}
\item[(a)] $L_1 = L^R$ is the language consisting of the ``reversals'' of the words in $L$ (or $L$'s transposes). That is, 
$$L_1 = \{ b^ia^i | i> 1 \}.$$
\item[(b)] $L_2 = L^2$ is the language consisting of words that are the concatenation of two words from $L$. That is, 
$$L_2 = \{ a^ib^ia^jb^j | i,j \geq 1 \}.$$
\item[(c)] $L_3 = L^*$ are those words that you can construct by concatenating as many (including zero) words from $L$. That is
$$L_3 = \{ a^{i_1}b^{i_1}a^{i_2}b^{i_2}\cdots a^{i_k}b^{i_k}|i_1,\ldots ,i_k\geq 1,k\geq 0\}$$
\end{enumerate}

\subsection*{Question 2 - $18$ marks}
\begin{enumerate}
\item[(a)] The set of all words that have $aab$ as a subword.
\begin{center}
\begin{VCPicture}{(-1,-2)(8,1)}
% states
\SmallState
\State{(0,0)}{A}\State{(2,0)}{B}\State{(4,0)}{C}\FinalState{(6,0)}{D}
% initial--final
\Initial{A}
% transitions 
\EdgeR{A}{B}{a}
\EdgeR{B}{C}{a}
\EdgeR{C}{D}{b}
\ArcR{B}{A}{b}
\LoopE{D}{a,b}
\LoopS{C}{a}
\LoopS{A}{b}
%
\end{VCPicture}
\end{center}
\item[(b)] The set of words that have $aab$ as a scattered subword (also called a subsequence).
\begin{center}
\begin{VCPicture}{(-1,-2)(8,1)}
% states
\SmallState
\State{(0,0)}{A}\State{(2,0)}{B}\State{(4,0)}{C}\FinalState{(6,0)}{D}
% initial--final
\Initial{A}
% transitions 
\EdgeR{A}{B}{a}
\EdgeR{B}{C}{a}
\EdgeR{C}{D}{b}
\LoopS{B}{b}
\LoopE{D}{a,b}
\LoopS{C}{a}
\LoopS{A}{b}
%
\end{VCPicture}
\end{center}

\item[(c)] The set of all words such that the third symbol from the right-end is an $a$.

\small\emph{There is a strategy which may make this question easier: create states for all possible length three substrings (there are eight of them) and draw rules for transitioning between them. The accepting states are those substrings that begin with `$a$'. (I've labelled the states below to reflect this.) You may interpret the start state as being `$bbb$' because you will at need at least three moves to get to an accepting state from there.}
\begin{center}
\begin{VCPicture}{(-5,-5)(5,6)}
% states
\State[bba]{(-4,0)}{bba}\FinalState[aba]{(4,0)}{aba}\FinalState[aaa]{(0,4)}{aaa}\FinalState[abb]{(0,-4)}{abb}
\State[baa]{(-3,3)}{baa}\FinalState[aab]{(3,3)}{aab}\State[bab]{(3,-3)}{bab}\State[bbb]{(-3,-3)}{bbb}
% initial--final
\Initial{bbb}
\EdgeBorder
\EdgeL{aaa}{aab}{b}
\EdgeL{aab}{aba}{a}
\ArcL{aba}{bab}{b}
\ArcL{bab}{aba}{a}
\EdgeL{bab}{abb}{b}
\EdgeL{abb}{bbb}{b}
\EdgeL{bbb}{bba}{a}
\EdgeL{bba}{baa}{a}
\EdgeL{baa}{aaa}{a}
\LoopN{aaa}{a}
\LoopS{bbb}{b}

\EdgeR{baa}{aab}{b}
\EdgeR[0.1]{aab}{abb}{b}
\EdgeL[0.1]{aba}{baa}{a}
\EdgeL[0.1]{bba}{bab}{b}
\EdgeR[0.15]{abb}{bba}{a}

\EdgeBorderOff
% transitions 
%
\end{VCPicture}
\end{center}
\end{enumerate}

\subsection*{Question 3 - $15$ marks}
Nonnegative integers divisible by eight.\\

\noindent\small\emph{Explanation (not required for full credit):} \\
This DFA describes all bit sequences that end in at least three zeros ( $(01)^* \circ 000$ ). Converting this binary number into an integer would be done like this: $$0\times2^0 + 0\times 2^1 + 0 \times 2^2 + s_3 \times 2^3 + s_4 \times 2^4 + \cdots$$ where $s_i \in \{0,1\}$. This number can be re-written as:
$$2^3 \brac{s_3 \times 2^0 + s_4 \times 2^1 + \cdots}$$ 
which are precisely those \textbf{nonnegative integers divisible by eight} ($8\ZZ^{+}$). 

\clearpage
\subsection*{Question 4 - $20$ marks}
The language $L= \{ a^i b^j a^k | i,j,k \geq 0 \text{ and } j < k \}$ is not accepted by any DFA.
\begin{proof}
Towards a contradiction assume there is a DFA $$M=(Q,\{a,b\},\delta,s,F)$$
such that $L=L(M)$. Let $n=\#Q$ (the number of states). Consider the accepting configuration for $b^{n}a^{n+1}$,
$$\underbrace{s_0b^na^{n+1}\vdash s_1 b^{n-1}a^{n+1} \vdash \cdots \vdash s_n a^{n+1}}_{n+1 \text{ states required}} \vdash \cdots s_{2n+1}$$
where $s_0 = s$ and $s_n \in F$. Note that $n+1$ states are required to read $b^n$ (as indicated). However, by our assumption there is only $n$ states; therefore by Pigeonhole principle two of these states must be the same. That is, there is $s_i=s_j$ such that $0 \leq i < j \leq n$.

Now consider the accepting configuration,
\begin{align*}
b^{n+(j-i)}a^{n+1} \vdash \cdots & \vdash s_jb^{n-i}a^{n+1} \\
& \vdash s_i b^{n-i}a^{n+1} \vdash \cdots \vdash s_{2n+1}
\end{align*}
which shows that $b^{n+(j-i)}a^{n+1} \in L$ as well. Since $j-i>0$ this is a contraction.

\end{proof}
\small\emph{Note: a similar proof was given on page 53 of the lecture notes.}

\subsection*{Question 5 - $16$ marks}
\begin{enumerate}
\item[(a)] The set of all words that have the subword $aababcc$.
\begin{center}
\begin{VCPicture}{(-1,-2)(13,1)}
% states
\SmallState
\State{(0,0)}{A}\State{(1.5,0)}{B}\State{(3,0)}{C}\State{(4.5,0)}{D}
\State{(6,0)}{E}\State{(7.5,0)}{F}\State{(9,0)}{G}\FinalState{(10.5,0)}{H}
% initial--final
\Initial{A}
% transitions 
\EdgeL{A}{ B}{a}
\EdgeL{B}{C}{a}
\EdgeL{C}{D}{b}
\EdgeL{D}{E}{a}
\EdgeL{E}{F}{b}
\EdgeL{F}{G}{c}
\EdgeL{G}{H}{c}

\LoopE{H}{a,b,c}
\LoopS{A}{a,b,c}
%
\end{VCPicture}
\end{center}

\item[(b)] The set of all words such that the sixth symbol from the right-end is $c$.

\begin{center}
\begin{VCPicture}{(-1,-2)(13,1)}
% states
\SmallState
\State{(0,0)}{A}\State{(2,0)}{B}\State{(4,0)}{C}\State{(6,0)}{D}
\State{(8,0)}{E}\State{(10,0)}{F}\FinalState{(12,0)}{G}
% initial--final
\Initial{A}
% transitions 
\EdgeL{A}{ B}{c}
\EdgeL{B}{C}{a,b,c}
\EdgeL{C}{D}{a,b,c}
\EdgeL{D}{E}{a,b,c}
\EdgeL{E}{F}{a,b,c}
\EdgeL{F}{G}{a,b,c}

\LoopS{A}{a,b,c}
%
\end{VCPicture}
\end{center}
\vspace{2mm}
\small\emph{Note: consider how hard it would be to create a DFA to do this (as in Question 2 c); such a machine would require $3^6 = 729$ states for each length six substring of $\{a,b,c\}^*$. Yet, a NFA is no more powerful than a DFA (theoretically) as every DFA can be converted to a NFA (next question). Isn't that amazing!? }

\end{enumerate}

\subsection*{Question 6 - $13$ marks}
We use the iterative subset construction (lecture notes page 65), sets in bold indicate first appearance. \small\emph{This table is required for full credit}
\begin{center}
  \begin{tabular}{| c || c c c | }
  \hline
  \text{states} & a & b & c \\
    \hline \hline
  $\emptyset$ & $\emptyset$ & $\emptyset$ & $\emptyset$\\
  $\{0\}$ & $\{0\}$ & $\mathbf{\{0,1\}}$ & $\{0\}$\\
%  $\{1\}$ & $\emptyset$ & $\{2\}$ & $\emptyset$\\
%  $\{2\}$ & $\emptyset$ & $\emptyset$ & $\emptyset$\\
  $\{0,1\}$ & $\{0\}$ & $\mathbf{\{0,1,2\}}$ & $\{0\}$\\
%  $\{0,2\}$ & $\{0\}$ & $\{0,1\}$ & $\{0\}$\\
%  $\{1,2\}$ & $\emptyset$ & $\{2\}$ & $\emptyset$\\
  $\{0,1,2\}$ & $\{0\}$ & $\{0,1,2\}$ & $\{0\}$\\
    \hline
  \end{tabular}
\end{center}
\begin{center}

\begin{VCPicture}{(-2,-4)(4.5,3)}
% states
\LargeState
\ChgStateLabelScale{0.70}
\State[\{0\}]{(0,0)}{A} \State[\{0,1\}]{(3,0)}{B} \ \FinalState[\{0,1,2\}]{(1.5,-2.5)}{C}
% initial--final
\Initial{A}
% transitions 
\EdgeL{B}{C}{b} 
\EdgeR{A}{B}{b}
\ArcR{B}{A}{a,c}
\EdgeL{C}{A}{a,c}
\LoopN{A}{a,c}
\LoopE{C}{b}
%
\end{VCPicture}
\end{center}


\PutEnd
\end{document}