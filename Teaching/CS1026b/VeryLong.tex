% !TEX root = OurExam.tex
\clearpage
\section*{\S 4 Very Long Answer}
There is one question in this section with five parts. Each part is worth \emph{4 points} for a total of \emph{20 pionts}.

\subsection*{Question 12}
For this question you will create a new class that represents information about students in a university course. Use the following code as a starting point:

\begin{verbatim}
    public class Course {
        private String instructor;   //the name of the instructor
        private String course;       //the name of the course
        private Student[] students;  //an array for the students in the course
    }
\end{verbatim}

\noindent Specifications for \texttt{Student} class is part of the extra exam material (i.e. the ``cheat sheet'').

\subsubsection*{Part i.}
Write a constructor method for inclusion in the \texttt{Course} class with the header: 
$$\texttt{public Course (String instructorName, String courseName, int maxClassSize)}$$
that initializes the course in the obvious way (\texttt{maxClassSize} is used to initialize the \texttt{students} array).\\

\AnsBox{10cm}

\subsubsection*{Part ii.}
Write three methods for inclusion in the \texttt{Course} class with headers:
\begin{enumerate}
\item[1.] \texttt{  public String getInstructor() } 
\item[2.] \texttt{  public String getCourse() }
\item[3.] \texttt{  public int getMaxClassSize() }
\end{enumerate}
which return (respectively) the name of the instructor, the name of the course, and the maximum class size.\\

\AnsBox{17cm}

\subsubsection*{Part iii.}
Write a method for inclusion in the \texttt{Course} class with the header 
$$ \texttt{ public boolean enrollStudent(Student theStudent) } $$
that \emph{adds} a students to the course. More precisely, this method returns \texttt{false} if there is no room for the student in the course; otherwise it puts \texttt{theStudent} in the \texttt{students} array and returns \texttt{true}.
\\

\AnsBox{18cm} \clearpage

\subsubsection*{Part iv.}
Write a method for inclusion in the \texttt{Course} class with the header 
$$ \texttt{ public boolean removeStudent(Student theStudent) } $$
that \emph{removes} a student from the course. Return \texttt{true} if successful and \texttt{false} otherwise. (Removing a student that is not in the course should be considered a success).\\

\AnsBox{18cm} \clearpage

\subsubsection*{Part v.}
Write a method for inclusion in the \texttt{Course} class with the header 
$$ \texttt{ public int getNumberOfEnrolledStudents() } $$
that returns the number of enrolled students in the course.\\

\AnsBox{19cm} \clearpage