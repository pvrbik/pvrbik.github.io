% !TEX root =  OurExam.tex
\clearpage
\section*{\S 3 Long Answers}
Each solution in this section is worth \emph{10 points} for a total of \emph{40 points}.\\

\subsection*{Question 8}
\begin{enumerate}
\item Write an algorithm in pseudo-code (i.e. code mixed with English) which prints the following pattern when given \texttt{n}. Your method should work for any value of $n$ and not only on the examples given below.
Note that for a given \texttt{n}, the printed pattern has \texttt{n} rows
and \texttt{n} columns.

\begin{center}
\begin{tabular}{l  c l c l }
$n=3$ 		&\hspace{50px}\hfill	& $n=4$ 		&\hspace{50px}\hfill& $n=5$ \\
\hline
\texttt{..1}		&&\texttt{...1}		&& \texttt{....1} \\
\texttt{.22}		&&\texttt{..22}		&& \texttt{...22} \\
\texttt{333}	&&\texttt{.333}		&& \texttt{..333} \\
			&&\texttt{4444}		&& \texttt{.4444} \\
			&&				&& \texttt{55555} \\
\hline
\end{tabular}
\end{center}

\hspace{-0.55cm}\AnsBox{10cm} \clearpage

\item Write a java class method \texttt{static void PrintPattern (int n) } that prints the same pattern as in Part a.

\hspace{-0.55cm}\AnsBox{20cm}

\end{enumerate}

\clearpage
\subsection*{Question 9}
Write a java object method \texttt{void StairCase (int n, int s)} for inclusion in the \texttt{Turtle} class that takes as input $n$ and $s$ and draws a staircase with $n$ stairs of side length $s$. This illustrated for $n=2$ and $n=3$ below (but remember your code should work for \emph{any} value of $n$).

You may assume that drawing these stairs will never cause  your turtle to go out of its world's boundaries. 
\begin{center}
% !TEX root = OurExam.tex
\begin{tikzpicture}
% The graphic
%\draw[style=help lines,step=1cm] (0,0.5) grid (10,4.5);

\draw (1,1) -- (1,2);
\draw (1,2) -- (2,2);
\draw (2,2) -- (2,3);
\draw (2,3) -- (3,3);
\draw (1,1) -- (3,1);
\draw (3,3) -- (3,1);
\draw (1,1.5) node[left] {$s$};
\draw (1.5,2.0) node[above] {$s$};
\draw (2,.5) node[below] {$n=2$};


\draw (6,1) -- (6,2);
\draw (6,2) -- (7,2);
\draw (7,2) -- (7,3);
\draw (7,3) -- (8,3);
\draw (8,3) -- (8,4);
\draw (8,4) -- (9,4);
\draw (6,1) -- (9,1);
\draw (9,1) -- (9,4);

\draw (6,1.5) node[left] {$s$};
\draw (6.5,2.0) node[above] {$s$};
\draw (7.5,.5) node[below] {$n=3$};


\end{tikzpicture}
\\
\end{center}

\AnsBox{14cm}

\subsection*{Question 10}
Write a java class method \texttt{int sum (int n)} that finds the sum of the first $n$ positive integers (i.e. $1+2+3+\cdots+n.$) For example \texttt{sum(3)} returns $6$ because $1+2+3=6$ and \texttt{sum(8)} returns $36$ because $1+2+3+\cdots+8 = 36$. Your code should work for these examples \emph{and} any other positive integer $n$.\\

\AnsBox{18.5cm} \clearpage

\subsection*{Question 11}
Determining if a \texttt{Pixel} object is white is relatively straightforward: if a \texttt{Pixel} object's \texttt{red}, \texttt{green}, and \texttt{blue} components are all 255, then the \texttt{Pixel} object is white.

\begin{enumerate}
\item Write a java method: \texttt{boolean isWhitePixel (Pixel P)}, for inclusion in the \texttt{Picture} class that tests if a pixel \texttt{P} is white.\\

\hspace{-0.55cm}\AnsBox{18cm}

\item Using your method from Part a., write a java object method: \texttt{int countWhitePixels()}, for inclusion in the \texttt{Picture} class, that counts the number of white pixels in a picture. This method should take no parameters and return an integer that is the number of white pixels in the picture.\\

\hspace{-0.55cm}\AnsBox{14cm}

\item Suppose \texttt{Picture Sunset = new Picture(...)} has been properly initialized. What would you type in the interactions pane to determine the number of white pixels in \texttt{Sunset} (you may assume all classes have loaded properly).\\

\hspace{-0.55cm}\AnsBox{2cm}

\end{enumerate}
