\documentclass{seminar}
\nonstopmode

% \usepackage{geometry}
% \geometry{letterpaper,landscape}
% \usepackage{pictex}
%  pictex causes me to run out of dimensions (no room for new #3)
\usepackage{pictex}
\usepackage{dcpic}
\usepackage{semhelv}
\usepackage{color}
\usepackage{alltt}
\usepackage{latexsym}
\usepackage{amsfonts}
\usepackage{amsmath}
\usepackage{calc}
\usepackage{graphicx}
\usepackage{pifont}

\definecolor{CodeColor}{named}{BlueViolet}
% \definecolor{SystemColor}{named}{Mulberry}
% \definecolor{SystemColor}{named}{OliveGreen}
\definecolor{SystemColor}{named}{RedOrange}
\definecolor{MathColor}{named}{Magenta}
\definecolor{QuoteColor}{named}{Salmon}
\definecolor{LawColor}{named}{Green}
\definecolor{DefinitionColor}{named}{Red}
\definecolor{AlgorithmColor}{named}{RawSienna}
\definecolor{BackgroundColor}{named}{Yellow}
\definecolor{OtherColor}{named}{PineGreen} % {Maroon}
\definecolor{Green}{named}{OliveGreen}
% \definecolor{BackgroundColor}{cmyk}{0,0,0.6,0}

\newcommand{\semester}{Fall 2006}

\newcommand{\bs}{\ensuremath{\backslash}}
\newcommand{\zero}{\textcolor{MathColor}{\ensuremath{\mathbf{0}}~}}
\newcommand{\one}{\textcolor{MathColor}{\ensuremath{\mathbf{1}}~}}

\newcommand{\law}[1]{\textcolor{black}{(\textcolor{LawColor}{#1})}}

\newcommand{\sublaw}[1]{\textcolor{black}{(\textcolor{black}{#1})}}

\newcommand{\reason}[1]{\hspace{0.15in}
    \meq\hspace{0.075in}\law{#1}}

\newcommand{\subreason}[1]{\hspace{0.15in}
    \meq\hspace{0.075in}\sublaw{#1}}

\newcommand{\definition}[1]{\hspace{0.05in}
    \textcolor{DefinitionColor}{\textit{#1}\hspace{0.05in}}}

\newcommand{\meq}{\textcolor{MathColor}{\ensuremath{\equiv}}}
\newcommand{\bottom}{\textcolor{MathColor}{\ensuremath{\bot}}}
\newcommand{\dom}{\textcolor{MathColor}{\ensuremath{\mathit{dom~}}}}
\newcommand{\cod}{\textcolor{MathColor}{\ensuremath{\mathit{cod~}}}}
\newcommand{\rarrow}{\textcolor{MathColor}{\ensuremath{\rightarrow}}}

\newcommand{\textcd}[2][CodeColor]{\textcolor{#1}{\texttt{#2}}}
\newcommand{\tc}[2][CodeColor]{\textcolor{#1}{\texttt{#2}}}
\newcommand{\tcx}[2][OtherColor]{\textcolor{#1}{\texttt{#2}}}
\newcommand{\ts}[2][SystemColor]{\textcolor{#1}{\texttt{#2}}}
\newcommand{\textcdl}[2][CodeColor]{\textcolor{#1}{\hspace{0.1in}\texttt{#2}}}
\newcommand{\textcdr}[2][CodeColor]{\textcolor{#1}{\texttt{#2}\hspace{0.1in}}}
\newcommand{\textcdlr}[2][CodeColor]{\textcolor{#1}
{\hspace{0.1in}\texttt{#2}\hspace{0.1in}}}

\newcommand{\tmm}[1]{\textcolor{AlgorithmColor}{\ensuremath{#1}}}


\newcommand{\textmath}[1]{\textcolor{MathColor}{\ensuremath{#1}}}
\newcommand{\tm}[1]{\textcolor{MathColor}{\ensuremath{#1}}}
\newcommand{\tmcc}[1]{\textcolor{CodeColor}{\ensuremath{#1}}}
\newcommand{\tmtt}[1]{\textcolor{MathColor}{\texttt{#1}}}
\newcommand{\unmath}[1]{\textcolor{black}{\mathrm{#1}}}
\newcommand{\tmi}[1]{\textcolor{MathColor}{\ensuremath{\mathit{#1}}}}
\newcommand{\tmr}[1]{\textcolor{MathColor}{\ensuremath{\mathrm{#1}}}}
\newcommand{\tmb}[1]{\textcolor{MathColor}{\ensuremath{\mathbf{#1}}}}


\newcommand{\tbr}[1]{\textcolor{red}{\textbf{#1}}}

\newcommand{\bla}[1]{\textcolor{black}{#1}}
\newcommand{\gre}[1]{\textcolor{Green}{#1}}
\newcommand{\red}[1]{\textcolor{red}{#1}}
\newcommand{\blu}[1]{\textcolor{blue}{#1}}
\newtheorem{lemma}{Lemma}
\newtheorem{theorem}{Theorem}
\newtheorem{corollary}{Corollary}
\newenvironment{proof}{\textbf{Proof:}}{\hfill $\Box$}

\slideframe{none}

\centerslidesfalse

\newlength{\tempwidth}

\newenvironment{cdiag}
   {\vspace{0.25in}
    \begin{center}
    \setlength{\tempwidth}{\linewidth}
    \addtolength{\tempwidth}{-2\columnsep}
    \begin{minipage}[t]{\tempwidth}
    \begindc{\commdiag}[10]
   }
   {\enddc
    \end{minipage}
    \end{center}
    \vspace{0.25in}
   }

\newenvironment{cdiag2}
   {\vspace{0.25in}
    % \begin{center}
    \setlength{\tempwidth}{\linewidth/2}
    \begin{minipage}[t]{\tempwidth}
    \begindc{\commdiag}[10]
   }
   {\enddc
    \end{minipage}
    % \end{center}
    \vspace{0.25in}
   }

\newenvironment{udg}
   {\vspace{0.25in}
    \begin{center}
    \setlength{\tempwidth}{\linewidth}
    \addtolength{\tempwidth}{-2\columnsep}
    \begin{minipage}[t]{\tempwidth}
    \begindc{2}[10]
   }
   {\enddc
    \end{minipage}
    \end{center}
    \vspace{0.25in}
   }

\newenvironment{codenott}[1][CodeColor]
   {\color{#1}
    % \begin{center}
    \setlength{\tempwidth}{\linewidth}
    \addtolength{\tempwidth}{-2\columnsep}
    \hspace*{0.25in}
    \begin{minipage}[t]{\tempwidth}
    \begin{tabbing}
    \hspace{0.075in} \= \hspace{0.075in} \= \hspace{0.075in} \= \hspace{0.075in} \= \hspace{0.075in} \= \kill
   }
   {\end{tabbing}
    \end{minipage}
    % \end{center}
    \color{black}
   }

\newenvironment{codenott2}[1][CodeColor]
   {\color{#1}
    % \begin{center}
    \setlength{\tempwidth}{\linewidth/3}
    \hspace*{0.25in}
    \begin{minipage}[b]{\tempwidth}
    \begin{tabbing}
    \hspace{0.075in} \= \hspace{0.075in} \= \hspace{0.075in} \= \hspace{0.075in} \= \hspace{0.075in} \= \kill
   }
   {\end{tabbing}
    \end{minipage}
    % \end{center}
    \color{black}
   }

\newenvironment{laws}
   {\setlength{\tempwidth}{\linewidth}
    \addtolength{\tempwidth}{-2\columnsep}
    \hspace*{0.25in}
    \begin{minipage}[t]{\tempwidth}
    \begin{tabular}{lr}
   }
   {\end{tabular}
    \end{minipage}
   }

\newenvironment{code}[1][CodeColor]
   {\color{#1}
    % \begin{center}
    \setlength{\tempwidth}{\linewidth}
    \addtolength{\tempwidth}{-2\columnsep}
    \hspace*{0.25in}
    \begin{minipage}[t]{\tempwidth}
    \begin{alltt}
    \begin{tabbing}
    \hspace{0.075in} \= \hspace{0.075in} \= \hspace{0.075in} \= \hspace{0.075in} \= \hspace{0.075in} \= \kill
   }
   {\end{tabbing}
    \end{alltt}
    \end{minipage}
    % \end{center}
    \color{black}
   }

\newenvironment{mcode}[1][MathColor]
   {\color{#1}
    % \begin{center}
    \setlength{\tempwidth}{\linewidth}
    \addtolength{\tempwidth}{-2\columnsep}
    \hspace*{0.25in}
    \begin{minipage}[t]{\tempwidth}
    \[\begin{array}{lrll}}
   {\end{array}\]
    \end{minipage}
    % \end{center}
    \color{black}
   }

\newcommand{\mF}{\textcolor{MathColor}{\ensuremath{\tc{fmap}_{F}~}}}
\newcommand{\Id}{\textcolor{MathColor}{\ensuremath{\mathit{Id}}}}
\newcommand{\nt}{\textcolor{MathColor}{\ensuremath{\stackrel{.}{\rightarrow}}}}
\newcommand{\List}{\textcolor{MathColor}{\ensuremath{\tc{[}?\tc{]}}}}
\newcommand{\LL}{\textcolor{MathColor}{\ensuremath{\tc{[[}?\tc{]]}}}}
\newcommand{\LA}{\textcolor{MathColor}{\ensuremath{\tc{[}A\tc{]}}}}
\newcommand{\LB}{\textcolor{MathColor}{\ensuremath{\tc{[}B\tc{]}}}}
\newcommand{\LLA}{\textcolor{MathColor}{\ensuremath{\tc{[[}A\tc{]]}}}}
\newcommand{\LLB}{\textcolor{MathColor}{\ensuremath{\tc{[[}B\tc{]]}}}}
\newcommand{\lam}{\textcolor{CodeColor}{\ensuremath{\lambda}}}
\newcommand{\redu}{\textcolor{MathColor}{\ensuremath{\longrightarrow}}}
\newcommand{\Has}{\textit{Haskell}\,}

\newcommand{\ack}{I would like to thank Tai Meng for many corrections
and suggestions for improvements to these notes.  Of course, the remaining
errors are my fault.}


% \newpagestyle{lectures}{CMPT 383 \hfill Comparative Programming Languages
% \hfill \semester}{Functional Programming
% \hfill F. Warren Burton \hfill \thepage}

\newpagestyle{lectures}{\makebox[1in][l]{CMPT 481/731} \hfill Functional Programming
\hfill \makebox[1in][r]{\semester}}{\makebox[2in][l]{The Lambda Calculus} \hfill F. Warren
Burton \hfill \makebox[2in][r]{\thepage}}

\pagestyle{lectures}

\begin{document}
\begin{slide}

\begin{center}
\begin{LARGE}
Functional Programming
\end{LARGE}\\[3ex]
\end{center}


Sometimes people divide programming languages into various paradigms,
where a paradigm is ``a theoretical framework''.

These might be:
\begin{itemize}
\item Procedural,
\item Object-Oriented,
\item Functional, and
\item Logic
\end{itemize}

These are neither exhaustive or exclusive.  A lot of research is being done
on how to combine the best aspects of different paradigms.

\newslide

\section*{What is Functional Programming?}

The book (p. 471) mentions the following properties:
\begin{itemize}
\item The uniform view of programs as functions,
\item The treatment of functions as data,
\item The limitation of side effects, and
\item The use of automatic memory management
\end{itemize}

I would add that functional languages support \definition{higher order 
functions}
(functions that take other functions as arguments or compute new
functions as results).

\newslide

From the comp.lang.functional FAQ:

{\color{QuoteColor}
What is a ``functional programming language''?

Opinions differ, even within the functional programming community,
on the precise definition of what constitutes a functional programming
language. However, here is a definition that, broadly speaking,
represents the kind of languages that are discussed in
comp.lang.functional:

Functional programming is a style of programming that emphasizes
the evaluation of expressions, rather than execution of commands.
The expressions in these language are formed by using functions to
combine basic values.  A functional language is a language that
supports and encourages programming in a functional style.
}

\newslide

A functional program is a set of recursive mathematical equations
defining the output in terms of the input.

A pure functional language contains no assignment statements.  A
variable never changes its value.

However, there may be a number of different copies of a variable,
at different levels of recursion, each with its own value.

\newslide

Functional languages do not require a strong type system.  Scheme does not
have strong typing.

However, most modern functional programming languages do have a strong
polymorphic type system.  Many users and designers of these systems
regard the strong typing as being even more important than the functional
nature of their languages.

The type systems of languages such as Pascal and Ada were not sufficient
for functional languages.  A polymorphic type system was invented to
meet the needs of functional programming languages.

\newslide

\section*{Getting Started with Hugs}

\begin{code}[SystemColor]
[burton@amadeus Examples]% \textcd{hugs}\\
\color{red}
{\LARGE ...}\\
Reading file "/USO/local/lib/hugs/lib/Prelude.hs":\\
                   \\
Hugs session for:\\
/usr/local/lib/hugs/lib/Prelude.hs\\
Type :? for help\\
Prelude> \color{CodeColor}1+3\\
4\\
Prelude>\textcd{ sum . map square \$ [1..10]}\\
ERROR - Undefined variable "square"\\
Prelude>\textcd{ sum . map (^2) \$ [1..10]}\\
385\\
Prelude>
\end{code}

\newslide

In Hugs, expressions can be entered interactively,
but declarations and definitions must be placed in
a file and loaded.

Typing \textcdlr{:?} in Hugs will produce a list of
possible commands.

Finally, typing \textcdlr{:quit} will exit Hugs.

\newslide

\begin{code}
\color{SystemColor}
[burton@amadeus Examples]% \textcd{head Sect11_6.hs} \\
\color{SystemColor}
-- Collected Haskell code of Section 11.6 from\\
\color{SystemColor}
-- Kenneth C. Louden, Programming Languages\\
\color{SystemColor}
-- Principles and Practice 2nd Edition\\
\color{SystemColor}
-- Copyright (C) Brooks-Cole/ITP, 2003\\
\\
\color{SystemColor}
-- page 513\\
\\
\color{SystemColor}
fact 0 = 1\\
\color{SystemColor}
fact n = n * fact (n - 1)\\
\color{SystemColor}
square x = x * x\\
\color{SystemColor}
[burton@amadeus Examples]%
\end{code}

\newslide
\begin{code}[SystemColor]
Prelude>\color{CodeColor} :load Sect11\_6.hs\\
Reading file "Sect11\_6.hs":\\
                   \\
Hugs session for:\\
/usr/local/lib/hugs/lib/Prelude.hs\\
Sect11\_6.hs\\
Main>\textcd{ sum . map square \$ [1..10]}\\
385\\
Main>\textcd{ fact 40}\\
815915283247897734345611269596115894272000000000\\
Main> 
\end{code}

\newslide

\section*{Haskell Basics}

Many of the ML examples in the book apply more or less to Haskell.
Some examples of syntactic differences to look out for are the following:

\begin{center}
\begin{tabular}{lll}
ML & Haskell &
Purpose\\
\hline
\textcd{::} & \textcd{:} & cons\\
\textcd{:} & \textcd{::} & Has type\\
\textcd{int} & \textcd{Int} & integer\\
\textcd{int * int} & \textcd{(Int,Int)} & integer pair\\
\textcd{int list} & \textcd{[Int]} & List of integers\\
\textcd{'a} & \textcd{a} & Type variable\\
\textcd{fun} &  & Keyword for function definition\\
\textcd{val} &  & Keyword for value definition\\
\textcd{datatype} & \textcd{data} & Keyword for algebraic
datatype definition\\
\end{tabular}
\end{center}

\newslide

\subsection*{Predefined Types}

Haskell has a number of predefined types including:

\begin{code}
'A'  :: Char\\
"ABC"  :: String \color{black} \textrm{or} \color{CodeColor} [Char]\\
False :: Bool\\
5 :: Int\\
5 :: Integer\\
5.0 :: Float\\
5.0 :: Double\\
() :: () \color{black} \textrm{similar to type} void\\
(True, 'A') :: (Bool, Char) \color{black} \textrm{and tuple types in general}\\
[1,2,3] :: [Int]
\end{code}

\newslide

In most cases, you are not required to declare the types of variables
and functions in Haskell.

However, if you declare the type of almost everything,
then error messages are much more likely to be meaningful.
I will not answer questions about error messages for
programs where type declarations are not given, at least
for functions.

Also, declaring types helps avoid some special cases.

\newslide

\begin{code}[SystemColor]

[burton@alison Examples]% \textcd{cat CMPT383.hs} \\
a = 5\\
[burton@alison Examples]% \textcd{hugs CMPT383.hs} \\
\color{red}{\LARGE ...}\\

Main> \textcd{5 + 3.2}\\
8.2
\end{code}

\newslide

\begin{code}[SystemColor]
Main> \textcd{a + 3.2}\\
ERROR - Illegal Haskell 98 class constraint in \\
inferred type \\
*** Expression : a + 3.2\\
*** Type       : Fractional Integer => Integer\\
\\
Main> \textcd{:set +t}\\
Main> \textcd{a}\\
5 :: Integer\\
Main> 
\end{code}

\newslide

\subsection*{Function Types}

Consider:

\begin{code}
f, g :: Integer -> Integer\\
f x = x + 3\\
g y = f y + f (y + 1)
\end{code}

The first line declares that both \textcd{f}
and \textcd{g} are functions from \textcd{Integer}s
to \textcd{Integer}s.

Notice that we do not need parenthesis around the arguments
or parameters of a function.

Function application binds more tightly that any other operation
(e.g. \textcd{+} or \textcd{`mod`}) and is represented by juxtaposition.

\newslide

Function application is left associative, so \textcdlr{f g x} means
\textcdlr{(f g) x}.

\begin{code}
f :: (a -> a) -> (a -> a)\\
f h a = h (h a)\\
\\
g :: Double -> Double\\
g x = sqrt x + 1.0
\end{code}

Quiz:  \textcolor{QuoteColor}{What is the value of:
\textcdlr{f g 9.0}?}

The function \textcd{f} is a higher order function that both takes a
function as an argument and returns a function as a result.

\newslide

\subsection*{Curried Functions and Higher Order Functions}

If we load a file with the following definitions
\begin{code}
plus :: (Int, Int) -> Int\\
plus (a,b) = a + b\\
\\
plus\_curried :: Int -> Int -> Int\\
plus\_curried a b = a + b
\end{code}

\newslide

then we can evaluate the following expressions
\begin{code}[SystemColor]
Main> \textcd{plus (3, 4)}\\
7\\
Main> \textcd{plus\_curried 3 4}\\
7\\
Main>\textcd{(+) 3 4}\\
7\\
Main> \textcd{3 `plus\_curried` 4}\\
7\\
Main>
\end{code}

A function that takes it arguments one at a time, like \textcd{plus\_curried},
is said to be \definition{curried} (see p. 504 of \textit{Louden}).

\newslide

There are two different ways to think of curried functions.

A useful, informal way to think of a curried function
is as a function that takes
multiple arguments one at a time.

Actually, a curried function is really a higher order function that
takes a single argument and returns a function.

The operator \textcd{->} is right associative, so
\begin{code}
plus\_curried :: Int -> Int -> Int
\end{code}
means
\begin{code}
plus\_curried :: Int -> (Int -> Int)
\end{code}

Formally, all Haskell functions take exactly one argument and return
a single result.  The argument and/or result may be a data structure
or a function.

\newslide

Assume that
\textcdl{f :: Int -> (Int -> Int)}
\newline
so \textcdlr{g :: Int -> Int} in the following examples.
\begin{code}[SystemColor]
Main> \textcd{let f = (+); g = f 3 in f (g 4) (g 6)}\\
16\\
Main> \textcd{let f = (+); g = f 3 in (f.g) 4 \$ g 6}\\
16\\
Main>
\end{code}

The second example uses the infix function composition operator, \textcd{.},
and an infix, low priority, explicit function application operator,
\textcd{\$}.

Note:  We could have included type declarations in the above expressions.
\begin{code}
let f :: Int -> Int; f x = x + 1 in f 3
\end{code}

My excuse for not doing this is
that I am trying to fit examples onto slides.  In your
homework you won't have this excuse.

\newslide

\subsection*{Some Useful Higher Order Functions}
On p. 491, \textit{Louden} defines the function \textcd{map} in Scheme.
This is a standard and frequently used function in Haskell.  It is defined
in the \textit{Haskell Standard Prelude} as follows:

\begin{code}
map :: (a -> b) ->[a] -> [b]\\
map f []     = []\\
map f (x:xs) = f x : map f xs
\end{code}

The result of evaluating \textcdlr{map f [x1, x2, x3]}
\newline
is \textcdlr{[f x1, f x2, f x3]}.

The higher order function, \textcd{map}, encapsulates what would be a
control structure in a procedural language.

\newslide

In a procedural language, such as C, with just recursion and some sort of conditional
branching you can do
anything you can do with other control structures.

Similarly, in a functional language, all you need is recursion and
some form of branching to enable any desired control over program
execution.

Just as common patterns of control are encapsulated in control structures in
procedural languages, these common patterns of control are captured by
standard higher order functions in a functional language.

However, while control structures must be part of the fundamental
structure of the syntax of a procedural language, higher order functions
can be implemented in a functional language using more primitive constructs.

This means that it is easy to define additional ``control structures'',
for example to use with particular data structures.

\newslide

We can think of the \textcd{map} function in a couple of different ways.

Informally, we can think of \textcd{map} as a function that takes two
arguments.

However, we can also think of \textcd{map} as a function that raises
a function of type
\newline
\textcdr{a -> b} for any types \textcd{a} and
\textcd{b} to a function of type \textcdlr{[a] -> [b]}.  That
is, \textcd{map} lifts a function from the space of ordinary type up to
the subspace of list types.
\begin{cdiag}
\obj(1,1)[A]{\textcd{a}}
\obj(10,1)[B]{\textcd{b}}
\obj(1,6)[FA]{\textcd{[a]}}
\obj(10,6)[FB]{\textcd{[b]}}
\mor{A}{B}{\textcd{f}}
\mor{FA}{FB}{\textcd{map f}}
\end{cdiag}

In functional programming, it is common to lift functions.

\newslide

Let \textcd{T} be a type constructor and
\begin{code}
fmap :: (a -> b) -> (T a -> T b)
\end{code}
be a higher
order (lifting) function.  If
\begin{eqnarray*}
\textcd{fmap id} & \textcd[MathColor]{$\equiv$} & \textcd{id}\\
\textcd{fmap (f.g)} & \textcd[MathColor]{$\equiv$} & \textcd{fmap f . fmap g}
\end{eqnarray*}
whenever the types make these equations valid,
then the \textcd{T} and \textcd{fmap} together are called a
\definition{functor}.

We can reresent these two equations with \definition{commutative diagrams}.

\newslide

\begin{cdiag}
\obj(0,4)[A]{\textcd{a}}
\obj(10,0)[B]{\textcd{b}}
\obj(20,4)[C]{\textcd{c}}
\obj(0,14)[FA]{\textcd{T a}}
\obj(10,10)[FB]{\textcd{T b}}
\obj(20,14)[FC]{\textcd{T c}}
\obj(4,20)[TA1]{\textcd{T a}}
\obj(16,20)[TA2]{\textcd{T a}}
\mor(5,21)(15,21){\textcd{fmap id}}
\mor(5,19)(15,19){\textcd{id}}[-1,0]
\mor{A}{B}{\textcd{g}}
\mor{B}{C}{\textcd{f}}
\mor{A}{C}{\textcd{f.g}}
\mor{FA}{FB}{\textcd{fmap g}}[-1,0]
\mor{FB}{FC}{\textcd{fmap f}}[-1,0]
\mor{FA}{FC}{\textcd{fmap (f.g)}}
\end{cdiag}

\newslide

In a commutative diagram, nodes represent types and directed edges
represent functions.

A path in a commutative diagram represents a composition of functions.

Any two paths between a pair of nodes must represent equivalent functions.

The top two diagrams on the previous page illustrate the two functor
laws.  The bottom diagram follows from the definition of function
composition, but when lifted gives the middle diagram.

\newslide

A higher order function for working with lists that is
even more powerful than \textcd{map} is the
\textcd{foldr} function:
\begin{code}
foldr            :: (a -> b -> b) -> b -> [a] -> b\\
foldr f z []     =  z\\
foldr f z (x:xs) =  f x (foldr f z xs)
\end{code}

All elements of a list must have the same type.  Let's consider a list of
\textcd{Int}s.

Any (finite and total) \textcd{[Int]} is constructed using
the constructors \textcd{[]} and \textcd{:} and, of course, \textcd{Int}s.

The list \textcdlr{[1,2,3,4]} is really the list
\textcdlr{1:2:3:4:[]}.

The expression \textcdlr{foldr ($\oplus$) z xs} evaluates to the result
of replacing every \textcd{:} is \textcd{xs} with a \textcd{$\oplus$} and
the \textcd{[]} at the end of \textcd{xs} with \textcd{z}.

Hence, \textcdlr{foldr (+) 0 [1,2,3,4]} evaluates to
\newline
\textcdr{1 + (2 + (3 + (4 + 0)))} or \textcdlr{10}.


\newslide

By using infix notation, we can rewrite the
\textcd{foldr} function as follows:
\begin{code}
foldr            :: (a -> b -> b) -> b -> [a] -> b\\
foldr f z []     =  z\\
foldr f z (x:xs) =  x `f` (foldr f z xs)
\end{code}

This should make it easier to see our previous claim, restated below.

The expression \textcdlr{foldr ($\oplus$) z xs} evaluates to the result
of replacing every \textcd{:} is \textcd{xs} with a \textcd{$\oplus$} and
the \textcd{[]} at the end of \textcd{xs} with \textcd{z}.

Hence,
\newline
\textcdr{foldr (+) 0 [1 ,~~2 ,~~3 ,~~4]}, or
\newline
\textcdr{foldr (+) 0 (1 :~(2 :~(3 :~(4 :~[]))))}, produces
\newline
\verb|            |\textcdr{(1 + (2 + (3 + (4 + 0 ))))} or \textcdlr{10}.

\newslide

Before looking at more examples,
let's review a bit of notation that has been slipped into earlier example,
and add some more.

Putting an operator inside parentheses, e.g. \textcd{(+)}, makes it
behave as an ordinary function, and putting a function name in back
quotes, e.g. \textcd{`f`} makes it act like an infix operator.

You can also include either argument of a binary operator within the
parentheses.

\begin{eqnarray*}
\textcd{($\oplus$) a b} & \meq &\textcd{a $\oplus$ b}\\
\textcd{(a $\oplus$) b} & \meq &\textcd{a $\oplus$ b}\\
\textcd{($\oplus$ b) a} & \meq &\textcd{a $\oplus$ b}\\
\textcd{a `f` b} & \meq &\textcd{f a b}
\end{eqnarray*}

\newslide

You can declare the priority and associativity 
of operators and functions used as operators.
 
When operators are enclosed in parentheses, or functions are not enclosed
in back quotes, then their priority and associativity is ignored.

That is, only infix operators, including functions inside back quotes,
have priorities and associativity.

The following examples come from the Standard Prelude.
\begin{code}
infixr 9  .\\
infixr 8  ^, ^^, **\\
infixl 7  *, /, `quot`, `rem`, `div`, `mod`\\
infixl 6  +, -\\
\\
infixr 0  \$
\end{code}

\newslide

Syntax is important.

Since expressions are so fundamental in functional programming languages, 
being able to define new operators and specify their priority and
associativity, as well as being able to use function names in
an infix position, is handy.

However, this doesn't really have anything to do with functional
programming, and could be supported in almost any sort of language.

\newslide

An active area of research, well funded in the U.S., is 
\textit{domain specific
languages}.

Prof.~Paul Hudak, of Yale University, has been able to get a lot of mileage
out of Haskell.

He has designed a number of new, domain specific languages, which are
really just Standard Haskell, unmodified.

Hudak defines new operators that are well suited to an application area,
and captures common patterns of control for the application area, or for
the commom data structures of the application area, in new higher order
functions.  In some cases, the new higher order functions take the form
of operators.

This yields a new and powerful domain specific notation, which any Haskell
compiler can understand without modification.

\newslide

Let us now return to some examples using the \textcd{foldr} function.

First, let us reconsider our claim that \textcd{foldr} is even more
powerful than \textcd{map}

We could define \textcd{map} as follows.

\begin{code}
map f = foldr ((:).f) []
\end{code}

Try computing \textcdlr{map (2*) [1..3]} with both definitions of
\textcd{map}.

In this case, the earlier definition of \textcd{map} is actually easier
to understand.

\newslide

Some useful functions from the Standard Prelude include the following.
\begin{codenott}
\texttt{(\$)~~~~~~~~~::~(a~->~b)~->~a~->~b}\\
\texttt{f~\$~x~~~~~~~=~~f~x}\\[1.5ex]

\texttt{(.)~~~~~~~~~::~(b~->~c)~->~(a~->~b)~->~a~->~c}\\
\texttt{f~.~g~~~~~~~=~}~~$\backslash$~\texttt{x~->~f~(g~x)}\\[1.5ex]

\texttt{id~~~~~~~~~~::~a~->~a}\\
\texttt{id a~~~~~~~~=~~a}\\[1.5ex]

\texttt{const~~~~~~~::~a~->~b~->~a}\\
\texttt{const a b~~~=~~a}\\[1.5ex]

\texttt{flip~~~~~~~~::~(a~->~b~->~c)~->~b~->~a~->~c}\\
\texttt{flip~f~x~y~~=~~f~y~x}
\end{codenott}

An equivalent defining equation for \textcd{.} would be:
\begin{code}
(f . g) x = f (g x)
\end{code}

\newslide

Other functions that can be defined using \textcd{foldr}
include the following.
\begin{code}
sum, product :: Num a => [a] -> a\\
sum = foldr (+) 0\\
product = foldr (*) 1\\
factorial :: Integer -> Integer\\
factorial n = product [1..n]
\end{code}

\newslide

Other functions that can be defined using \textcd{foldr}
include the following.
\begin{code}
sum, product :: Num a => [a] -> a\\
sum = foldr (+) 0\\
product = foldr (*) 1\\
factorial :: Integet -> Integer\\
factorial n = product [1..n]\\
append :: [a] -> [a] -> [a]\\
append = flip (foldr (:))\\
concat :: [[a]] -> [a]\\
concat = foldr append []
\end{code}

\newslide

Other functions that can be defined using \textcd{foldr}
include the following.
\begin{code}
sum, product :: Num a => [a] -> a\\
sum = foldr (+) 0\\
product = foldr (*) 1\\
factorial :: Integet -> Integer\\
factorial n = product [1..n]\\
append :: [a] -> [a] -> [a]\\
append = flip (foldr (:))\\
concat :: [[a]] -> [a]\\
concat = foldr append []\\
reverse :: [a] -> [a]\\
reverse = foldr (flip append) [] . map (:[])
\end{code}

\newslide

Other functions that can be defined using \textcd{foldr}
include the following.
\begin{code}
sum, product :: Num a => [a] -> a\\
sum = foldr (+) 0\\
product = foldr (*) 1\\
factorial :: Integet -> Integer\\
factorial n = product [1..n]\\
append :: [a] -> [a] -> [a]\\
append = flip (foldr (:))\\
concat :: [[a]] -> [a]\\
concat = foldr append []\\
reverse :: [a] -> [a]\\
reverse = foldr (flip append) [] . map (:[])\\
length :: [a] -> Int\\
length = sum . map (const 1)
\end{code}

In some cases there are better ways to implement these functions.

\newslide

\textcd{foldr} groups list elements from the right for evaluation.
To group from the left, we can use \textcd{foldl} instead.

Informally,

\begin{codenott}
\textcd{foldl ($\oplus$) z [x1, x2, ... , xn]}
\meq\\
\> \textcd{(...((z {$\oplus$} x1) {$\oplus$} x2) ... ) {$\oplus$} xn}
\end{codenott}

whereas

\begin{codenott}
\textcd{foldr ($\oplus$) z [x1, x2, ... , xn]}
\meq\\
\> \textcd{x1 {$\oplus$} (x2 {$\oplus$} ( ... (xn {$\oplus$} z)...))}
\end{codenott}

so we have the following law:

\begin{laws}
\textcd{foldl ($\oplus$) z xs}
\meq\\
\hspace{0.075in}\textcd{foldr (flip ($\oplus$)) z (reverse xs)} & \law{foldl-foldr.1}\\
\end{laws}

\newslide

Let us revisit some of our previous examples that used \textcd{foldr}.

In the Haskell Standard Prelude, \textcd{sum} and \textcd{product} are defined
using \textcd{foldl} instead of \textcd{foldr}.  In some cases this may
result in a more space efficient program.  We will reconsider this
when we talk about lazy evaluation.

\newslide

The \textcd{append} function is provided in the Haskell Standard Prelude
as an infix operator, \textcd{++}, defined as follows:

\begin{code}
(++) :: [a] -> [a] -> [a]\\
[]     ++ ys = ys\\
(x:xs) ++ ys = x : (xs ++ ys)
\end{code}

We can easily prove that \textcd{++} is equivalent to
our earlier \textcd{append} function.

We will consider three cases, depending on the form of the first
argument to \textcd{append} (and \textcd{++}).  These cases will be
exclusive and exhaustive.

\newslide

We will use the following laws, which come from the corresponding
function definitions:

\begin{laws}
\textcd{append \meq ~flip (foldr (:))}               & \sublaw{append.1} \\
\textcd{foldr f z []     \meq ~z}                   & \law{foldr.1} \\
\textcd{foldr f z (x:xs) \meq ~f x (foldr f z xs)}   & \law{foldr.2} \\
\textcd{[]     ++ ys \meq ~ys}                       & \law{++.1} \\
\textcd{(x:xs) ++ ys \meq ~x : (xs ++ ys)}           & \law{++.2} \\
\textcd{flip f x y \meq ~f y x}                      & \law{flip.1}\\
\textcd{(f.g) x \meq ~f (g x)}                       & \law{(.).1}\\
\textcd{id x \meq ~x}                                & \law{id.1}
\end{laws}

\newslide

We will start by considering the case
where the first argument is to \textcd{append}
is \textcd{[]}.

\begin{lemma}\label{LemmaAppend1} 
\textcdlr{append [] ys = ys}.
\end{lemma}

\begin{codenott}
\textcd{append [] ys}\\
\subreason{append.1}\\
\textcd{flip (foldr (:)) [] ys}\\
\reason{flip.1}\\
\textcd{foldr (:) ys []}\\
\reason{foldr.1}\\
\textcd{ys}\\
\end{codenott}

\newslide

Next, we will consider the case
where the first argument is to \textcd{append} is of the
form \textcd{(x:xs)} (i.e. a nonempty list).

\begin{lemma}\label{LemmaAppend2}
\textcdlr{append (x:xs) ys = x : (append xs ys)}.
\end{lemma}

\begin{codenott}
\textcd{append (x:xs) ys}\\
\subreason{append.1}\\
\textcd{flip (foldr (:)) (x:xs) ys}\\
\reason{flip.1}\\
\textcd{foldr (:) ys (x:xs)}\\
\reason{foldr.2}\\
\textcd{x : foldr ys xs}\\
\reason{flip.1}\textcolor{black}{~~ used backwards}\\
\textcd{x : flip (foldr (:)) xs ys}\\
\subreason{append.1}\textcolor{black}{~~ used backwards}\\
\textcd{x : append xs ys}\\
\end{codenott}

\newslide

From Lemmas \ref{LemmaAppend1} and \ref{LemmaAppend2} we see that
changing the original definition of \textcd{append},

\begin{code}
append = flip (foldr (:))
\end{code}

to

\begin{code}
append [] ys = ys\\
append (x:xs) ys = x : (append xs ys)
\end{code}

will preserve the meaning of the function in the
cases where the first argument is either an empty list or a nonempty list.

One other case must be considered in order to justify this change.

\newslide

We must consider the case where the first argument of \text{append}
is completely undefined.

Let us consider what happens with our original definition of
\textcd{append}.

\begin{codenott}
\textcd{append~\bottom ~ys}\\
\subreason{append.1}\\
\textcd{flip (foldr (:))~\bottom ~ys}\\
\reason{flip.1}\\
\textcd{foldr (:)~\bottom ~ys}\\
\reason{case exhaustion on the definition of \textcd{foldr}}\\
\bottom
\end{codenott}

\newslide

Similarily, if we consider the proposed new definition of \textcd{append}

\begin{code}
append [] ys = ys\\
append (x:xs) ys = x : (append xs ys)
\end{code}

we get

\begin{codenott}
\textcd{append~\bottom ~ys}\\
\reason{case exhaustion on the new definition of \textcd{append}}\\
\bottom
\end{codenott}

\newslide

We can now use our new definition of \textcd{append}

\begin{code}
append [] ys = ys\\
append (x:xs) ys = x : (append xs ys)
\end{code}

in place of our
original definition.
However, this is identical to the definition of \textcd{++}, except for
the name change and the use of infix notation.

\begin{code}
(++) :: [a] -> [a] -> [a]\\
[]     ++ ys = ys\\
(x:xs) ++ ys = x : (xs ++ ys)
\end{code}

Therefore,

\begin{codenott}
\textcd{append~\meq~(++)}
\end{codenott}

\newslide

Minor syntatic note:

The Haskell Standard Prelude declares

\begin{code}
infixr 5 ++
\end{code}

while the default fixity is \textcd{infixl 9}.  Therefore, using
\textcd{`append`} as an infix operator in place of \textcd{++} may
be a syntatic error or even produce the wrong result.

\newslide

\ack

\end{slide}

\end{document}
